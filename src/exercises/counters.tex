
Zum Zählen der Aufgabennummern werden zuerst Zähler definiert. Anschließend Hilfs-Makros erstellt, welche im späteren Verlauf zur Anzeige der Punkte und zum Zusammensetzen der Überschriften dienen.

\begin{macro}{\l__edu_exepoints_tl}
\begin{macro}{\l__edu_exelabeltext_tl}
\begin{macro}{\l__edu_subexelabeltext_tl}
\begin{MacroCode}{class}
\newcounter{exercisecounter}
\newcounter{subexercisecounter}[exercisecounter]
\newcounter{multiexecounter}

\renewcommand{\thesubexercisecounter}{\theexercisecounter.\arabic{subexercisecounter}}

\tl_new:N \l__edu_exepoints_tl        % For optional number points
\tl_new:N \l__edu_exelabeltext_tl     % Concatenation of the exercise-label
\tl_new:N \l__edu_subexelabeltext_tl  % Concatenation of the subexercise-label

\end{MacroCode}
\end{macro}
\end{macro}
\end{macro}

\begin{macro}{\l__edu_sollabeltext_tl}
\begin{macro}{\l__edu_subsollabeltext_tl}
\begin{MacroCode}{class}
\newcounter{solutioncounter}
\newcounter{subsolutioncounter}[solutioncounter]
\newcounter{multisolcounter}

\renewcommand{\thesubsolutioncounter}{\thesolutioncounter.\arabic{subsolutioncounter}}

\tl_new:N \l__edu_sollabeltext_tl     % Concatenation of the exercise-label
\tl_new:N \l__edu_subsollabeltext_tl  % Concatenation of the subexercise-label

\end{MacroCode}
\end{macro}
\end{macro}

\begin{macro}{\rstexe}
\begin{macro}{\rstsubexe}
\begin{macro}{\rstmultiexe}
Mit den folgenden Makros können Counter für Aufgaben manuell zurückgesetzt werden:
\begin{MacroCode}{class}
\DeclareDocumentCommand \rstexe { } {\setcounter{exercisecounter}{0}}
\DeclareDocumentCommand \rstsubexe { } {\setcounter{subexercisecounter}{0}}
\DeclareDocumentCommand \rstmultiexe { } {\setcounter{multiexecounter}{0}}

\end{MacroCode}
\end{macro}
\end{macro}
\end{macro}

