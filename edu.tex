%% edu Documentclass for typesetting educational documents
%%
%% Copyright (C) 2013 by Daniel Wunderlich <code@wu-web.de>
%% 
%% This work may be distributed and/or modified under the
%% conditions of the LaTeX Project Public License, either version 1.3
%% of this license or (at your option) any later version.
%% The latest version of this license is in
%%   http://www.latex-project.org/lppl.txt
%% and version 1.3 or later is part of all distributions of LaTeX
%% version 2005/12/01 or later.
%% 
%% This work has the LPPL maintenance status `maintained'.
%% 
%% The current maintainer of this work is Daniel Wunderlich.
%% 
%% This work consists of the file edu.tex and the derived files:
%%  * edu.cls
%%  * edu-styles-wu.sty
%%  * edu-colors-wu.sty
%%  * edu.pdf
%%  * edu-de.pdf
\documentclass{skdoc}

\usepackage[T1]{fontenc} 
\usepackage[utf8]{inputenc}

\usepackage{amsmath}
\usepackage{amssymb}
\usepackage{array}
\usepackage{booktabs}
\usepackage{cancel}
\usepackage{esvect}
\usepackage{etoolbox}
\usepackage{eurosym}
\usepackage{fancyvrb}
\usepackage{geometry}
\usepackage[notcomma, notperiod, notquote, notquery]{hanging}
%\usepackage{listings}
\usepackage{multicol}
\usepackage{polynom}
\usepackage{tabularx}
\usepackage{tikz}
\usepackage{units}
\usepackage[only, lightning]{stmaryrd}
\usepackage[newcommands]{ragged2e}
\usepackage{hyperref}

\let\SI\relax
\usepackage{siunitx}
\DeclareSIUnit\point{pt}
\usepackage{hologo,booktabs,xcoffins,calc}
\usepackage[style=authoryear]{biblatex}
\usepackage{csquotes}
\usepackage{varioref,cleveref}
%\usepackage{chslacite}

\usetikzlibrary{positioning}

\ExplSyntaxOn
\cs_set_protected_nopar:Npn\ExplHack{
    \char_set_catcode_letter:n{ 58 }
    \char_set_catcode_letter:n{ 95 }
}
\ExplSyntaxOff

% Geometry
\geometry{left=4cm, top=2.5cm, right=2.5cm, bottom=2.5cm, includefoot}

% Fancyvrb
\fvset{frame=single}

% Fontsize
\KOMAoption{fontsize}{10}

% Multilanguage support
\providecommand\locale{de}
\ExplSyntaxOn

\tl_new:N \g_edu_warningname
\tl_gset:Nn \g_edu_warningname {Warning}

\cs_generate_variant:Nn \tl_if_eq:nnT { xnT }
\cs_generate_variant:Nn \tl_if_eq:nnTF { xnTF }
\cs_generate_variant:Nn \tl_if_eq_p:NN { xn }
\cs_generate_variant:Nn \tl_if_eq_p:NN { XN }

\tl_if_eq:xnT {\locale} {de} {
  \usepackage{ngerman}
  \AtBeginDocument{
    \cs_gset:Npn\abstractname{Zusammenfassung}
    \cs_gset:Npn\appendixname{Anhang}
    \cs_gset:Npn\bibname{Bibliographie}
    \cs_gset:Npn\contentsname{Inhalt}
    \cs_gset:Npn\figurename{Abbildung}
    \cs_gset:Npn\glossaryname{Abbildung}
    \cs_gset:Npn\refname{Referenzen}
    \cs_gset:Npn\tablename{Tabelle}
    \tl_gset:Nn \g_edu_warningname {Warnung}
    \tl_gset:Nn \partname {Teil}
  }
}

\cs_new:Npn \__edu_use_language:nn #1 #2 {
  \tl_if_eq:xnT {\locale} {#1} {
	  #2}
}

\DeclareExpandableDocumentCommand \localeDE { +m } {
  \__edu_use_language:nn {de} {#1}
}


\DeclareDocumentCommand \localeEN { +m } {
  \__edu_use_language:nn {en} {#1}
}

\DeclareDocumentCommand \codeDE { O{0} v } {
  \prg_replicate:nn {#1} {\phantom{m}\phantom{m}}
  \__edu_use_language:nn {de} {#2}
}

\newsavebox{\codebox}

\DeclareDocumentEnvironment {codeblockDE} { } {
  \localeDE{%
    \begin{lrbox}{\codebox}
      \begin{minipage}[c]{\linewidth}
      \begin{ttfamily}
  }
}{
  \localeDE{
      \end{ttfamily}
      \end{minipage}
    \end{lrbox}
    \vspace{0.75\baselineskip}
    \par\noindent
    \color{gray}\fbox{\color{black}\usebox{\codebox}}
    \vspace{0.75\baselineskip}
  }
}

\DeclareDocumentCommand \endcode { } {
  \relax
}

\DeclareDocumentCommand \skipcode { u \endcode } {
  \relax
}

\DeclareDocumentCommand \startcode { m } {
  \bool_if:nT {!\str_if_eq_p:nV {#1} {\locale}} {\skipcode}
%  \skipcode
%  \tl_if_eq:xnT {\locale} {#1} {\skipcode}
  
}


% Used Commands:
\DeclareDocumentCommand \fpbox { O{\linewidth} m } {
  \noindent\fbox{\parbox{#1}{#2}}
}

\DeclareDocumentCommand \LongWarning { +m } {
  \vspace{\baselineskip}
  \par\noindent
  \fbox{
    \begin{minipage}[c]{\linewidth}
      \textsf{\textbf{\g_edu_warningname{}:}}~
      #1
    \end{minipage}
  }
  \vspace{\baselineskip}
  \par
}

\ExplSyntaxOff

\DeclareDocumentCommand \WarningStylethemeDE { s } {
  \IfBooleanTF {#1} {
    \LongWarning{Obige Optionen können durch Styletheme überschrieben werden.}
  }{
    \LongWarning{Obige Option kann durch Styletheme überschrieben werden.}
  }
}

\DeclareDocumentCommand \WarningColorthemeDE { s } {
  \IfBooleanTF {#1} {
    \LongWarning{Obige Optionen können durch Colortheme überschrieben werden.}
  }{
    \LongWarning{Obige Option kann durch Colortheme überschrieben werden.}
  }
}

% Hide the implementation
\OnlyDescription

% Bibliography entries
\begin{filecontents}{edu.bib}
    @online{koma,
        author = {Markus Kohm},
        title = {koma-script -- A bundle of versatile classes and packages},
        year = {2012},
        url = {http://www.ctan.org/pkg/koma-script}
    }
    @online{fonts,
        author = {Walter Schmidt},
        title = {Using common PostScript fonts with LaTeX},
        year = {2004},
        url = {http://mirrors.ctan.org/macros/latex/required/psnfss/psnfss2e.pdf}
    }
    @online{fontcatalogue,
        author = {Palle J{\o}rgensen},
        title = {The LaTeX Font Catalogue},
        year = {2012},
        url = {http://www.tug.dk/FontCatalogue}
    }
    @online{tikz,
        author = {Till Tantau},
        title = {pgf -- Create PostScript and PDF graphics in \TeX},
        year = {2008},
        url = {http://www.ctan.org/pkg/pgf}
    }
%        @article{kpfonts,
%        author = {Christophe Caignaert},
%        title = {KP-Fonts 3.31},
%        year = {2010},
%        url = {http://www.tex.ac.uk/tex-archive/fonts/kpfonts/doc/kpfonts.pdf}
%    }
%    @standard{ISO216,
%        title = {ISO 216:2007},
%        subtitle = {Writing paper and certain classes of printed matter -- Trimmed sizes -- A and B series, and indication of machine direction},
%        author = {{International Organization for Standardization, Technical Committee 6}},
%        year = {2007}
%    }
%    @standard{ISO8601,
%        title = {ISO 8601:2004},
%        subtitle = {Data elements and interchange formats -- Information interchange -- Representation of dates and times},
%        author = {{International Organization for Standardization, Technical Committee 154}},
%        year = {2004}
%    }
\end{filecontents}
\addbibresource{edu.bib}

% Declare the target files
\SelfPreambleTo{\mypreamble}
\DeclareFile[key=class,preamble=\mypreamble]{edu.cls}




% This is where the documentation begins

\usepackage[scaled=0.8]{beramono}

\begin{document}
% Change & version info
\version{0.5}
\changes{0.5}{%
  \localeEN{Initial version}%
  \localeDE{Initiale Version}%
}

% Metadata
\package[ctan=edu,vcs=https://github.com/wunderlich]{edu}

\localeEN{\title{The \textbf{\thepackage} document class}}%
\localeDE{\title{Die \textbf{\thepackage} Dokumentenklasse}}%



\author{Daniel Wunderlich}
\email{code@wu-web.de}

% First page
\maketitle
\begin{abstract}
\localeEN{This documentclass provides the easy typesetting of documents for educational use, like worksheets, exams, summaries, lecture notes or scheduling lessons. It loads many useful packages and allows the easy setup of documents (e.\,g. font/-size, margins, header and footer with metadata). It also defines a lot of macros for creating exercises, solutions, questions and for scheduling lessons. At the moment it focuses on mathematics and german schools.}
\localeDE{Diese Dokumentenklasse dient dem Satz von Dokumenten im Kontext der Bildung wie Arbeitsblättern, Klassenarbeiten/Klausuren, Zusammenfassungen, Skripten oder zur Unterrichtsplanung. Sie lädt viele gebräuchliche Packages, erlaubt eine erleichterte Einrichtung der Dokumente (z.\,B. Schriftarten, Seitenränder, Kopf- und Fußzeilen mit Metadaten). Außerdem stellt sie diverse Makros zur Verfügung, u.\,A. zur Erstellung von Aufgaben, Lösungen, Fragen oder der Unterrichtsplanung. Zur Zeit liegt der Schwerpunkt auf Mathematik und deutschen Gegebenheiten.}
\end{abstract}

\tableofcontents

\newpage

\localeDE{\part{Einleitung}}

\localeDE{\section{Über \thepackage}}

\localeDE{
Das komfortable Erstellen von Arbeitsblättern, Klassenarbeiten/Klausuren oder ähnlichen Dokumentenmit \LaTeX\ Standard-Dokumentenklassen erweist sich für Lehrende je nach Anspruch als umständliches Unterfangen. Diese Dokumentenklasse stellt eine Vielzahl von Makros zur Verfügung, die das effiziente Erstellen solcher Dokumente ermöglicht. Sie erlaubt außerdem das Ändern vieler gängiger \LaTeX{}-Optionen, die in diesem Kontext relevant sind.

Hierzu erweitert \thepackage\ die herausragende Dokumentenklasse \pkg{scrartcl} des \emph{KOMA-Scripts} von \textsc{Michael Kohm} \footcite{koma} um die gewünschte Funktionalität und passt entsprechende Parameter (nach Meinung des Autors) sinnvoll an. Außerdem werden in diesem Kontext häufig verwendete Packages geladen und konfiguriert.

% Abschnitt~\ref{verwendung} dieser Dokumentation stellt neben den neuen Befehlen der \textsf{edu}-Klasse auch viele der geladenen und dem Benutzer eventuell unbekannte Packages vor, welche für Lehrende von Interesse sein könnten und erläutert deren grundlegende Verwendung. Die Funktionen dieser Packages werden in den meisten Fällen jedoch nur angeschnitten -- bei Interesse empfiehlt sich ein Blick in die jeweiligen Dokumentationen. Auch grundlegende Aspekte von \LaTeX\ und Typographie im Allgemeinen werden an wenigen Stellen thematisiert. Ein Beispieldokument, welches die Funktionalität der Dokumentenklasse in deutscher Sprache demonstriert, stellt |edu-demo-de.sty| mit dem zugehörigen PDF |edu-demo-de.pdf| dar.
%
% Die Implementierung des Packages in Abschnitt~\ref{implementierung} enthällt den Code der Klasse und ist im Allgemeinen nur für Autoren von Klassen oder Packages interessant. Benutzer können diesen Abschnitt im Normalfall vernachlässigen.
}

\localeDE{
\section{Über diese Dokumentation}

Diese Dokumentation verwendet verschiedene Schriftarten und -stile zur Auszeichnung unterschiedlicher Komponenten. Tabelle~\ref{tab:components} zeigt diese Arten der Auszeichnungen.

\begin{table}[h!tp]
  \centering
  \caption{Auszeichnung durch Schriftarten und -stile dieser Dokumentation.}
  \label{tab:components}
  \medskip
  \begin{tabular}{ll} \toprule
    \textsf{Schrift} & \textsf{Beschreibung} \\ \midrule
    \pkg*{package} & Package \\
    \opt*{option} & Option \\
    \cs*{macro} & Makro\footnotemark \\
    \env*{umgebung} & Umgebung \\
    \meta{argument} & Argument (allgemein) \\
    \marg{argument} & Notwendiges Argument \\
    \oarg{argument} & Optionales Argument \\
    \bottomrule
  \end{tabular}
\end{table}

\footnotetext{{Es wird versucht, in dieser Dokumentation ausschließlich den Begriff \emph{Makro} zu verwenden. Die Begriffe \emph{Befehl}, \emph{Funktion} und \emph{Kommando} sind -- sollten sie wider Erwarten doch verwendet werden -- als Synonyme zu verstehen.}}

An einem Beispiel sei an dieser Stelle kurz der Unterschied zwischen notwendigen und optionalen Argumenten erläutert: Wir betrachten das Makro \Macro\fpbox[<Breite>=\cs*{linewidth}]{<Inhalt>} erzeugt einen umrahmten Absatz. Das Argument \meta{Inhalt} ist \emph{notwendig} und \emph{muss} angegeben werden. (Was wäre ein Absatz ohne Inhalt?). Der Aufruf \Macro\fpbox{Dies ist ein umrahmter Absatz.} erzeugt einen umrahmten Absatz, der sich über die gesamte Breite der aktuellen Zeile erstreckt:


\medskip
\fpbox{Dies ist ein umrahmter Absatz.}

\noindent Das Argument \meta{Breite} ist \emph{optional}, es \emph{kann} angegeben werden. Im obigen Beispiel war dies nicht der Fall. Das Makro greift dann ggf. auf einen Standardwert, in diesem Fall \cs*{linewidth} (Breite der aktuellen Zeile), zurück. Der Aufruf \Macro\fpbox[3cm]{Dies ist ein umrahmter Absatz.} hingegen erzeugt einen umrahmten Absatz der Breite 5\,cm:

\medskip
\fpbox[3cm]{Dies ist ein umrahmter Absatz.}

}

\newpage
\localeDE{\part{Installation}}

\localeDE{\section{Voraussetzungen}}

\localeDE{
Die \thepackage-Klasse benötigt neben Standard-\LaTeX-Packages die Dokumentenklasse \pkg{scrartcl} des \emph{KOMA-Scripts} \footcite{koma} und die folgenden Packages:


\begin{multicols}{5}
{\sffamily 
\noindent amsmath\\
amssymb\\
amsthm\\
beramono*\\
bibgerm\\
booktabs\\
calc\\
cancel\\
ccicons\\
enumitem\\
esvect\\
etoolbox\\
eurosym\\
expdlist\\
expl3\\
gauss\\
geometry\\
graphicx\\
hanging\\
hyperref\\
icomma\\
l3keys2e\\
lastpage\\
lato*\\
listings\\
mathpazo*\\
menukeys\\
multicol\\
multirow\\
pdflscape\\
pifont\\
polynom\\
roboto*\\
rotating\\
ragged2e\\
scrpage2\\
setspace\\
siunitx\\
sourcecodepro*\\
sourcesanspro*\\
struktex\\
subfig\\
tabularx\\
thmtools\\
tikz\\
tikzsymbols\\
tikz-qtree\\
titlesec\\
stmaryrd\\
ulem\\
xcolor\\
xlop\\
xparse\\
xspace}
\end{multicols}
\noindent\textsf{* optional}
 
\bigskip
Alle Packages sind über \emph{CTAN} erhältlich -- sie können unter Linux über \emph{TeX Live}, unter Windows über \emph{MiKTeX} und unter MacOSX über \emph{MacTeX} bezogen werden.
}


%\section{Installation}
% 
%Die manuelle Installation von Packages bzw. Dokumentenklassen wird an vielen Stellen im Internet erläutert. Deshalb wird sie hier nur sehr kompakt beschrieben. Bei Problemen bieten diverse Websiten Hilfestellung.
% 
%\subsection{Linux (Ubuntu 12.04/Linux Mint 13)}
%
%\begin{enumerate}
%  \item Per Kommandozeile in den Ordner navigieren, indem sich die heruntergeladene Datei \verb+edu.cls+ befindet.
%  \item Einen Ordner für die Dokumentenklasse im \TeX-Verzeichnisbaum erstellen:
%  \begin{verbatim}
%sudo mkdir /usr/share/texmf/tex/latex/edu
%  \end{verbatim}\vspace{-\baselineskip}
%  \item Nun wird die Datei \verb+edu.cls+ in den neuen Ordner kopiert:
%  \begin{verbatim}
%sudo cp edu.cls /usr/share/texmf/tex/latex/edu/
%  \end{verbatim}\vspace{-\baselineskip}
%    \item Abschließend muss der \TeX-Verzeichnisbaum neu aufgebaut werden:
%  \begin{verbatim}
%sudo mktexlsr
%  \end{verbatim}\vspace{-\baselineskip}
%\end{enumerate}
%
% 
%\subsection{Windows\,7}
%
%\begin{enumerate}
%  \item Bei einer Standardinstalltion von MiKTeX~2.9 unter Windows\,7 zuerst den Ordner
%  \begin{verbatim}
%C:\Program Files (x86)\MiKTeX 2.9\tex\latex\edu
%  \end{verbatim}
%  \vspace{-\baselineskip}
%  erstellen.
%  \item Dann die Datei \verb+edu.cls+ in diesen Ordner verschieben.
%  \item Das Programm \verb+Settings+ von MiKTeX öffnen:
%  \begin{center}
%    \itshape Startmenü $\rightarrow$ Alle Programme $\rightarrow$ MiKTeX 2.9 $\rightarrow$ Maintenance $\rightarrow$ Settings
%  \end{center}
%  \item Über die Schaltfläche \emph{Refresh~FNDB} wird die neue Datei eingelesen.
% \end{enumerate}

\newpage
\localeDE{\part{Allgemeines zum Textsatz in der Schule}}

\localeDE{\section{Kopf- und Fußzeilen}\label{sec:kopfzeile}}

\localeDE{\section{Absatzauszeichnung}}

\localeDE{
 Es gibt zwei gängige Möglichkeiten, Absätze auszuzeichnen (d.\,h. kenntlich zu machen). Zum einen kann vor einem Absatz ein Abstand eingefügt werden. Diese Methode hat jedoch den gravierenden Nachteil, dass Absatzumbrüche an Seitenenden, Gleitobjekten (z.\,B. durch \env*{figure}- oder \env*{table}- Umgebungen) und abgesetzten Formeln nicht zu erkennen sind. Auch die Tatsache, dass in der letzten Zeile eines Absatzes kein Blocksatz erzwungen wird und sie somit "`linksbündig"' erscheint, beseitigt diesen Nachteil nicht zufriedenstellend: Da je nach Beschaffung eines Absatzes auch die letzte Zeile als Blocksatz erkannt werden kann (wenn sie "`komplett gefüllt"' ist), genügt dieses Kriterium im Allgemeinen nicht. Außerdem ist zu Beginn einer neuen Seite die letzte Zeile des letzten Absatzes nicht zu erkennen und erfordert ggf. ein Umblättern.

Deshalb ist die zweite Methode \emph{grundsätzlich} vorzuziehen: In jedem Absatz wird die erste Zeile eingerückt. Hierdurch werden alle Nachteile beseitigt und man erkennt beim Lesen unmittelbar den Beginn eines neuen Absatzes. In vielen gedruckten Büchern wird diese Art der Absatzauszeichnung deshalb verwendet.

\emph{Im Kontext der Schule}, insbesondere bei Dokumenten wie Arbeitsblättern und Klausuren im naturwissenschaftlichen Bereich, werden jedoch überwiegend sehr kurze "`Absätze"' verwendet. Häufige Absatzeinrückungen sind die Folge. Diese stören das Gesamtbild des Dokuments in diesem Fall jedoch stark. Aus diesem Grund wird in \thepackage\ standardmäßig Absatzabstand zur Auszeichnung von Absätzen verwendet. Es sei jedoch noch einmal darauf hingewiesen, dass dies eine Ausnahme darstellt, welche der Beschaffenheit von Dokumenten in der Schule geschuldet ist.
}

\newpage
\localeDE{\part{Tutorials: Der schnelle Einstieg}}

\localeDE{\section{Horst erstellt ein Arbeitsblatt}}

\localeDE{
In diesem Teil begleiten wir einen fiktiven Lehrenden -- namentlich Horst -- dabei, ein Arbeitsblatt mit \thepackage zu erstellen. Horst hat mit \LaTeX\ bereits einige Erfahrungen gesammelt und ist in der Lage, einfache Dokumente zu erstellen.
}

\localeDE{\subsection{Laden der Dokumentenklasse}}

\localeDE{
Der erste Schritt besteht im Laden der Dokumentenklasse. Dies geschieht im einfachsten Fall durch die Zeile
}

\startcode{de}
\begin{Verbatim}
\documentclass{edu}
\end{Verbatim}
\endcode
%
\localeDE{
Viele Einstellungen von \thepackage\ lassen sich direkt beim Laden der Dokumentenklasse konfigurieren. Dies geschieht über sogenannte \emph{Optionen}. Horst hat z.\,B. nicht die besten Augen. Deshalb entscheidet er sich, die Standardschriftgröße etwas zu erhöhen. Dies geschieht über die Option \opt*{fontsize}. Diese wird beim Laden der Dokumentenklasse angegeben:
}

\startcode{de}
\begin{Verbatim}
\documentclass[fontsize=12pt]{edu}
\end{Verbatim}
\endcode
%
Möchte man mehrere Optionen laden, sollte man sie -- zugunsten der Lesbarkeit -- in einzelne Zeilen schreiben:

\startcode{de}
\begin{Verbatim}
\documentclass[
  fontsize=14pt,
  footer=false,
]{edu}
\end{Verbatim}
\endcode

\localeDE{\subsection{Metadaten eingeben}}

\localeDE{
Horst kennt seine Schüler -- die meisten von ihnen sind unordentlich. Um sie zu unterstützten möchte er die wichtigsten Metadaten (d.\,h. in diesem Fall Daten über das Arbeitsblatt, nicht der Inhalt des Blattes selbst) auf dem Blatt platzieren. Viele der hierfür verwendeten Makros kennt Horst bereits aus den Standard-Dokumentenklassen. Doch ein Blick auf Tabelle~\ref{tab:metadaten} beinhaltet auch Makros, die für ihn neu sind.
}

\localeDE{
\begin{table}[h!tb]
  \centering
  \caption{Metadaten und die zugehörigen Makros (in alphabetischer Reihenfolge).}
  \label{tab:metadaten}
  \medskip
   \begin{tabular}{lll}\toprule
     \sffamily Datum & \sffamily Makro  & \sffamily Beispiel \\\midrule
     Author & \Macro\author & \Macro\author{Horst} \\ 
     Klasse & \Macro\class & \Macro\class{Klasse 10} \\ 
     Datum & \Macro\date & \Macro\date{6.3.2013} \\ 
     E-Mail & \Macro\email & \Macro\email{horst@intern.et} \\ 
     Gebiet & \Macro\field & \Macro\class{Programmierung} \\ 
     Gruppe & \Macro\group & \Macro\group{A} \\ 
     Lizenz & \Macro\license & \Macro\license{Public Domain} \\ 
     Fach & \Macro\subject & \Macro\subject{Informatik} \\ 
     Untertitel & \Macro\subtitle & \Macro\subtitle{Mach es mehrmals!} \\ 
     Titel & \Macro\title & \Macro\title{Die While-Schleife} \\ 
     Version & \Macro\version & \Macro\version{1.0} \\ \bottomrule
   \end{tabular}
 \end{table}
}

\localeDE{
Nach etwas Bedenkzeit (s.\,Abschnitt~\ref{sec:kopfzeile}) hat sich Horst für die wichtigsten Metadaten entschieden und ergänzt diese in seiner Quelldatei in der Präambel:
}

\startcode{de}
\begin{Verbatim}
\author{Horst}
\class{Klasse 10}
\date{6.3.2013}
\field{Zuordnungen}
\subject{Informatik}
\title{Die While-Schleife}
\end{Verbatim}
\endcode
%
\localeDE{
Diese Auswahl an Metadaten sollte in jedem Dokument angegeben werden. Weitere Metadaten können wahlweise verwendet werden. Die Makros bewirken, dass die angegebenen Daten sinnvoll im Titel und in Kopf- und Fußzeile des Dokuments gesetzt werden.

Horst möchte seiner Kollegin Ingeborg Hubmüller-Weidenfels die Vorteile von \thepackage\ näherbringen. Diese stellt jedoch fest, dass ihr Name zu einem Problem führt: Wegen seiner Länge passt dieser -- zusammen mit weiteren Metadaten -- in die Fußzeile\footnote{Wir werden uns erst später mit Fußzeilen beschäftigen.}. Aus diesem Grund verfügen die Makros zur Angabe der Metadaten zusätzlich über ein optionales Argument. Dieses kann verwendet werden, um eine kürzere Variante des eigentlichen Datums zu wählen. Diese kurze Version wird in Kopf- bzw. Fußzeile verwendet. Ingeborg kürzt deshalb ihren Vornamen für die Fußzeile ab:
}

\startcode{de}
\begin{Verbatim}
\author[I. Hubmüller-Weidenfels]{Ingeborg Hubmüller-Weidenfels}
\end{Verbatim}
\endcode
%

\localeDE{\subsection{Kopfzeile erzeugen}}

\localeDE{
Nachdem Horst in seinem Arbeitsblatt alle wichtigen Metadaten angegeben hat, möchte er es nun endlich mit "`richtigem"' Inhalt füllen. Er ergänzt die \env*{document}-Umgebung, welche den Dokumentenkörper beinhaltet:
}

\startcode{de}
\begin{Verbatim}
\begin{document}
  % Hier wird der Inhalt des Dokuments ergänzt.
\end{document}
\end{Verbatim}
\endcode

\localeDE{
Zu Beginn der ersten Seite möchte Horst die Kopfzeile\footnote{In Abschnitt~\ref{sec:kopfzeile} wird erläutert, weshalb standardmäßig lediglich die erste Seite mit einer Kopfzeile versehen wird.} erstellen. Dies geschieht  sehr leicht gleich zu Beginn des Dokumentenkörpers durch das Makro \Macro\makeheader:
}

\startcode{de}
\begin{Verbatim}
\makeheader
\end{Verbatim}
\endcode
%
\localeDE{
Horst kompiliert seinen Quelltext und freut sich über das Ergebnis: Die Kopfzeile wurde erstellt und beinhaltet die angegebenen Metadaten! Und auch die Fußzeile\footnote{Durch die Option \opt{footer} kann die Fußzeile deaktiviert werden.} wurde automatisch durch \thepackage\ erstellt.
}

\localeDE{\subsection{Titel erzeugen}}

\localeDE{
Als nächstes möchte sich Horst den Titel seines Arbeitsblattes einfügen. Nachdem er den Titel bereits mithilfe des Makros \Macro\title eingegeben hat, muss dieser noch erzeugt werden. Dies geschieht durch das Makro \Macro\maketitle.

Die normale Variante \Macro\maketitle erzeugt einen "`ausführlichen"' Titel, der neben Titel und Untertitel noch fast alle weiteren Metadaten des Dokuments beinhaltet. Sie eignet sich zur Verwendung in Dokumenten größeren Umfangs (Zusammenfassungen u.\,Ä.).

Aus Platzgründen empfiehlt sich bei Arbeitsblättern die Verwendung der Sternvariante \Macro\maketitle*. Diese erzeugt lediglich den Titel. Falls angegeben, werden außerdem Untertitel gesetzt.
}

\startcode{de}
\begin{Verbatim}
\maketitle*
\end{Verbatim}
\endcode

\localeDE{\subsection{Aufgaben erstellen}}

\localeDE{
"`Genug des Vorgeplänkels!"', denkt sich Horst. Nun ist es Zeit, Aufgaben zu erstellen. Um Überschriften für Aufgaben zu erstellen (1.~Aufgbabe, 2.~Augabe, \dots) verwendet man das Makro \Macro\exe\ (engl. \emph{exercise} -- \emph{Aufgabe}). Hierbei werden Aufgaben automatisch arabisch nummeriert. Anschließend verfasst man wie gewünscht den Inhalt der Aufgabe:
}

\startcode{de}
\begin{Verbatim}
\exe{}

Lorem ipsum dolor sit amet ...
\end{Verbatim}
\endcode
%
\localeDE{
Horst fände es praktisch, wenn er den Aufgaben einen Titel geben könnte. Gleichzeitig wundert er sich über das geschwungene Klammerpaar nach \Macro\exe. Zu Recht! Das Makro verfügt über ein notwendiges Argument, den Titel der Aufgabe. Gibt man keinen Titel an (wie es im obigen Beispiel der Fall ist), erscheint lediglich \emph{1.~Aufgabe}. Ansonsten wird der Titel direkt hinter dieser Beschriftung gesetzt. Horst möchte diese Funktion nutzen und ergänzt:
}

\startcode{de}
\begin{Verbatim}
\exe{Mehrfache Ausgabe}
Lorem ipsum dolor sit amet ...
\end{Verbatim}
\endcode
%
\localeDE{
Es besteht die Möglichkeit, als zusätzliche Gliederungsebene "`Unteraufgaben"' zu erstellen. Wie bei \cs*{section} und \cs*{subsection} bezeichnet man das entsprechende Pendant zu \Macro\exe\ mit \Macro\subexe. Auch die Verwendung geschieht analog. Die Nummerierung hingegen geschieht durch eine zweite Ebene (1.1, 1.2, 1.3, \dots). Auch \Macro\subexe verfügt über ein notwendiges Argument zur Angabe eines Titels für die Aufgabe. Soll kein Titel verwendet werden, die Klammern dennoch angegeben werden. Horst gefällt die Idee der Unteraufgaben und er möchte sie gleich verwenden:
}

\startcode{de}
\begin{Verbatim}
\exe{Mehrfache Ausgabe}

\subsexe{Text}
Lorem ipsum dolor sit amet ...

\subsexe{Berechnungen}
Lorem ipsum dolor sit amet ...
\end{Verbatim}
\endcode
%
\localeDE{
Mit jeder Erstellung einer neuen Aufgabe durch \Macro\exe wird die Nummerierung von \Macro\subexe vorn vorne begonnen.

Unteraufgaben eignen sich hauptsächlich bei sehr Umfangreichen Aufgaben.  Horst fragt sich, ob es auch die Möglichkeit gibt, mehrere kurze Aufgaben in einer alphabetisch nummerierten Liste zu setzen, wie man es z.\,B. aus Mathematik-Büchern kennt. Die gibt es tatsächlich: "`Teilaufgaben"' können zum einen als Liste gesetzt werden. Hierzu dient die Umgebung \env{multiexelist}. Sie verhält sich wie die \env*{itemize} Umgebung, nummeriert die Einträge jedoch alphabetisch.
}

\startcode{de}
\begin{Verbatim}
\begin{multiexelist}
  \item Erste Teilaufgabe
  \item Zweite Teilaufgabe
  \item Dritte Teilaufgabe
\end{multiexelist}
\end{Verbatim}
\endcode
%
\localeDE{
Folgt eine Teilaufgabenliste direkt auf eine (Unter-)Aufgabenüberschrift (d.\,h. auf \Macro\exe oder \Macro\subexe), sollte die Sternvariante \env{multiexelist} verwendet werden, da sie einen zusätzlichen Abstand über der Liste einfügt.

Falls gewünscht, kann über ein optionales Argument ein zusätzlicher Abstand zwischen den Einträgen eingefügt werden:
}

\startcode{de}
\begin{Verbatim}
\begin{multiexelist}[2ex]
  \item Erste Teilaufgabe
  \item Zweite Teilaufgabe
  \item Dritte Teilaufgabe
\end{multiexelist}
\end{Verbatim}
\endcode
%
\localeDE{
Möchte man mehrere kurze Teilaufgaben verfassen -- was häufig bei Mathematikaufgaben der Fall ist -- kann die Umgebung \env{multiexearray} verwenden. Sie ordnet die Teilaufgaben wie in einer Tabelle an. Als notwendiges Argument benötigt sie deshalb die Anzahl der Spalten und wird ansonsten wie eine Tabelle verwendet. Hier ein Beispiel für den dreispaltigen Satz von Teilaufgaben:
}

\startcode{de}
\begin{Verbatim}
\begin{multiexearray}{3}
	Erste Teilaufgabe & Zweite Teilaufgabe & Dritte Teilaufgabe \\
	Vierte Teilaufgabe & 	Fünfte Teilaufgabe
\end{multiexearray}
\end{Verbatim}
\endcode
%
\localeDE{
Es wir deutlich, dass die unterste Zeile nicht komplett "`befüllt"' werden muss. Es existiert zusätzlich eine Sternvariante \env{multiexearray*}. Diese versetzt die Zellen der Tabelle automatisch in den Mathematikmodus:
}

\startcode{de}
\begin{Verbatim}
\begin{multiexearray*}{4}
  2x = 3,5 & 3x = 4 & 4x = 5 & 5x = 6 \\
  6x = 7 & 7x = 8
\end{multiexearray*}
\end{Verbatim}
\endcode
%
\localeDE{
Insbesondere beim Satz von Mathematikaufgaben kann es vorkommen, dass sich einzelne Zeilen berühren -- vor allem bei "`höheren"' Elementen wie Brüchen oder Integralen. Deshalb kann man auch bei \env{multiexearray} und \env{multiexearray*} als optionales Argument einen zusätzlichen Zeilenabstand angeben:
}

\startcode{de}
\begin{Verbatim}
\begin{multiexearray*}[2ex]{3}
  \frac{1}{2} x = 3 & \frac{2}{3} x = 4 & \frac{3}{4} x = 5 \\ 
  \frac{4}{5} x = 6 & \frac{5}{6} x = 7 & \frac{6}{7} x = 8
\end{multiexearray*}
\end{Verbatim}
\endcode
%
\localeDE{
Innerhalb einer (Unter-)Aufgabe können mehrere Umgebungen mit Teilaufgaben verwendet werden. Die Nummerierung beginnt von vorne, sobald eine (Unter-)Aufgabe beginnt. Eine Limitierung ist jedoch zu berücksichtigen: Es können pro (Unter-)Aufgabe maximal 26 Teilaufgaben erstellt werden -- mehr Kleinbuchstaben bietet das deutsche Alphabet nicht (abgesehen von Umlauten und scharfem S).

Zähler können manuell zurückgesetzt werden. Nach dem Aufruf den Makros \Macro\rstexe, \Macro\rstsubexe oder \Macro\rstmultiexe\ (engl. \emph{reset} -- \emph{zurücksetzen}) beginnt die Nummerierung von \Macro\exe, \Macro\subexe bzw. \env{multiexelist}/\env{multiexearray} bei der nächsten Verwendung von vorne.
}

\localeDE{\subsection{Spalten verwenden}}

\localeDE{
Zwar unterrichtet Horst keine Sprachen im üblichen Sinne, dennoch sei an dieser Stelle erwähnt, dass \thepackage\ eine Vielzahl von Möglichkeiten bietet, Text in Spalten zu setzen. Den Möglichkeiten ist gemeinsam, dass der Fokus auf manuellem Spaltenumbruch basiert. Dies ist insbesondere bei Arbeitsblättern sinnvoll, da hier platzsparend gearbeitet werden soll.

Die grundlegende Umgebung zur Erstellung von Spalten ist \env{cols}. Standardmäßig erzeugt \env{cols} zwei Spalten. Der Spaltenumbruch geschieht durch das Makro \Macro\colbreak:
}

\startcode{de}
\begin{Verbatim}
\begin{cols}
  Erste Spalte ...
  \colbreak
  Zweite Spalte ...
\end{cols}
\end{Verbatim}
\endcode
%
\localeDE{
Durch ein optionales Argument kann \env{cols} eine Spaltenanzahl übergeben werden:
}


\startcode{de}
\begin{Verbatim}
\begin{cols}[4]
  Erste Spalte ...
  \colbreak
  Zweite Spalte ...
  \colbreak
  Dritte Spalte ...
  \colbreak
  Vierte Spalte ...
\end{cols}
\end{Verbatim}
\endcode
%
\localeDE{
Wie bei Text meistens gewünscht, werden die beiden Spalten oben aneinander ausgerichtet. Die Sternvariante \env{cols*} (ebenfalls mit optionaler Spaltenanzahl) zentriert den Inhalt der beiden Spalten vertikal.
}

\localeDE{
Manchmal ist es -- insbesondere bei zwei Spalten -- hilfreich, die Breite einer Spalte manuell angeben zu können. Hierzu kann die Umgebung \env{cols2} verwendet werden. Ohne optionales Argument verhält sie sich wie \env{cols} und erzeugt zwei gleich breite Spalten. Durch das optionale Argument kann jedoch die Breite der linken Spalte angegeben werde. Die Breite der rechten Spalte erstreckt sich über den restlichen Platz der Zeilenbreite.

Hier zwei Beispiele -- eines mit einer \SI{5}{\cm} breiten Spalte und eines mit einer linken Spalte, deren Breite \SI{35}{\%} der Zeilenbreite entspricht:
}


\startcode{de}
\begin{Verbatim}
\begin{cols2}[5cm]
  Erste Spalte ...
  \colbreak
  Zweite Spalte ...
\end{cols2}

\begin{cols2}[0.35\linewidth]
  Erste Spalte ...
  \colbreak
  Zweite Spalte ...
\end{cols2}
\end{Verbatim}
\endcode
%
\localeDE{
Auch \env{cols2} verfügt über eine Sternvariante, bei der die Spalten vertikal zentriert werden.
}

\localeDE{
\emph{Bemerkung:} Es handelt sich bei der Angabe der linken Spaltenbreite nicht um die exakte Breite der Spalte. Genau genommen handelt es sich um den Abstand zwischen dem Beginn der Zeile und der Mitte des Abstandes zwischen den Spalten. Mathematisch formuliert:
\[
	\text{Tatsächliche linke Spaltenbreite} = \text{Angegebene linke Spaltenbreite} - 0.5 \cdot \text{Spaltenabstand}
\]
}

\localeDE{\subsection{Rastergrafiken (Bilder) einfügen}}

\localeDE{
"`A propros Spalten!"', schaltet sich Horst wieder ein. "`Ich würde gerne neben einer Aufgabe eine Abbildung einfügen. Geht das auch durch Spalten?"' Eine berechtigte Frage! Bevor wir uns Horsts Problem widmen, einige grundlegende Aspekte zu Rastergrafiken in \thepackage:

Rastergrafiken werden von \thepackage\ im Unterverzeichnis\footnote{Der Begriff \emph{Verzeichnis} wird in dieser Dokumentation synonym zu \emph{Ordner} verwendet.} \texttt{img} (engl. \emph{image} -- \emph{Bild}, \emph{Abbildung}) gesucht. Diese Maßnahme dient dazu, Ordnung im eigentlichen "`Arbeitsverzeichnis"' zu wahren. Das Einbinden von Grafiken geschieht grundsätzlich wie in \LaTeX\ üblich durch \Macro\includegraphics.

Zurück zu Horsts Anliegen. Die Umgebung \env{graphicscol} dient genau dem von ihm angesprochenen Zweck: Er erzeugt zwei Spalten, wobei in der linken Spalte der Text und in der rechten Spalte die Grafik eingefügt wird. Die Verwendung von \env{graphicscol} soll anhand eines Beispiels erläutert werden. Als erstes Argument kann optional die Breite der linken Spalte angegeben werden (z.\,B. \SI{70}{\%} der Zeilenbreite, \texttt{0.7}\cs*{linewidth}). Es folgt die notwendige Angabe des Dateinamens (z.\,B. \texttt{image-example.jpg}). Abschließend können optional Optionen des Makros \cs*{includegraphics} angegeben werden, die auf die Grafik angewendet werden (z.\,B. \texttt{width=4cm}). Der Inhalt der Umgebung entspricht dem Text, der neben der Grafik angezeigt wird -- z.\,B. eine Aufgabe:
}

\startcode{de}
\begin{Verbatim}
\begin{graphicscol}[0.7\linewidth]{image-example.jpg}[width=4cm]
  Lorem ipsum dolor sit amet ...
\end{graphicscol}
\end{Verbatim}
\endcode
%
\localeDE{
Um die Grafik links des Textes zu setzen, kann die Sternvariante \env{graphicscol*} genutzt werden, die ansonsten analog zu \env{graphicscol} verwendet wird.

Zusätzlich kann durch ein weiteres, letztes optionales Argument bestimmt werden, wie die Grafik innerhalb ihrer Spalte horizontal ausgerichtet werden soll. Mögliche Angaben sind \texttt{l} für linksbündig, \texttt{r} für rechtsbündig und \texttt{c} für zentriert (engl. \emph{centered} -- \emph{zentriert}). Standardmäßig werden Grafiken in \env{graphicscol} linksbündig und in \env{graphicscol*} rechtsbündig ausgerichtet.
}


\localeDE{\subsection{TikZ-Grafiken einfügen}}

\localeDE{
Horst erstellt häufig Grafiken mit Makropackage Ti\textit{k}z \cite{tikz}. Den Quellcode der Grafiken speichert er in PGF-Dateien ab. Durch das Makro \cs{input} bindet er diese Dateien in sein Dokument ein. \thepackage\ erweitert das Einbinden von Ti\textit{k}z-Grafiken um einige Funktionen.

Wie bei Rastergrafiken müssen die Grafiken in einem Unterverzeichnis des aktuellen Dokuments -- in diesem Fall mit dem Namen \texttt{tikz} -- vorliegen. An der jeweiligen Stelle des Dokument können die Grafiken durch \Macro\tikzinput eingebunden werden, z.\,B.
}

\startcode{de}
\begin{Verbatim}
\tikzinput{tikz-example.pgf}
\end{Verbatim}
\endcode
%
\localeDE{
"`Schön und gut -- aber ich hätte meine Grafik gerne zentriert. Muss ich jetzt jedes Mal eine \env{center}-Umgebung um \Macro\tikzinput packen?"', erwidert Horst. Nein, das muss Horst nicht. Exakt zu diesem Zweck gibt es die Sternvariante \Macro\tikzinput*. Sie zentriert die Grafik durch eine \env{center}-Umgebung:
}

\startcode{de}
\begin{Verbatim}
\tikzinput*{tikz-example.pgf}
\end{Verbatim}
\endcode
%
\localeDE{
Horst ist fast zufrieden. Text neben einer Ti\textit{k}z-Grafik, das würde ihn glücklich stimmen. Er erinnert sich an \env{graphicscol} aus dem vorherigen Abschnitt und mutmaßt, dass es eine analoge Umgebung \env{tikzcol} gibt:
}

\startcode{de}
\begin{Verbatim}
\begin{tikzcol}[0.45\linewidth]{tikz-example.pgf}
  Lorem ipsum dolor sit amet ...
\end{tikzcol}
\end{Verbatim}
\endcode
%
\localeDE{
Und er liegt richtig! Durch das erste, optionale Argument kann die Breite der linken Spalte angegeben werden, die den Text beinhaltet. Es folgt der notwendig anzugebende Dateiname \texttt{tikz-example.pgf}. Bei der Sternvariante \env{tikzcol*} wird der Text rechts der Grafik gesetzt.

Durch ein zusätzliches (hier nicht verwendetes) drittes und optionales Argument kann die horizontale Ausrichtung der Grafik angegeben werden: Mögliche Angaben sind auch hier \texttt{l} für linksbündig, \texttt{r} für rechtsbündig und \texttt{c} für zentriert. Standardmäßig werden Ti\textit{k}z-Grafiken in \env{tikzcol} linksbündig und in \env{tikzcol*} rechtsbündig ausgerichtet.
}


\localeDE{\subsection{Lösungen erstellen}}

\localeDE{\subsection{DIN-A4-Druckvorlage und Folien}}



\localeDE{\section{Horst erstellt eine Klausur}}



\localeDE{\section{Horst bereitet Mathematikunterricht vor}}



\newpage
\localeDE{\part{Dokumentation}}

\localeDE{\subsection{Allgemeines}}

\localeDE{\subsubsection{Farben}}

\localeDE{
\subsection{Optionen}

Die Optionen von \thepackage\ werden thematisch gruppiert beschrieben. Der Standardwert ist rechtsbündig in Klammern angegeben.
}

\localeDE{\subsubsection{Medienarten}\label{sec:medienarten}}

\localeDE{
\thepackage\ verfügt über Modi für spezielle Medienarten. Diese Modi führen zu einer Anpassung insbesondere der Schriftgröße und der Seitenränder. In manchen fällen kann außerdem die Schriftfamilie geändert werden.

\LongWarning{Diese Optionen werden mit hoher Priorität verarbeitet. D.\,h. andere Optionen (z.\,B. \opt{fontsize}) werden gegebenenfalls überschrieben und habe keine Wirkung mehr.}
}

\localeDE{
\Option{twoup}\WithValues{true, false}\AndDefault{false}
Diese Option dient zum Ausdruck einer DIN-A4-Seite, die nach oder vor dem Druck auf DIN-A5 herunter skaliert wird. Dies dient insbesondere dem Druck von zwei DIN-A5-Seiten auf ein DIN-A4-Blatt. Hierzu wird die Schrift vergrößert und die Seitenränder angepasst.

Im Folgenden wird dieser Zustand als \opt*{twoup}-Modus bezeichnet.
}

\localeDE{
\Option{transparency}\WithValues{true, false}\AndDefault{false}
Möchte man Folien für Overhead-Projektoren erstellen, sollten einige Eigenschaften des Dokuments angepasst werden: Die Schrift muss (drastisch) vergrößert werden, Seitenränder können hingegen minimiert werden. Außerdem sollte serifenlose Schrift als Standardschrift verwendet werden, da sie die Lesbarkeit auf größerer Entfernung verbessert. Diese Anpassungen werden durch die Option \opt*{transparency} vorgenommen.

Im Folgenden wird dieser Zustand als \opt*{transparency}-Modus bezeichnet.
}

\localeDE{
\Option{glue}\WithValues{true, false}\AndDefault{false}
In niedrigeren Klassenstufen werden Ausdrucke häufig in ein Heft geklebt. Um dies zu vereinfachen, kann man das Blatt rechts und unten etwas beschneiden. Die Option \opt*{glue} passt die Seitenränder so an, dass der untere und rechte Seitenrand 5\,mm größer als der obere und linke Seitenrand sind. Nach dem Beschneiden besitzt der Inhalt ungefähr den gleichen Abstand zu allen Rändern des Blattes.

Im Folgenden wird dieser Zustand als \opt*{glue}-Modus bezeichnet.
}


\localeDE{\subsubsection{Schriftarten}}

\localeDE{
\Option{sfdefault}\WithValues{true, false}\AndDefault{false}
Mit dieser Option kann als Standardschriftart serifenlose Schrift gewählt werden.
}

%\medskip
%
%\localeDE{
%\Option{beramono}\WithValues{true, false}\AndDefault{true}
%Mit dieser Option wird \emph{Bera Mono} mithilfe des Packages \pkg*{bera} als Standardschriftart für nichtproportionale Schrift gewählt.
%}
%
%\medskip
%
%\localeDE{
%\Option{lato}\WithValues{true, false}\AndDefault{false}
%Mit dieser Option wird \emph{Lato} mithilfe des Packages \pkg*{lato} als Standardschriftart für serifenlose Schrift gewählt.
%}
%
%\medskip
%
%\localeDE{
%\Option{palatino}\WithValues{true, false}\AndDefault{true}
%Mit dieser Option wird \emph{Palatino} mithilfe des Packages \pkg*{mathpazo} als Standardschriftart für serifenlose Schrift gewählt. Dies umfasst auch die Mathemik-Schriftarten.
%}
%
%\medskip
%
%\localeDE{
%\Option{sourcecodepro}\WithValues{true, false}\AndDefault{false}
%Mit dieser Option wird \emph{Source Code Pro} mithilfe des Packages \pkg*{sourcecodepro} als Standardschriftart für serifenlose Schrift gewählt.
%}
%
%\medskip
%
%\localeDE{
%\Option{sourcesanspro}\WithValues{true, false}\AndDefault{true}
%Mit dieser Option wird \emph{Source Sans Pro} mithilfe des Packages \pkg*{sourcesanspro} als Standardschriftart für serifenlose Schrift gewählt.
%}


\localeDE{\subsubsection{Schriftgrößen}}

\localeDE{
\Option{fontsize}\WithValues{\meta{Schriftgröße}}\AndDefault{11pt}
Gibt die Schriftgröße des Fließtextes im Dokument an.
}

\medskip

\localeDE{
\Option{twoupfontsize}\WithValues{\meta{Schriftgröße}}\AndDefault{14}
Gibt die Schriftgröße des Fließtextes im Dokument an, falls der \opt{twoup}-Modus gewählt wurde.
}

\medskip

\localeDE{
\Option{transparencyfontsize}\WithValues{\meta{Schriftgröße}}\AndDefault{20pt}
Gibt die Schriftgröße des Fließtextes im Dokument an, falls der \opt{transparency}-Modus gewählt wurde.
}


\localeDE{\subsubsection{Absatzauszeichnung}}

\localeDE{
\Option{parindent}\WithValues{true, false}\AndDefault{false}
Absatzeinzug kann (de-)aktiviert werden.
}

\medskip

\localeDE{
\Option{parskip}\WithValues{true, false}\AndDefault{true}
Absatzabstand kann (de-)aktiviert werden.
}


\localeDE{\subsubsection{Seitenränder}}

\localeDE{
\Option{top}\WithValues{\meta{Oberer Seitenrand}}\AndDefault{15mm}
\Option{right}\WithValues{\meta{Rechter Seitenrand}}\AndDefault{15mm}
\Option{bottom}\WithValues{\meta{Unterer Seitenrand}}\AndDefault{15mm}
\Option{left}\WithValues{\meta{Linker Seitenrand}}\AndDefault{22.5mm}
Diese Optionen dienen der Anpassung der Seitenränder. Sie beziehen sich auf den gesamten Inhalt der Seite, inklusive Kopf- und Fußzeile.
}

\medskip

\localeDE{
\Option{twouptop}\WithValues{\meta{Oberer Seitenrand}}\AndDefault{20mm}
\Option{twoupright}\WithValues{\meta{Rechter Seitenrand}}\AndDefault{20mm}
\Option{twoupbottom}\WithValues{\meta{Unterer Seitenrand}}\AndDefault{20mm}
\Option{twoupleft}\WithValues{\meta{Linker Seitenrand}}\AndDefault{25mm}
Diese Optionen dienen der Anpassung der Seitenränder im \opt{twoup}-Modus. Sie beziehen sich auf den gesamten Inhalt der Seite, inklusive Kopf- und Fußzeile.
}

\medskip

\localeDE{
\Option{transparencytop}\WithValues{\meta{Oberer Seitenrand}}\AndDefault{10mm}
\Option{transparencyright}\WithValues{\meta{Rechter Seitenrand}}\AndDefault{10mm}
\Option{transparencybottom}\WithValues{\meta{Unterer Seitenrand}}\AndDefault{10mm}
\Option{transparencyleft}\WithValues{\meta{Linker Seitenrand}}\AndDefault{22.5mm}
Diese Optionen dienen der Anpassung der Seitenränder im \opt{transparency}-Modus. Sie beziehen sich auf den gesamten Inhalt der Seite, inklusive Kopf- und Fußzeile.
}

\medskip

\localeDE{
\Option{gluetop}\WithValues{\meta{Oberer Seitenrand}}\AndDefault{10mm}
\Option{glueright}\WithValues{\meta{Rechter Seitenrand}}\AndDefault{15mm}
\Option{gluebottom}\WithValues{\meta{Unterer Seitenrand}}\AndDefault{15mm}
\Option{glueleft}\WithValues{\meta{Linker Seitenrand}}\AndDefault{10mm}
Diese Optionen dienen der Anpassung der Seitenränder im \opt{glue}-Modus. Sie beziehen sich auf den gesamten Inhalt der Seite, inklusive Kopf- und Fußzeile.
}

\localeDE{\subsubsection{Teile (Parts)}}

\localeDE{
\Option{parts}\WithValues{true, false}\AndDefault{false}
Werden im Dokument Teile als Gliederungsebene verwendet (\cs*{part}), sollte diese Option gewählt werden. Es wird dann die Schriftgröße anderer Textelemente angemessen vergrößert.
}

\localeDE{\subsubsection{Listen}}

\localeDE{
\Option{listarraysep}\WithValues{\meta{Länge}}\AndDefault{0.5em}
Bestimmt in Aufgaben den Abstand zwischen Aufgabennummerierung und dem folgenden Text (s.\,Abb.\,\ref{fig:lists}).

\begin{figure}[h!tp]
  \centering
  \localeDE{\caption{\opt*{listarraymargin} und \opt*{listarraysep}.}}
  \label{fig:lists}
  \medskip
  \setlength{\fboxsep}{0cm}
 
 \noindent \tikz{\draw (0,0cm) -- (0, 1.3cm) node[right, yshift=-5pt]{\localeDE{Textrand}}; \draw[<->] (0,2.5pt) -- node[above]{\opt*{listarraymargin}} +(3cm,0);}\fbox{\textsf{m)}}\tikz{\draw (0,0); \draw[<->] (0,2.5pt) -- node[above]{\opt*{listarraysep}} +(2.5cm,0);}\fbox{Lorem ipsum \dots}
\end{figure}
}

 


\localeDE{
\Option{listarraymargin}\WithValues{\meta{Länge}}\AndDefault{0.5em}
Bestimmt in Aufgaben den Abstand zwischen Textrand und Aufgabennummerierung (s.\,Abb.\,\ref{fig:lists}).

\LongWarning{Um ein einheitliches Bild zu gewährleisten, wirken sich die Optionen \opt*{listarraysep} und \opt*{listarraymargin} analog auch auf \env*{itemize} und \env*{enumerate} aus.}
}


\localeDE{\subsubsection{Metadaten}}

\localeDE{
\Option{authorstyle}\WithValues{\meta{Format}}\AndDefault{\cs*{large}\cs*{sffamily}\cs*{scshape}}
\Option{classstyle}\WithValues{\meta{Format}}\AndDefault{\cs*{large}\cs*{sffamily}}
\Option{datestyle}\WithValues{\meta{Format}}\AndDefault{\cs*{small}\cs*{sffamily}}
\Option{emailstyle}\WithValues{\meta{Format}}\AndDefault{\cs*{footnotesize}\cs*{sffamily}}
\Option{fieldstyle}\WithValues{\meta{Format}}\AndDefault{\cs*{large}\cs*{sffamily}}
\Option{groupstyle}\WithValues{\meta{Format}}\AndDefault{\cs*{Large}\cs*{sffamily}\cs*{bfseries}}
\Option{licensestyle}\WithValues{\meta{Format}}\AndDefault{\cs*{small}\cs*{sffamily}}
\Option{subjectstyle}\WithValues{\meta{Format}}\AndDefault{\cs*{large}\cs*{sffamily}}
\Option{subtitlestyle}\WithValues{\meta{Format}}\AndDefault{\cs*{Large}\cs*{sffamily}\cs*{bfseries}}
\Option{titlestyle}\WithValues{\meta{Format}}\AndDefault{\cs*{LARGE}\cs*{sffamily}\cs*{bfseries}}
\Option{versionstyle}\WithValues{\meta{Format}}\AndDefault{\cs*{small}\cs*{sffamily}}
Mit diesen Optionen kann das Erscheinungsbild des Titels angepasst werden.

\WarningStylethemeDE
}


\localeDE{\subsubsection{Titel}}

\localeDE{
\Option{titleskip}\WithValues{\meta{Abstand}}\AndDefault{1.75cm}
Definiert den Abstand zwischen dem gesamten Titel und dem folgenden Inhalt des Dokuments. Dies betrifft jedoch nur den ausführlichen Titel, erzeugt durch \cs{maketitle}, nicht den kurzen Titel (\cs{maketitle*}).
}
\medskip

\localeDE{
\Option{titlefg}\WithValues{\meta{Farbe}}\AndDefault{black}
\Option{titlebg}\WithValues{\meta{Farbe}}\AndDefault{white}
Bestimmen die Vorder- und Hintergrundfarbe des Titels.
}
\medskip

\localeDE{
\Option{groupfg}\WithValues{\meta{Farbe}}\AndDefault{white}
\Option{groupbg}\WithValues{\meta{Farbe}}\AndDefault{black}
Bestimmen die Vorder- und Hintergrundfarbe der Gruppe.

\WarningColorthemeDE*
}



\localeDE{\subsubsection{Kopf- und Fußzeile}}

\localeDE{
\Option{headerrulewidth}\WithValues{\meta{Breite}}\AndDefault{0.5pt}
Bestimmt die Breite der Trennlinie der Kopfzeile an.
}
\medskip

\localeDE{
\Option{footer}\WithValues{true, false}\AndDefault{true}
Schaltet die Fußzeile ein oder aus.
}
\medskip

\localeDE{
\Option{pagecount}\WithValues{true, false}\AndDefault{true}
Bestimmt, ob die Gesamtzahl der Seiten des Dokuments in der Fußzeile angezeigt werden soll.
}
\medskip

\localeDE{
\Option{footskip}\WithValues{\meta{Abstand}}\AndDefault{1cm}
\Option{twoupfootskip}\WithValues{\meta{Abstand}}\AndDefault{0.75cm}
Bestimmt, den Abstand zwischen der Grundlinie der Fußzeile und der Grundlinie der letzten Zeile des Seiteninhalts. \opt*{twoupfootskip} bezieht sich hierbei auf den \opt{twoup}-Modus.
}


\localeDE{\subsubsection{Inhaltsverzeichnis}}

\localeDE{
\Option{exetoc}\WithValues{true, false}\AndDefault{false}
Mit dieser Option kann bestimmt werden, ob Aufgaben, Unteraufgaben, Lösungen und Unterlösungen im Inhaltsverzeichnis aufgeführt werden.
}


\localeDE{\subsubsection{Aufzählungen, Nummerierungen und Beschreibungen}}

\localeDE{
\Option{itemizefg}\WithValues{\meta{Farbe}}\AndDefault{black}
\Option{enumeratefg}\WithValues{\meta{Farbe}}\AndDefault{black}
\Option{descriptionfg}\WithValues{\meta{Farbe}}\AndDefault{black}
Diese Optionen bestimmen die Farbe der Beschriftung von Aufzählungen, Nummerierungen und Beschreibungen.

\WarningColorthemeDE*
}


\localeDE{\subsubsection{Überschriften}}

\localeDE{Zum einen können die Schriftgrößen der Überschriften angegeben werden:}

\localeDE{
\Option{partnumbersize}\WithValues{\meta{Größe}}\AndDefault{\cs*{Large}}
\Option{sectionnumbersize}\WithValues{\meta{Größe}}\AndDefault{\cs*{normalsize}}
\Option{subsectionnumbersize}\WithValues{\meta{Größe}}\AndDefault{\cs*{footnotesize}}
Diese Optionen bestimmen die Schriftgrößen von Überschriften (\cs*{part}, \cs*{section} und \cs*{subsection}).
}

\localeDE{Zum anderen können die Farben der Überschriften angegeben werden:}

\localeDE{
\Option{partnumberfg}\WithValues{\meta{Farbe}}\AndDefault{white}
\Option{partnumberbg}\WithValues{\meta{Farbe}}\AndDefault{black}
\Option{sectionnumberfg}\WithValues{\meta{Farbe}}\AndDefault{white}
\Option{sectionnumberbg}\WithValues{\meta{Farbe}}\AndDefault{black}
\Option{subsectionnumberfg}\WithValues{\meta{Farbe}}\AndDefault{white}
\Option{subsectionnumberbg}\WithValues{\meta{Farbe}}\AndDefault{black}
Farben der Nummerierung von Überschriften (\cs*{part}), \cs*{section} und \cs*{subsection}).
}
\medskip

\localeDE{
\Option{partfg}\WithValues{\meta{Farbe}}\AndDefault{black}
\Option{sectionfg}\WithValues{\meta{Farbe}}\AndDefault{black}
\Option{subsectionfg}\WithValues{\meta{Farbe}}\AndDefault{black}
Farben von Überschriften (\cs*{part}), \cs*{section} und \cs*{subsection}).
}

\localeDE{\subsubsection{Verweise}}


% TODO Makros in beschreibung falsch?

\localeDE{\thepackage\ definiert die Makros \cs{seeappendix}, \cs{seeexercise}, \cs{seefigurelabel}, \cs{seelistinglabel}, \cs{seesectionlabel} und \cs{seesolutionlabel} für Verweise auf die entsprechenden Textelemente (Aufgaben, Lösungen, Abbildungen, Abschnitte), standardmäßig in der Form (s.\,Abb.\,X). Abbildung~\ref{fig:see} veranschaulicht die hierbei verwendeten Optionen am Beispiel eines Verweises auf eine Abbildung (deshalb \opt*{seefigurelabel}).



\begin{figure}[h!tp]
  \centering
  \localeDE{\caption{Aufbau von Verweisen}}
  \label{fig:see}
  \medskip
  \setlength{\fboxsep}{0cm}
	\begin{tikzpicture}[font=\huge]
		\node[draw, inner sep=2pt] (seeleft) at (-0.55,0) {(};
		\node[draw, inner sep=2pt] (seelabel) at (0,0) {s.};
		\node[draw, inner sep=2pt] (seefigurelabel) at (3.25,0) {Abb.};
		\node[draw, inner sep=2pt] (seeref) at (6,0) {5};
		\node[draw, inner sep=2pt] (seeright) at (6.49,0) {)};
		\draw[<->] (seelabel) -- node[above, font=\small] {seelabelsep} (seefigurelabel);
		\draw[<->] (seefigurelabel) -- node[above, font=\small] {seerefsep} (seeref);
		\begin{scope}[font=\small, inner sep=2pt]
			\node[below = 0.5 of seeleft] (seeleftl) {seeleft};
			\node[below right = 0.3 of seelabel] (seelabell) {seelabel};
			\node[below = 0.5 of seefigurelabel] (seefigurelabell) {seefigurelabel};
			\node[below left = 0.3 of seeref] (seerefl) {seeref};
			\node[below = 0.5 of seeright] (seerightl) {seeright};
		\end{scope}
		\draw[->] (seeleftl) -- (seeleft);
		\draw[->] (seelabell) -- (seelabel);
		\draw[->] (seefigurelabell) -- (seefigurelabel);
		\draw[->] (seerefl) -- (seeref);
		\draw[->] (seerightl) -- (seeright);
	\end{tikzpicture}
\end{figure}
}

\localeDE{
\Option{seelabel}\WithValues{\meta{Text}}\AndDefault{s.}
Abkürzung für \emph{siehe} (\emph{see}).
}
\medskip

\localeDE{
\Option{seeappendixlabel}\WithValues{\meta{Text}}\AndDefault{Anh.}
\Option{seeexerciselabel}\WithValues{\meta{Text}}\AndDefault{Aufg.}
\Option{seefigurelabel}\WithValues{\meta{Text}}\AndDefault{Abb.}
\Option{seelistinglabel}\WithValues{\meta{Text}}\AndDefault{List.}
\Option{seesectionlabel}\WithValues{\meta{Text}}\AndDefault{Abschn.}
\Option{seesolutionlabel}\WithValues{\meta{Text}}\AndDefault{L\"os.}
Abkürzungen für \emph{Aufgabe} (\emph{Exercise}), \emph{Abbildung} (\emph{Figure}), \emph{Listing} (\emph{Listing}), \emph{Abschnitt} (\emph{Section}) und \emph{Lösung} (\emph{Solution}) (s.\,Abb.\,\ref{fig:see}).
}
\medskip

\localeDE{
\Option{seelabelsep}\WithValues{\meta{Trenner}}\AndDefault{\cs*{,}}
\Option{seerefsep}\WithValues{\meta{Trenner}}\AndDefault{\cs*{,}}
Trenner/Abstände nach \opt*{seelabel} (\emph{siehe}) und \opt*{seeXlabel} (\emph{Aufg.}, \emph{Abb.}, \dots) (s.\,Abb.\,\ref{fig:see}).
}
\medskip

\localeDE{
\Option{seeleft}\WithValues{\meta{Begrenzer}}\AndDefault{(}
\Option{seeright}\WithValues{\meta{Begrenzer}}\AndDefault{)}
Begrenzer für Verweise (s.\,Abb.\,\ref{fig:see}).
}

\localeDE{\subsubsection{Schulbuch-Aufgaben angeben}}

\localeDE{
\Option{pglabel}\WithValues{\meta{Text}}\AndDefault{S.}
\Option{nolabel}\WithValues{\meta{Begrenzer}}\AndDefault{Nr.}
Abkürzungen für \emph{Seite} (\emph{page} -- \emph{pg}) und \emph{Nummer} (\emph{Number} -- \emph{no}) im Makro \cs{pgno}.
}

\localeDE{\subsubsection{Formatierungen}}

\localeDE{
\Option{cemphfg}\WithValues{\meta{Farbe}}\AndDefault{wuSemiDarkRed}
Das Makro \cs{cemph} zeichnet Text durch farbigen Satz in \cs*{emph} aus. Die verwendete Farbe kann durch diese Option bestimmt werden.

\WarningColorthemeDE
}

\localeDE{\subsubsection{Symbole}}

\localeDE{
\Option{ccscale}\WithValues{\meta{Skalierungsfaktor}}\AndDefault{1.5}
\thepackage\ definiert skalierte Versionen der Icons der Creative-Commons-Lizenzen. Der Faktor der Skalierung kann durch diese Option bestimmt werden.
}
\medskip

\localeDE{
\Option{actionfg}\WithValues{\meta{Farbe}}\AndDefault{black}
\Option{speechfg}\WithValues{\meta{Farbe}}\AndDefault{black}
\thepackage\ definiert Symbole für Handlungen und Sprache, die in Ablauflisten verwendet werden können (\cs{action}, \cs{speech}, \cs{itemta}, \cs{itemts}, \cs{itemsa} und \cs{itemss}). Die Farbe dieser Symbole kann durch diese Optionen bestimmt werden.

\WarningColorthemeDE*
}

\localeDE{\subsubsection{Themes}}

\localeDE{
\Option{colorthemes}\WithValues{\meta{Themename}}\AndDefault{}
Mithilfe dieser Option kann ein Colortheme geladen werden.
}
\medskip

\localeDE{
\Option{styletheme}\WithValues{\meta{Themename}}\AndDefault{}
Mithilfe dieser Option kann ein Styletheme geladen werden.
}

\localeDE{\subsubsection{Grafik}}

\localeDE{
\Option{graphicspath}\WithValues{\meta{Pfad}}\AndDefault{img/}
Durch \cs*{includegraphics} eingebundene Grafiken sollten sich -- um Ordnung zu wahren -- in einem Unterverzeichnis befinden. Der Pfad dieses Verzeichnisses kann durch diese Option angegeben werden. Dies geschieht relativ zum Verzeichnis, in dem sich die entsprechende \texttt{tex}-Datei befindet. Standardmäßig muss ein Ordner \texttt{img} verwendet werden. Die Angabe muss durch einen Schrägstrich beendet werden.
}
\medskip

\localeDE{
\Option{tikzpath}\WithValues{\meta{Pfad}}\AndDefault{tikz/}
Durch \cs*{tikzinput} bzw. \cs*{tikzinput*} eingebundene TikZ-Dateien sollten sich -- um Ordnung zu wahren -- in einem Unterverzeichnis befinden. Der Pfad dieses Verzeichnisses kann durch diese Option angegeben werden. Dies geschieht relativ zum Verzeichnis, in dem sich die entsprechende \texttt{tex}-Datei befindet. Standardmäßig muss ein Ordner \texttt{tikz} verwendet werden. Die Angabe muss durch einen Schrägstrich beendet werden.
}

\localeDE{\subsubsection{Aufgaben}}

\localeDE{
\Option{exelabel}\WithValues{\meta{Beschriftung}}\AndDefault{Aufgabe}
\Option{subexelabel}\WithValues{\meta{Beschriftung}}\AndDefault{}
Mithilfe dieser Option kann die Beschriftung von Aufgaben und Unteraufgaben angegeben werden.
}
\medskip

\localeDE{
\Option{exenumberstyle}\WithValues{\meta{Format}}\AndDefault{\cs*{footnotesize}}
\Option{exenumberseparator}\WithValues{\meta{Trenner}}\AndDefault{.}
Mithilfe dieser Optionen kann die Formatierung der Aufgabennummerierung und ein auf die Nummerierung folgender Trenner bestimmt werden.

\WarningStylethemeDE*
}
\medskip

\localeDE{
\Option{exelabelstyle}\WithValues{\meta{Format}}\AndDefault{\cs*{large}\cs*{bfseries}\cs*{sffamily}}
\Option{exestyle}\WithValues{\meta{Format}}\AndDefault{\cs*{large}\cs*{sffamily}}
\Option{exepointsstyle}\WithValues{\meta{Format}}\AndDefault{\cs*{small}\cs*{bfseries}\cs*{sffamily}}
Mithilfe dieser Optionen kann die Formatierung von Aufgabenbeschriftung, Aufgabenname und Punktzahl bestimmt werden.

\WarningStylethemeDE*
}
\medskip

\localeDE{
\Option{exenumberfg}\WithValues{\meta{Farbe}}\AndDefault{white}
\Option{exenumberbg}\WithValues{\meta{Farbe}}\AndDefault{black}
\Option{exelabelfg}\WithValues{\meta{Farbe}}\AndDefault{black}
\Option{exelabelbg}\WithValues{\meta{Farbe}}\AndDefault{white}
\Option{exefg}\WithValues{\meta{Farbe}}\AndDefault{black}
\Option{exebg}\WithValues{\meta{Farbe}}\AndDefault{white}
\Option{exepointsfg}\WithValues{\meta{Farbe}}\AndDefault{black}
Mithilfe dieser Optionen können die Farben von Aufgabennummer, Aufgabenbeschriftung, Aufgabenname und Punktzahl bestimmt werden.

\WarningColorthemeDE*
}
\medskip

\localeDE{
\Option{subexenumberstyle}\WithValues{\meta{Format}}\AndDefault{\cs*{scriptsize}}
\Option{subexenumberseparator}\WithValues{\meta{Trenner}}\AndDefault{}
Mithilfe dieser Optionen kann die Formatierung der Unteraufgabennummerierung und ein auf die Nummerierung folgender Trenner bestimmt werden.

\WarningStylethemeDE*
}
\medskip

\localeDE{
\Option{subexelabelstyle}\WithValues{\meta{Format}}\AndDefault{\cs*{bfseries}\cs*{sffamily}}
\Option{subexestyle}\WithValues{\meta{Format}}\AndDefault{\cs*{sffamily}}
\Option{subexepointsstyle}\WithValues{\meta{Format}}\AndDefault{\cs*{scriptsize}\cs*{bfseries}\cs*{sffamily}}
Mithilfe dieser Optionen kann die Formatierung von Unteraufgaben bestimmt werden: Beschriftung, Name und Punktzahl.

\WarningStylethemeDE*
}
\medskip

\localeDE{
\Option{subexenumberfg}\WithValues{\meta{Farbe}}\AndDefault{white}
\Option{subexenumberbg}\WithValues{\meta{Farbe}}\AndDefault{black}
\Option{subexelabelfg}\WithValues{\meta{Farbe}}\AndDefault{black}
\Option{subexelabelbg}\WithValues{\meta{Farbe}}\AndDefault{white}
\Option{subexefg}\WithValues{\meta{Farbe}}\AndDefault{black}
\Option{subexebg}\WithValues{\meta{Farbe}}\AndDefault{white}
\Option{subexepointsfg}\WithValues{\meta{Farbe}}\AndDefault{black}
Mithilfe dieser Optionen können die Farben von Unteraufgaben bestimmt werden: Nummerierung, Beschriftung, Name und Punktzahl.

\WarningColorthemeDE*
}
\medskip

\localeDE{
\Option{multiexenumberstyle}\WithValues{\meta{Format}}\AndDefault{}
\Option{multiexepointsstyle}\WithValues{\meta{Format}}\AndDefault{\cs*{sffamily}\cs*{footnotesize}\cs*{bfseries}}
Mithilfe dieser Optionen kann die Formatierung von Teilaufgaben bestimmt werden: Beschriftung und Punktzahl.

\WarningStylethemeDE*
}
\medskip

\localeDE{
\Option{multiexenumberfg}\WithValues{\meta{Farbe}}\AndDefault{black}
\Option{multiexepointsfg}\WithValues{\meta{Farbe}}\AndDefault{black}
Mithilfe dieser Optionen können die Farben von Unteraufgaben bestimmt werden: Nummerierung und Punktzahl.

\WarningColorthemeDE*
}
\medskip

\localeDE{
\Option{multiexenumberleft}\WithValues{\meta{Abstand}}\AndDefault{}
\Option{multiexenumberright}\WithValues{\meta{Abstand}}\AndDefault{)}
Bestimmt Begrenzer der Teilaufgabennummerierung.
}
\medskip

\localeDE{
\Option{exepointslabel}\WithValues{\meta{Text}}\AndDefault{\cs*{,}P}
\Option{subexepointslabel}\WithValues{\meta{Text}}\AndDefault{\cs*{,}P}
\Option{multiexepointslabel}\WithValues{\meta{Text}}\AndDefault{}
Bestimmt die Beschriftung von Punkten in Aufgaben, Unteraufgaben und Teilaufgaben.
}
\medskip

\localeDE{
\Option{exepointsleft}\WithValues{\meta{Begrenzer}}\AndDefault{[}
\Option{exepointsright}\WithValues{\meta{Begrenzer}}\AndDefault{]}
\Option{subexepointsleft}\WithValues{\meta{Begrenzer}}\AndDefault{[}
\Option{subexepointsright}\WithValues{\meta{Begrenzer}}\AndDefault{]}
\Option{multiexepointsleft}\WithValues{\meta{Begrenzer}}\AndDefault{[}
\Option{multiexepointsright}\WithValues{\meta{Begrenzer}}\AndDefault{]}
Bestimmt Begrenzer von Punkten in Aufgaben, Unteraufgaben und Teilaufgaben.
}
\medskip

\localeDE{
\Option{exebeforeskip}\WithValues{\meta{Abstand}}\AndDefault{2.25ex}
\Option{exeafterskip}\WithValues{\meta{Abstand}}\AndDefault{0.5ex}
Mithilfe dieser Optionen können die Abstände vor und nach Aufgaben(-überschriften) bestimmt werden.
}
\medskip

\localeDE{
\Option{subexebeforeskip}\WithValues{\meta{Abstand}}\AndDefault{1.5ex}
\Option{subexeafterskip}\WithValues{\meta{Abstand}}\AndDefault{0.5ex}
Mithilfe dieser Optionen können die Abstände vor und nach Unteraufgaben(-überschriften) bestimmt werden.
}
\medskip

\localeDE{
\Option{arraybeforeskip}\WithValues{\meta{Abstand}}\AndDefault{0.5\cs*{baselineskip}}
\Option{arrayafterskip}\WithValues{\meta{Abstand}}\AndDefault{0.5\cs*{baselineskip}}
Mithilfe dieser Optionen können die Abstände vor und nach Teilaufgaben-Aufzählungen bestimmt werden.
}
\medskip


\localeDE{\subsubsection{Lösungen}}

\localeDE{
\Option{sollabel}\WithValues{\meta{Text}}\AndDefault{L\"osung}
\Option{subsollabel}\WithValues{\meta{Text}}\AndDefault{}
Mithilfe dieser Option kann die Beschriftung von Lösungen und UnterLösungen angegeben werden.
}
\medskip

\localeDE{
\Option{solnumberstyle}\WithValues{\meta{Format}}\AndDefault{\cs*{footnotesize}}
\Option{solnumberseparator}\WithValues{\meta{Trenner}}\AndDefault{.}
Mithilfe dieser Optionen kann die Formatierung der Lösungsnummerierung und ein auf die Nummerierung folgender Trenner bestimmt werden.

\WarningStylethemeDE*
}
\medskip

\localeDE{
\Option{sollabelstyle}\WithValues{\meta{Format}}\AndDefault{\cs*{large}\cs*{bfseries}\cs*{sffamily}}
\Option{solstyle}\WithValues{\meta{Format}}\AndDefault{\cs*{large}\cs*{sffamily}}
Mithilfe dieser Optionen kann die Formatierung von Lösungsbeschriftung und Lösungsnamen bestimmt werden.

\WarningStylethemeDE*
}
\medskip

\localeDE{
\Option{solnumberfg}\WithValues{\meta{Farbe}}\AndDefault{white}
\Option{solnumberbg}\WithValues{\meta{Farbe}}\AndDefault{black}
\Option{sollabelfg}\WithValues{\meta{Farbe}}\AndDefault{black}
\Option{sollabelbg}\WithValues{\meta{Farbe}}\AndDefault{white}
\Option{solfg}\WithValues{\meta{Farbe}}\AndDefault{black}
\Option{solbg}\WithValues{\meta{Farbe}}\AndDefault{white}
Mithilfe dieser Optionen können die Farben von Lösungen bestimmt werden: Nummerierung, Beschriftung und Name.

\WarningColorthemeDE*
}
\medskip

\localeDE{
\Option{subsolnumberstyle}\WithValues{\meta{Format}}\AndDefault{\cs*{scriptsize}}
\Option{subsolnumberseparator}\WithValues{\meta{Trenner}}\AndDefault{}
Mithilfe dieser Optionen kann die Formatierung der Nummerierung von Unterlösungen und ein auf die Nummerierung folgender Trenner bestimmt werden.

\WarningStylethemeDE*
}
\medskip

\localeDE{
\Option{subsollabelstyle}\WithValues{\meta{Format}}\AndDefault{\cs*{bfseries}\cs*{sffamily}}
\Option{subsolstyle}\WithValues{\meta{Format}}\AndDefault{\cs*{sffamily}}
Mithilfe dieser Optionen kann die Formatierung von Unterlösungen bestimmt werden: Beschriftung, Name und Punktzahl.

\WarningStylethemeDE*
}
\medskip

\localeDE{
\Option{subsolnumberfg}\WithValues{\meta{Farbe}}\AndDefault{white}
\Option{subsolnumberbg}\WithValues{\meta{Farbe}}\AndDefault{black}
\Option{subsollabelfg}\WithValues{\meta{Farbe}}\AndDefault{black}
\Option{subsollabelbg}\WithValues{\meta{Farbe}}\AndDefault{white}
\Option{subsolfg}\WithValues{\meta{Farbe}}\AndDefault{black}
\Option{subsolbg}\WithValues{\meta{Farbe}}\AndDefault{white}
Mithilfe dieser Optionen können die Farben von Unterlösungen bestimmt werden: Nummerierung, Beschriftung und Name.

\WarningColorthemeDE*
}
\medskip

\localeDE{
\Option{solbeforeskip}\WithValues{\meta{Abstand}}\AndDefault{2.25ex}
\Option{solafterskip}\WithValues{\meta{Abstand}}\AndDefault{0.5ex}
Mithilfe dieser Optionen können die Abstände vor und nach Lösungen (bzw. Lösungsüberschriften) bestimmt werden.
}
\medskip

\localeDE{
\Option{subsolbeforeskip}\WithValues{\meta{Abstand}}\AndDefault{1.5ex}
\Option{subsolafterskip}\WithValues{\meta{Abstand}}\AndDefault{0.5ex}
Mithilfe dieser Optionen können die Abstände vor und nach Unterlösungen (bzw. den Überschriften von Unterlösungen) bestimmt werden.
}

\localeDE{\subsubsection{Lösungen in Aufgaben}}

\localeDE{
\Option{showresults}\WithValues{true, false}\AndDefault{false}
Mithilfe dieser Optionen können Lösungen (erstellt durch \cs{res} oder \cs{resr}) in Aufgaben angezeigt werden.
}
\medskip

\localeDE{
\Option{resultsfg}\WithValues{\meta{Farbe}}\AndDefault{gray}
Mithilfe dieser Optionen kann die Farbe der Lösung bestimmt werden.

\WarningColorthemeDE
}
\medskip

\localeDE{
\Option{resultrule}\WithValues{\meta{Breite}}\AndDefault{0.4pt}
\Option{resultrulelength}\WithValues{\meta{Länge}}\AndDefault{5cm}
Mithilfe dieser Optionen können Breite und Länge der Linie von \cs{resr} bestimmt werden.
}

\localeDE{\subsubsection{Fragen}}

\localeDE{
\Option{questlabel}\WithValues{\meta{Text}}\AndDefault{Frage}
Die Option bestimmt die Beschriftung von Fragen.
}
\medskip

\localeDE{
\Option{questpointslabel}\WithValues{\meta{Text}}\AndDefault{\cs*{,}P}
Die Option bestimmt die Beschriftung der Punkte von Fragen.
}
\medskip

\localeDE{
\Option{questlabelstyle}\WithValues{\meta{Format}}\AndDefault{\cs*{sffamily}\cs*{bfseries}\cs*{small}}
\Option{queststyle}\WithValues{\meta{Format}}\AndDefault{\cs*{sffamily}\cs*{small}}
\Option{questpointsstyle}\WithValues{\meta{Format}}\AndDefault{\cs*{sffamily}\cs*{small}}
Bestimmen die Formatierung von Fragen: Beschriftung, Frage und Punkte.

\WarningStylethemeDE*
}
\medskip

\localeDE{
\Option{questlabelfg}\WithValues{\meta{Farbe}}\AndDefault{black}
\Option{questmclabelfg}\WithValues{\meta{Farbe}}\AndDefault{black}
Bestimmen die Farbe von Fragen: Beschriftung und Multiple-Choice-Antwortmöglichkeiten.

\WarningColorthemeDE*
}
\medskip

\localeDE{
\Option{questpointsleft}\WithValues{\meta{Begrenzer}}\AndDefault{[}
\Option{questpointsright}\WithValues{\meta{Begrenzer}}\AndDefault{]}
Mit dieser Option können die Begrenzer von Punktzahlen in Fragen bestimmt werden.
}
\medskip

\localeDE{
\Option{questpointssep}\WithValues{\meta{Abstand}}\AndDefault{0.25em}
\Option{questsep}\WithValues{\meta{Abstand}}\AndDefault{0.5em}
Mit dieser Option können horizontale Abstände von Fragen angegeben werden: Der Abstand vor der Punktzahl und der Abstand vor der Frage.
}
\medskip

\localeDE{
\Option{questbeforeskip}\WithValues{\meta{Abstand}}\AndDefault{1ex}
\Option{questafterskip}\WithValues{\meta{Abstand}}\AndDefault{0ex}
Mit dieser Option können vertikale Abstände vor und nach Fragen angegeben werden.
}
\medskip


\localeDE{\subsubsection{Notizen}}

\localeDE{
\Option{shownotes}\WithValues{true, false}\AndDefault{true}
Mit dieser Option können Notizen (\cs*{notet}, \cs*{notehr}) eingeblendet werden.
}
\medskip

\localeDE{
\Option{notetstyle}\WithValues{\meta{Format}}\AndDefault{\cs*{sffamily}}
Diese Option bestimmt die Formatierung von Notizen (\cs*{notet}).

\WarningStylethemeDE
}
\medskip

\localeDE{
\Option{notehrule}\WithValues{\meta{Breite}}\AndDefault{0.4pt}
Diese Option bestimmt Breite der Linie, die durch \cs*{notehr} erzeugt wird.
}
\medskip

\localeDE{
\Option{notetfg}\WithValues{\meta{Farbe}}\AndDefault{wuRed}
\Option{notehrfg}\WithValues{\meta{Farbe}}\AndDefault{wuRed}
Diese Option bestimmt die Farbe von Notizen (\cs*{notet} und \cs*{notehr}).

\WarningColorthemeDE*
}
\medskip

\localeDE{\subsubsection{Unterrichtsablauf}}

\localeDE{
\Option{ttendtime}\WithValues{true, false}\AndDefault{true}
Diese Option gibt an, ob die Uhrzeit am Ende der Ablauftabelle in einer zusätzlichen Zeile angezeigt werden soll.
}
\medskip

\localeDE{
\Option{tttimelabel}\WithValues{\meta{Text}}\AndDefault{Zeit}
\Option{ttstagelabel}\WithValues{\meta{Text}}\AndDefault{Phase}
\Option{ttactivitylabel}\WithValues{\meta{Text}}\AndDefault{Ablauf}
\Option{ttmethodlabel}\WithValues{\meta{Text}}\AndDefault{Methoden}
\Option{ttmedialabel}\WithValues{\meta{Text}}\AndDefault{Medien/Material}
Die Beschriftung der Spalten der Ablauftabelle kann durch diese Optionen bestimmt werden.
}
\medskip

\localeDE{
\Option{tttimewidth}\WithValues{\meta{Breite}}\AndDefault{1cm}
\Option{ttstagewidth}\WithValues{\meta{Breite}}\AndDefault{2.25cm}
\Option{ttmethodwidth}\WithValues{\meta{Breite}}\AndDefault{2cm}
\Option{ttmediawidth}\WithValues{\meta{Breite}}\AndDefault{2.75cm}
Die Breite der Spalten der Ablauftabelle im Hochformat (\cs{ttable}) kann durch diese Optionen bestimmt werden. Die Breite der Spalte mit dem Ablauf der jeweiligen Phase erstreckt sich automatisch über die restliche Zeilenbreite.
}
\medskip

\localeDE{
\Option{tttimewidthlscape}\WithValues{\meta{Breite}}\AndDefault{1cm}
\Option{ttstagewidthlscape}\WithValues{\meta{Breite}}\AndDefault{3.5cm}
\Option{ttmethodwidthlscape}\WithValues{\meta{Breite}}\AndDefault{3.5cm}
\Option{ttmediawidthlscape}\WithValues{\meta{Breite}}\AndDefault{3.5cm}
Die Breite der Spalten der Ablauftabelle im Querformat (\cs{ttable*}) kann durch diese Optionen bestimmt werden. Die Breite der Spalte mit dem Ablauf der jeweiligen Phase erstreckt sich automatisch über die restliche Zeilenbreite.
}
\medskip

\localeDE{
\Option{ttshowtime}\WithValues{true, false}\AndDefault{true}
Diese Option gibt an, ob die aktuelle Uhrzeit in der Zeitspalte jeder Zeile der Ablauftabelle angezeigt werden soll.
}
\medskip

\localeDE{
\Option{ttentrytimelabel}\WithValues{\meta{Text}}\AndDefault{}
Mithilfe dieser Option kann die Uhrzeit in der Zeitspalte jeder Zeile der Ablauftabelle mit einer Beschriftung (z.\,B. Uhr) versehen werden.
}
\medskip

\localeDE{
\Option{seqteacherlabel}\WithValues{\meta{Text}}\AndDefault{L}
\Option{seqstudentlabel}\WithValues{\meta{Text}}\AndDefault{S}
Diese Option bestimmt die Beschriftung von Lehrern und Schülern in Ablauflisten.
}
\medskip

\localeDE{
\Option{seqteacherstyle}\WithValues{\meta{Format}}\AndDefault{\cs*{sffamily}\cs*{bfseries}}
\Option{seqetudentstyle}\WithValues{\meta{Format}}\AndDefault{\cs*{sffamily}\cs*{bfseries}}
Bestimmen die Formatierung von Einträgen für Lehrer und Schüler in Ablauflisten.

\WarningStylethemeDE*
}
\medskip

\localeDE{
\Option{seqteacherfg}\WithValues{\meta{Farbe}}\AndDefault{black}
\Option{seqstudentfg}\WithValues{\meta{Farbe}}\AndDefault{black}
Bestimmen die Farbe von Einträgen für Lehrer und Schüler in Ablauflisten.

\WarningColorthemeDE*
}

\localeDE{\subsubsection{Tafelbild}}

\localeDE{
\Option{bbstyle}\WithValues{\meta{Format}}\AndDefault{\cs*{sffamily}}
Gibt die Formatierung von Tafelbildern an.

\WarningStylethemeDE*
}
\medskip

\localeDE{
\Option{bbfontsize}\WithValues{\meta{Schriftgröße}}\AndDefault{8pt}
Gibt die Schriftgröße in Tafelbildern an.
}
\medskip

\localeDE{
\Option{bbbaselineoffset}\WithValues{\meta{Abstand}}\AndDefault{1pt}
Gibt den Baselineoffset in Tafelbildern an.
}
\medskip

\localeDE{
\Option{bbheight}\WithValues{\meta{Höhe}}\AndDefault{0.25\cs*{linewidth} - 2\cs*{fboxsep}}
Bestimmt die Höhe von Tafelbildern. Der Standardwert entspricht einem Viertel der Zei"-len"-brei"-te (abzüglich der inneren Abstände vom Tafelinhalt zur Umrandung).
}
\medskip





\localeDE{\subsubsection{Mathematik}}

\localeDE{
\Option{commasep}\WithValues{true, false}\AndDefault{true}
Mithilfe dieser Option kann bestimmt werden, ob das Komma als Dezimaltrenner verwendet werden soll. Andernfalls geschieht dies durch einen Punkt.
}
\medskip

\localeDE{
\Option{amsoptions}\WithValues{\meta{Optionen}}\AndDefault{intlimits}
Möchte man gezielt Optionen an das Package \pkg{amsmath} übergeben, sollte dies durch diese Option geschehen.
}
\medskip

\localeDE{
\Option{amsthm}\WithValues{true, false}\AndDefault{true}
\Option{framedthm}\WithValues{true, false}\AndDefault{true}
\Option{thmbox}\WithValues{true, false}\AndDefault{false}
Durch diese Optionen kann angegeben werden, ob gängige deutsche \env{theorem}-Umgebungen (\emph{Satz}, \emph{Definition}, \emph{Beispiel}, \dots) erstellt werden sollen. \opt*{amsthm} erzeugt Theoreme im Stile von \pkg{amsthm}, \opt*{framedthm} Theoreme die umrahmt und farbig hinterlegt sein können und \opt*{thmbox} Theoreme im Stile von \pkg{thmbox}.

\LongWarning{\opt*{thmbox} erfordert zwingend nummerierte Theoreme. Deshalb muss in diesem Fall zusätzlich \opt{thmimpnumbered} und \opt{thmunimpnumbered} gewählt werden.}
}
\medskip

\LongWarning{Irgendein Text}

\localeDE{
\Option{thmimpnumbered}\WithValues{true, false}\AndDefault{true}
\Option{thmunimpnumbered}\WithValues{true, false}\AndDefault{true}
Durch diese Optionen kann angegeben werden, ob wichtige Theoreme (\opt*{thmimpnumbered}) oder unwichtige Theoreme (\opt*{thmunimpnumbered}) nummeriert werden sollen.
}
\medskip

\localeDE{
\Option{thmlabelfg}\WithValues{\meta{Farbe}}\AndDefault{black}
Durch diese Option kann die Farbe für die Theorembeschriftung (\emph{Satz}, \emph{Definition}, \dots) angegeben werden.

\WarningColorthemeDE
}
\medskip

\localeDE{
\Option{thmframefg}\WithValues{\meta{Farbe}}\AndDefault{wuDarkerGray}
\Option{thmframebg}\WithValues{\meta{Farbe}}\AndDefault{wuLightGray}
Durch diese Optionen können Rahmenfarbe (\opt*{thmframefg}) und die Hintergrundfarbe (\opt*{thmfragebg}) von umrahmten Theoremen angegeben werden.

\WarningColorthemeDE*
}
\medskip

\localeDE{
\Option{thmimplabelstyle}\WithValues{\meta{Format}}\AndDefault{\cs*{sffamily}\cs*{bfseries}}
\Option{thmimpnotestyle}\WithValues{\meta{Format}}\AndDefault{\cs*{sffamily}\cs*{bfseries}}
\Option{thmimpbodystyle}\WithValues{\meta{Format}}\AndDefault{}
Durch diese Optionen kann die Formatierung wichtiger Theoreme bestimmt werden: Beschriftung (\opt*{thmimplabelstyle}), Name (\opt*{thmimpnotestyle}) und Inhalt (\opt*{thmimpbodystyle}).

\WarningStylethemeDE*
}
\medskip

\localeDE{
\Option{thmunimplabelstyle}\WithValues{\meta{Format}}\AndDefault{\cs*{sffamily}\cs*{bfseries}}
\Option{thmunimpnotestyle}\WithValues{\meta{Format}}\AndDefault{\cs*{sffamily}}
\Option{thmunimpbodystyle}\WithValues{\meta{Format}}\AndDefault{}
Durch diese Optionen kann die Formatierung unwichtiger Theoreme bestimmt werden: Beschriftung (\opt*{thmunimplabelstyle}), Name (\opt*{thmunimpnotestyle}) und Inhalt (\opt*{thmunimpbodystyle}).

\WarningStylethemeDE*
}
\medskip

\localeDE{
\Option{thmdefinitionlabel}\WithValues{\meta{Name}}\AndDefault{Definition}
\Option{thmdefitheolabel}\WithValues{\meta{Name}}\AndDefault{Definition/Satz}
\Option{thmexamplelabel}\WithValues{\meta{Name}}\AndDefault{Beispiel}
\Option{thmexampleexelabel}\WithValues{\meta{Name}}\AndDefault{Beispielaufgabe}
\Option{thmhintlabel}\WithValues{\meta{Name}}\AndDefault{Hinweis}
\Option{thmremarklabel}\WithValues{\meta{Name}}\AndDefault{Bemerkung}
\Option{thmsolutionlabel}\WithValues{\meta{Name}}\AndDefault{L\"osung}
\Option{thmtheoremlabel}\WithValues{\meta{Name}}\AndDefault{Satz}
Durch diese Optionen können die Beschriftungen der vordefinierten Theoreme angegeben werden.
}


\localeDE{\subsubsection{Informatik}}

\localeDE{
\Option{lstnumberfg}\WithValues{\meta{Farbe}}\AndDefault{black}
\Option{lstkeywordfg}\WithValues{\meta{Farbe}}\AndDefault{black}
\Option{lstrulefg}\WithValues{\meta{Farbe}}\AndDefault{gray}
Durch diese Optionen kann die Farbgestaltung von Listings bestimmt werden: Zeilennummerierung (\opt*{lstnumberfg}), Schlüsselwörter der Programmiersprache (\opt*{lstkeywordfg}) und Umrandung des Codes (\opt*{lstrulefg}).

\WarningColorthemeDE*
}







%%%%%%%%%%%%%%%%%%%%%%%%%%%%%%%%%%%%%%%%%%%%%%%%%%%

%\localeDE{\subsection{Aufgaben}}
%
%\DescribeMacro\exe[<Punke>]{<Name>}


\Implementation\ExplHack

\section{Implementation}

\subsection{Basic Packages}

Die folgenden Packages werden zur Erstellung und Bearbeitung der Optionen verwendet:

\begin{MacroCode}{class}
\RequirePackage{expl3, l3keys2e}
\RequirePackage{xparse}

\ExplSyntaxOn

\end{MacroCode}

\subsection{Messages}

Messages for future use:

\begin{MacroCode}{class}
\msg_new:nnn {edu} {option-dep-enable} {If\ you\ enable\ the\ option\ #1,\ you\ have\ to\ enable\ the\ following\ option(s):\ #2.}

\end{MacroCode}

\subsection{Variants}

\begin{MacroCode}{class}
\cs_generate_variant:Nn \tl_if_eq:nnT { V }
\cs_generate_variant:Nn \tl_if_eq:nnF { V }
\cs_generate_variant:Nn \tl_if_eq:nnTF { V }

\end{MacroCode}




\subsection{Optionen}

\subsubsection{Patches}

\begin{macro}{\__edu_keys_initial:n}[1]{<key = value list>}
Da die Zuweisung initialer Werte in relativen Einheiten (em, ex etc.) bei \cs*{key}-Option durch \cs*{.initial} nicht funktioniert, werden diese mithilfe dieses Hilsmakros am Dokumentenanfang zugewiesen.
\begin{MacroCode}{class}
\cs_new:Npn \__edu_keys_initial:n #1 {
  \AtBeginDocument{
    \keys_set:nn {edu} {#1}
  }
}

\end{MacroCode}
\end{macro}

\begin{macro}{\@usetl}[1]{<token list>}
\pkg{tikz}-Optionen können (noch?) nicht in \LaTeX3 eingebettet, d.\,h. innerhalb von \cs{ExplSyntaxOn} und \cs{ExplSyntaxOff} verwendet werden. Mithilfe dieses Makros können durch \cs{keys_define} definierte Optionen dennoch in \pkg{tikz} verwendet werden.
\begin{MacroCode}{class}
\DeclareExpandableDocumentCommand \@usetl { m m } {
  \tl_use:c {g_edu_#1_#2}
}

\end{MacroCode}
\end{macro}

Im Folgenden werden die Optionen nach Kategorie deklariert.

\subsubsection{Media Types}


\begin{option}{twoup}
\begin{option}{glue}
\begin{option}{transparency}
\begin{MacroCode}{class}
\keys_define:nn {edu} {
  twoup .bool_gset:N = \g_edu_twoup_bool,                 % Print at A5-paper
  twoup .initial:n = false,
  glue .bool_gset:N = \g_edu_glue_bool,                   % Additional margin for gluing sheet into exercise book
  glue .initial:n = false,
  transparency .bool_gset:N = \g_edu_transparency_bool,   % Print at transparency
  transparency .initial:n = false
}

\end{MacroCode}
\end{option}
\end{option}
\end{option}


\subsubsection{Schriftarten}

\begin{option}{rmfont}
\begin{option}{sffont}
\begin{option}{ttfont}

Options for fonts:

\begin{MacroCode}{class}
\ExplSyntaxOn
\tl_new:N \g_edu_rmfont_tl
\tl_new:N \g_edu_sffont_tl
\tl_new:N \g_edu_ttfont_tl

\keys_define:nn {edu} {
  rmfont .choice_code:n = {
    \tl_gset:NV \g_edu_rmfont_tl \l_keys_choice_tl
  },
  rmfont .generate_choices:n = {  % option for choosing the roman font
    computermodern,
    libertine,
    palatino
  },
  rmfont .initial:n = libertine
}

\keys_define:nn {edu} {
  sffont .choice_code:n = {
    \tl_gset:NV \g_edu_sffont_tl \l_keys_choice_tl
  },
  sffont .generate_choices:n = {  % option for choosing the sans serif font
    computermodern,
    sourcesanspro,
    roboto,
    lato
  },
  sffont .initial:n = sourcesanspro
}

\keys_define:nn {edu} {
  ttfont .choice_code:n = {
    \tl_gset:NV \g_edu_ttfont_tl \l_keys_choice_tl
  },
  ttfont .generate_choices:n = {  % option for choosing the typewriter font
    beramono,
    sourcecodepro
  },
  ttfont .initial:n = sourcecodepro
}

\end{MacroCode}
\end{option}
\end{option}
\end{option}

\begin{option}{sfdefault}
\begin{option}{sfmath}

Options for fonts:

\begin{MacroCode}{class}
\ExplSyntaxOn
\keys_define:nn {edu} {
  sfdefault .bool_gset:N = \g_edu_sfdefault_bool,          % sans-serif als familydefault
  sfdefault .initial:n = false,
  sfmath .bool_gset:N = \g_edu_sfmath_bool,                % sans-serif als math font
  sfmath .initial:n = false,
}


\end{MacroCode}
\end{option}
\end{option}

\subsubsection{Schriftgröße}


\begin{option}{fontsize}
\begin{option}{transparencyfontsize}
\begin{MacroCode}{class}
\keys_define:nn {edu} {
  fontsize .dim_gset:N = \g_edu_fontsize_dim,  % Fontsize
  fontsize .initial:n = 11pt,
  twoupfontsize .dim_gset:N = \g_edu_twoupfontsize_dim, % Fontsize for twoup
  twoupfontsize .initial:n = 14pt,
  transparencyfontsize .dim_gset:N = \g_edu_transparencyfontsize_dim,                    % Fontsize for transparency
  transparencyfontsize .initial:n = 20pt,
}

\end{MacroCode}
\end{option}
\end{option}


\subsubsection{Paragraph highlighting}

\begin{option}{parindent}
\begin{option}{parskip}
\begin{MacroCode}{class}
\keys_define:nn {edu} {
  parindent .bool_gset:N = \g_edu_parindent_bool,   % Enable/disable parindent
  parindent .initial:n = false,
  parskip .bool_gset:N = \g_edu_parskip_bool,       % Enable/disable parskip
  parskip .initial:n = true
}

\end{MacroCode}
\end{option}
\end{option}

\subsubsection{Margins -- normal mode}

\begin{option}{top}
\begin{option}{right}
\begin{option}{bottom}
\begin{option}{left}
\begin{MacroCode}{class}

\keys_define:nn {edu} {
  top .dim_gset:N = \g_edu_top_dim,
  top .initial:n = 15mm,
  right .dim_gset:N = \g_edu_right_dim,
  right .initial:n = 15mm,
  bottom .dim_gset:N = \g_edu_bottom_dim,
  bottom .initial:n = 15mm,
  left .dim_gset:N = \g_edu_left_dim,
  left .initial:n = 22.5mm,
}

\end{MacroCode}
\end{option}
\end{option}
\end{option}
\end{option}

\subsubsection{Margins -- twoup mode}

\begin{option}{twouptop}
\begin{option}{twoupright}
\begin{option}{twoupbottom}
\begin{option}{twoupleft}
\begin{MacroCode}{class}
\keys_define:nn {edu} {
  twouptop .dim_gset:N = \g_edu_twouptop_dim,        % Top margin A5-print
  twouptop .initial:n = 20mm,
  twoupright .dim_gset:N = \g_edu_twoupright_dim,    % Right margin A5-print
  twoupright .initial:n = 20mm,
  twoupbottom .dim_gset:N = \g_edu_twoupbottom_dim,  % Bottom margin A5-print
  twoupbottom .initial:n = 20mm,
  twoupleft .dim_gset:N = \g_edu_twoupleft_dim,      % Left margin A5-print
  twoupleft .initial:n = 25mm,
}

\end{MacroCode}
\end{option}
\end{option}
\end{option}
\end{option}

\subsubsection{Margins -- transparency mode}

\begin{option}{transparencytop}
\begin{option}{transparencyright}
\begin{option}{transparencybottom}
\begin{option}{transparencyleft}
\begin{MacroCode}{class}
\keys_define:nn {edu} {
  transparencytop .dim_gset:N = \g_edu_transparencytop_dim,        % Top margin at transparency
  transparencytop .initial:n = 10mm,
  transparencyright .dim_gset:N = \g_edu_transparencyright_dim,    % Right margin at transparency
  transparencyright .initial:n = 10mm,
  transparencybottom .dim_gset:N = \g_edu_transparencybottom_dim,  % Bottom margin at transparency
  transparencybottom .initial:n = 10mm,
  transparencyleft .dim_gset:N = \g_edu_transparencyleft_dim,      % Left margin at transparency
  transparencyleft .initial:n = 22.5mm,
}  

\end{MacroCode}
\end{option}
\end{option}
\end{option}
\end{option}

\subsubsection{Margins -- glue mode}

\begin{option}{gluetop}
\begin{option}{glueright}
\begin{option}{gluebottom}
\begin{option}{glueleft}
\begin{MacroCode}{class}
\keys_define:nn {edu} {
  gluetop .dim_gset:N = \g_edu_gluetop_dim,        % Top margin at glue
  gluetop .initial:n = 15mm,
  glueright .dim_gset:N = \g_edu_glueright_dim,    % Right margin at glue
  glueright .initial:n = 20mm,
  gluebottom .dim_gset:N = \g_edu_gluebottom_dim,  % Bottom margin at glue
  gluebottom .initial:n = 20mm,
  glueleft .dim_gset:N = \g_edu_glueleft_dim,      % Left margin at glue
  glueleft .initial:n = 15mm,
}  

\end{MacroCode}
\end{option}
\end{option}
\end{option}
\end{option}


\subsubsection{Parts}

\begin{option}{parts}
Set part as topmost structure.
\begin{MacroCode}{class}
\keys_define:nn {edu} {
  parts .bool_gset:N = \g_edu_parts_bool,  % Set part as topmost structure element
  parts .initial:n = false
}						

\end{MacroCode}
\end{option}


\subsubsection{Lists}

\begin{option}{listarraysep}
\begin{option}{listarraymargin}
These options affect to lists and arrays:
\begin{MacroCode}{class}
\keys_define:nn {edu} {
  listarraysep .dim_gset:N = \g_edu_listarraysep_dim,
  listarraymargin .dim_gset:N = \g_edu_listarraymargin_dim,
}

\__edu_keys_initial:n {listarraysep = 0.5em}
\__edu_keys_initial:n {listarraymargin = 0.25em}

\end{MacroCode}
\end{option}
\end{option}


\subsubsection{Metadata}


\begin{option}{authorstyle}
\begin{option}{classstyle}
\begin{option}{datestyle}
\begin{option}{emailstyle}
\begin{option}{fieldstyle}
\begin{option}{groupstyle}
\begin{option}{licensestyle}
\begin{option}{subjectstyle}
\begin{option}{topicstyle}
\begin{option}{versionstyle}
Styles of metadata:
\begin{MacroCode}{class}
\keys_define:nn {edu} {
  authorstyle .tl_gset:N = \g_edu_authorstyle_tl,
  authorstyle .initial:n = \large\sffamily\scshape,
  classstyle .tl_gset:N = \g_edu_classstyle_tl,
  classstyle .initial:n = \large\sffamily,
  datestyle .tl_gset:N = \g_edu_datestyle_tl,
  datestyle .initial:n = \small\sffamily,
  emailstyle .tl_gset:N = \g_edu_emailstyle_tl,
  emailstyle .initial:n = \footnotesize\sffamily,
  fieldstyle .tl_gset:N = \g_edu_fieldstyle_tl,
  fieldstyle .initial:n = \large\sffamily,
  groupstyle .tl_gset:N = \g_edu_groupstyle_tl,
  groupstyle .initial:n = \Large\sffamily\bfseries,
  licensestyle .tl_gset:N = \g_edu_licensestyle_tl,
  licensestyle .initial:n = \small\sffamily,
  subjectstyle .tl_gset:N = \g_edu_subjectstyle_tl,
  subjectstyle .initial:n = \large\sffamily,
  subtitlestyle .tl_gset:N = \g_edu_subtitlestyle_tl,
  subtitlestyle .initial:n = \Large\sffamily\bfseries,
  titlestyle .tl_gset:N = \g_edu_titlestyle_tl,
  titlestyle .initial:n = \LARGE\sffamily\bfseries,
  versionstyle .tl_gset:N = \g_edu_versionstyle_tl,
  versionstyle .initial:n = \small\sffamily,
}

\end{MacroCode}
\end{option}
\end{option}
\end{option}
\end{option}
\end{option}
\end{option}
\end{option}
\end{option}
\end{option}
\end{option}


\subsubsection{Title}


\begin{option}{titleskip}
\begin{option}{titlefg}
\begin{option}{titlebg}
\begin{option}{groupfg}
\begin{option}{groupbg}
Options of title:
\begin{MacroCode}{class}
\keys_define:nn {edu} {
  titleskip .skip_gset:N = \g_edu_titleskip_skip,
  titleskip .initial:n = 1.75cm,
  titlefg .tl_gset:N = \g_edu_titlefg_tl,
  titlefg .initial:n = black,
  titlebg .tl_gset:N = \g_edu_titlebg_tl,
  titlebg .initial:n = white,
  groupfg .tl_gset:N = \g_edu_groupfg_tl,
  groupfg .initial:n = white,
  groupbg .tl_gset:N = \g_edu_groupbg_tl,
  groupbg .initial:n = black,
}

\end{MacroCode}
\end{option}
\end{option}
\end{option}
\end{option}
\end{option}


\subsubsection{Header and Footer}

\begin{option}{headerrulewidth}
\begin{option}{footer}
\begin{option}{pagecount}
\begin{option}{footskip}
\begin{option}{twoupfootskip}
Options of header and footer:
\begin{MacroCode}{class}
\keys_define:nn {edu} {
  headerrulewidth .dim_gset:N = \g_edu_headerrulewidth_dim,  % Width of header rule
  headerrulewidth .initial:n = 0.5pt,
  footer .bool_gset:N = \g_edu_footer_bool,                   % Enable/disable footer
  footer .initial:n = true,
  pagecount .bool_gset:N = \g_edu_pagecount_bool,             % Enable/disable pagecount at footer
  pagecount .initial:n = true,
  footskip .dim_gset:N = \g_edu_footskip_dim,                % Skip to first baseline of  footer
  footskip .initial:n = 1cm,
  twoupfootskip .dim_gset:N = \g_edu_twoupfootskip_dim,      % Skip to first baseline of  footer in twoup-mode
  twoupfootskip .initial:n = 0.75cm,
}

\end{MacroCode}
\end{option}
\end{option}
\end{option}
\end{option}
\end{option}


\subsubsection{Inhaltsverzeichnis}


\begin{option}{exetoc}
Options of toc:
\begin{MacroCode}{class}
\keys_define:nn {edu} {
  exetoc .bool_gset:N = \g_edu_exetoc_bool,  % Add (sub)exercises and (sub)solutions to toc?
  exetoc .initial:n = false,
}

\end{MacroCode}
\end{option}



\subsubsection{Aufzählungen, Nummerierungen}

\begin{option}{itemizefg}
\begin{option}{enumeratefg}
\begin{option}{descriptionfg}
Options of lists:
\begin{MacroCode}{class}
\keys_define:nn {edu} {
  itemizefg .tl_gset:N = \g_edu_itemizefg_tl,          % Color of itemize-labels
  itemizefg .initial:n = black,
  enumeratefg .tl_gset:N = \g_edu_enumeratefg_tl,      % Color of enumerate-labels
  enumeratefg .initial:n = black,
  descriptionfg .tl_gset:N = \g_edu_descriptionfg_tl,  % Color of description-labels
  descriptionfg .initial:n = black,
}

\end{MacroCode}
\end{option}
\end{option}
\end{option}

\subsubsection{Parts, Überschriften, Unterüberschriften}

\begin{option}{partnumbersize}
\begin{option}{sectionnumbersize}
\begin{option}{subsectionnumbersize}
Sizes of sections:
\begin{MacroCode}{class}
\keys_define:nn {edu} {
  partnumbersize .tl_gset:N = \g_edu_partnumbersize_tl,              % Size of part numbers
  partnumbersize .initial:n = \Large,
  sectionnumbersize .tl_gset:N = \g_edu_sectionnumbersize_tl,        % Size of section numbers
  sectionnumbersize .initial:n = \normalsize,
  subsectionnumbersize .tl_gset:N = \g_edu_subsectionnumbersize_tl,  % Size of subsection numbers
  subsectionnumbersize .initial:n = \footnotesize,
}

\end{MacroCode}
\end{option}
\end{option}
\end{option}


\begin{option}{partnumberfg}
\begin{option}{partnumberbg}
\begin{option}{partfg}
\begin{option}{sectionnumberfg}
\begin{option}{sectionnumberbg}
\begin{option}{sectionfg}
\begin{option}{subsectionnumberfg}
\begin{option}{subsectionnumberbg}
\begin{option}{subsectionfg}
Styles of sections:
\begin{MacroCode}{class}
\keys_define:nn {edu} {
  partnumberfg .tl_gset:N = \g_edu_partnumberfg_tl,             % Foreground color of part numbers
  partnumberfg .initial:n = white,
  partnumberbg .tl_gset:N = \g_edu_partnumberbg_tl,             % Background color of part numbers
  partnumberbg .initial:n = black,
  partfg .tl_gset:N = \g_edu_partfg_tl,                         % Foreground color of parts
  partfg .initial:n = black,
  sectionnumberfg .tl_gset:N = \g_edu_sectionnumberfg_tl,       % Foreground color of section numbers
  sectionnumberfg .initial:n = white,
  sectionnumberbg .tl_gset:N = \g_edu_sectionnumberbg_tl,       % Background color of section numbers
  sectionnumberbg .initial:n = black,
  sectionfg .tl_gset:N = \g_edu_sectionfg_tl,                   % Foreground color of sections
  sectionfg .initial:n = black,
  subsectionnumberfg .tl_gset:N = \g_edu_subsectionnumberfg_tl, % Foreground color of subsection numbers
  subsectionnumberfg .initial:n = white,
  subsectionnumberbg .tl_gset:N = \g_edu_subsectionnumberbg_tl, % Background color of subsection numbers
  subsectionnumberbg .initial:n = black,
  subsectionfg .tl_gset:N = \g_edu_subsectionfg_tl,             % Foreground color of subsection
  subsectionfg .initial:n = black,
}

\end{MacroCode}
\end{option}
\end{option}
\end{option}
\end{option}
\end{option}
\end{option}
\end{option}
\end{option}
\end{option}


\subsubsection{Hyperref}

\begin{option}{colorlinks}
\begin{option}{linkfg}
\begin{option}{linkborderfg}
Options of hyperrefs:
\begin{MacroCode}{class}
\keys_define:nn {edu} {
  colorlinks .bool_gset:N = \g_edu_colorlinks_bool,
  colorlinks .initial:n = true,
  linkfg .tl_gset:N = \g_edu_linkfg_tl,
  linkfg .initial:n = black,
  linkborderfg .tl_gset:N = \g_edu_linkborderfg_tl,
  linkborderfg .initial:n = {1~0~0},
}

\end{MacroCode}
\end{option}
\end{option}
\end{option}


\subsubsection{Siehe Abschnitt, siehe Abbildung, etc.}

\begin{option}{seelabel}
\begin{option}{seeappendixlabel}
\begin{option}{seeexerciselabel}
\begin{option}{seefigurelabel}
\begin{option}{seelistinglabel}
\begin{option}{seesectionlabel}
\begin{option}{seesolutionlabel}
\begin{option}{seelabelsep}
\begin{option}{seerefsep}
\begin{option}{seeleft}
\begin{option}{seeright}
Options \cs{see}-commands:
\begin{MacroCode}{class}
\keys_define:nn {edu} {
  seelabel .tl_gset:N = \g_edu_seelabel_tl,% Label of 'see'
  seelabel .initial:n = s.,
  seeappendixlabel .tl_gset:N = \g_edu_seeappendixlabel_tl,% Label of 'appendix'
  seeappendixlabel .initial:n = Anh.,
  seeexerciselabel .tl_gset:N = \g_edu_seeexerciselabel_tl,% Label of 'exercise'
  seeexerciselabel .initial:n = Aufg.,
  seefigurelabel .tl_gset:N = \g_edu_seefigurelabel_tl,      % Label of 'figure'
  seefigurelabel .initial:n = Abb.,
  seelistinglabel .tl_gset:N = \g_edu_seelistinglabel_tl,    % Label of 'listing'
  seelistinglabel .initial:n = List.,
  seesectionlabel .tl_gset:N = \g_edu_seesectionlabel_tl,    % Label of 'section'
  seesectionlabel .initial:n = Abschn.,
  seesolutionlabel .tl_gset:N = \g_edu_seesolutionlabel_tl,  % Label of 'solution'
  seesolutionlabel .initial:n = L\"os.,
  seelabelsep .tl_gset:N = \g_edu_seelabelsep_tl,            % First seperator of 'see'
  seelabelsep .initial:n = {\,},
  seerefsep .tl_gset:N = \g_edu_seerefsep_tl,                % Second seperator of 'see'
  seerefsep .initial:n = {\,},
  seeleft .tl_gset:N = \g_edu_seeleft_tl,                    % Left delimiter of 'see'
  seeleft .initial:n = (,
  seeright .tl_gset:N = \g_edu_seeright_tl,                  % Right delimiter of 'see'
  seeright .initial:n = ),
}

\end{MacroCode}
\end{option}
\end{option}
\end{option}
\end{option}
\end{option}
\end{option}
\end{option}
\end{option}
\end{option}
\end{option}
\end{option}


\subsubsection{Aufgaben angeben: S.\,X, Nr.\,Y}

\begin{option}{pglabel}
\begin{option}{nolabel}
Options exercise references (in books):
\begin{MacroCode}{class}
\keys_define:nn {edu} {
  pglabel .tl_gset:N = \g_edu_pglabel_tl,  % Label of page in 'pgno'
  pglabel .initial:n = S.,
  nolabel .tl_gset:N = \g_edu_nolabel_tl,  % Label of number in 'pgno'
  nolabel .initial:n = Nr.,
}

\end{MacroCode}
\end{option}
\end{option}


\subsubsection{Kasten \emph{Bitte wenden}}

\begin{option}{ptolabel}
\begin{option}{ptosymbol}
Options for the appearence of the 'please turn over' (p. t. o.) label:
\begin{MacroCode}{class}
\keys_define:nn {edu} {
  ptolabel .tl_gset:N = \g_edu_ptolabel_tl,
  ptolabel .initial:n = Bitte~wenden\hspace*{0.25em},
  ptolabelstyle .tl_gset:N = \g_edu_ptolabelstyle_tl,
  ptolabelstyle .initial:n = \scriptsize\sffamily,
  ptosymbol .tl_gset:N = \g_ptosymbol_tl,
  ptosymbol .initial:n = $\blacktriangleright$,
  ptosymbolstyle .tl_gset:N = \g_ptosymbolstyle_tl,
  ptosymbolstyle .initial:n = \small\sffamily,
}

\end{MacroCode}
\end{option}
\end{option}



\subsubsection{Formatierungen}

\begin{option}{cemphfg}
Formatting options:
\begin{MacroCode}{class}
\keys_define:nn {edu} {
  cemphfg .tl_gset:N = \g_edu_cemphfg_tl,  % Color of cemph
  cemphfg .initial:n = wuSemiDarkRed,
}

\end{MacroCode}
\end{option}


\subsubsection{Symbole}

\begin{option}{ccscale}
\begin{option}{actionfg}
\begin{option}{speechfg}
Various options of symbols:
\begin{MacroCode}{class}
\keys_define:nn {edu} {
  ccscale .fp_gset:N = \g_edu_ccscale_fp,    % Scaling Creative Commons Icons
  ccscale .initial:n = 1.5,
  actionfg .tl_gset:N = \g_edu_actionfg_tl,  % Color of action-symbol
  actionfg .initial:n = black,
  speechfg .tl_gset:N = \g_edu_speechfg_tl,  % Color of speech-symbol
  speechfg .initial:n = black,
}

\end{MacroCode}
\end{option}
\end{option}
\end{option}


\subsubsection{Themes}

\begin{option}{colortheme}
\begin{option}{styletheme}
Option of color- and styletheme:
\begin{MacroCode}{class}
\keys_define:nn {edu} {
  colortheme .tl_gset:N = \g_edu_colortheme_tl, % Color zheme
  colortheme .initial:n =,
  styletheme .tl_gset:N = \g_edu_styletheme_tl, % Style theme
  styletheme .initial:n =,
}

\end{MacroCode}
\end{option}
\end{option}


\subsubsection{Grafik}

\begin{option}{graphicspath}
\begin{option}{tikzpath}
Options of graphics:
\begin{MacroCode}{class}
\keys_define:nn {edu} {
  graphicspath .tl_gset:N = \g_edu_graphicspath_tl,  % Path to graphics
  graphicspath .initial:n = img/,
  tikzpath .tl_gset:N = \g_edu_tikzpath_tl,          % Path to pgf-files (for tikz)
  tikzpath .initial:n = tikz/,
}

\end{MacroCode}
\end{option}
\end{option}



\subsubsection{Aufgaben}

\begin{option}{exelabel}
\begin{option}{subexelabel}
Labels of (sub-)exercises:
\begin{MacroCode}{class}
\keys_define:nn {edu} {
  exelabel .tl_gset:N = \g_edu_exelabel_tl,
  exelabel .initial:n = Aufgabe,
  subexelabel .tl_gset:N = \g_edu_subexelabel_tl,
  subexelabel .initial:n =,
}

\end{MacroCode}
\end{option}
\end{option}

\begin{option}{exenumberstyle}
\begin{option}{exenumberseparator}
\begin{option}{exelabelstyle}
\begin{option}{exestyle}
\begin{option}{exepointsstyle}
Style of exercises:
\begin{MacroCode}{class}
\keys_define:nn {edu} {
  exenumberstyle .tl_gset:N = \g_edu_exenumberstyle_tl,
  exenumberstyle .initial:n = \footnotesize,
  exenumberseparator .tl_gset:N = \g_edu_exenumberseparator_tl,
  exenumberseparator .initial:n = .,
  exelabelstyle .tl_gset:N = \g_edu_exelabelstyle_tl,
  exelabelstyle .initial:n = \large\bfseries\sffamily,
  exestyle .tl_gset:N = \g_edu_exestyle_tl,
  exestyle .initial:n = \large\sffamily,
  exepointsstyle .tl_gset:N = \g_edu_exepointsstyle_tl,
  exepointsstyle .initial:n = \bfseries\sffamily,
}

\end{MacroCode}
\end{option}
\end{option}
\end{option}
\end{option}
\end{option}

\begin{option}{exenumberfg}
\begin{option}{exenumberbg}
\begin{option}{exelabelfg}
\begin{option}{exelabelbg}
\begin{option}{exefg}
\begin{option}{exebg}
\begin{option}{exepointsfg}
Colors of exercises:
\begin{MacroCode}{class}
\keys_define:nn {edu} {
  exenumberfg .tl_gset:N = \g_edu_exenumberfg_tl,
  exenumberfg .initial:n = white,
  exenumberbg .tl_gset:N = \g_edu_exenumberbg_tl,
  exenumberbg .initial:n = black,
  exelabelfg .tl_gset:N = \g_edu_exelabelfg_tl,
  exelabelfg .initial:n = black,
  exelabelbg .tl_gset:N = \g_edu_exelabelbg_tl,
  exelabelbg .initial:n = white,
  exefg .tl_gset:N = \g_edu_exefg_tl,
  exefg .initial:n = black,
  exebg .tl_gset:N = \g_edu_exebg_tl,
  exebg .initial:n = white,
  exepointsfg .tl_gset:N = \g_edu_exepointsfg_tl,
  exepointsfg .initial:n = black,
}

\end{MacroCode}
\end{option}
\end{option}
\end{option}
\end{option}
\end{option}
\end{option}
\end{option}

\begin{option}{subexenumberstyle}
\begin{option}{subexenumberseparator}
\begin{option}{subexelabelstyle}
\begin{option}{subexestyle}
\begin{option}{subexepointsstyle}
Style of subexercises:
\begin{MacroCode}{class}
\keys_define:nn {edu} {
  subexenumberstyle .tl_gset:N = \g_edu_subexenumberstyle_tl,
  subexenumberstyle .initial:n = \scriptsize,
  subexenumberseparator .tl_gset:N = \g_edu_subexenumberseparator_tl,
  subexenumberseparator .initial:n =,
  subexelabelstyle .tl_gset:N = \g_edu_subexelabelstyle_tl,
  subexelabelstyle .initial:n = \bfseries\sffamily,
  subexestyle .tl_gset:N = \g_edu_subexestyle_tl,
  subexestyle .initial:n = \sffamily,
  subexepointsstyle .tl_gset:N = \g_edu_subexepointsstyle_tl,
  subexepointsstyle .initial:n = \small\bfseries\sffamily,
}

\end{MacroCode}
\end{option}
\end{option}
\end{option}
\end{option}
\end{option}

\begin{option}{subexenumberfg}
\begin{option}{subexenumberbg}
\begin{option}{subexelabelfg}
\begin{option}{subexelabelbg}
\begin{option}{subexefg}
\begin{option}{subexebg}
\begin{option}{subexepointsfg}
Colors of subexercises:
\begin{MacroCode}{class}
\keys_define:nn {edu} {
  subexenumberfg .tl_gset:N = \g_edu_subexenumberfg_tl,
  subexenumberfg .initial:n = white,
  subexenumberbg .tl_gset:N = \g_edu_subexenumberbg_tl,
  subexenumberbg .initial:n = black,
  subexelabelfg .tl_gset:N = \g_edu_subexelabelfg_tl,
  subexelabelfg .initial:n = black,
  subexelabelbg .tl_gset:N = \g_edu_subexelabelbg_tl,
  subexelabelbg .initial:n = white,
  subexefg .tl_gset:N = \g_edu_subexefg_tl,
  subexefg .initial:n = black,
  subexebg .tl_gset:N = \g_edu_subexebg_tl,
  subexebg .initial:n = white,
  subexepointsfg .tl_gset:N = \g_edu_subexepointsfg_tl,
  subexepointsfg .initial:n = black,
}

\end{MacroCode}
\end{option}
\end{option}
\end{option}
\end{option}
\end{option}
\end{option}
\end{option}

\begin{option}{multiexestyle}
\begin{option}{multiexepointslabel}
Style of multiexe-environments:
\begin{MacroCode}{class}
\keys_define:nn {edu} {
  multiexenumberstyle .tl_gset:N = \g_edu_multiexenumberstyle_tl,
  multiexenumberstyle .initial:n =,
  multiexepointsstyle .tl_gset:N = \g_edu_multiexepointsstyle_tl,
  multiexepointsstyle .initial:n = \sffamily\footnotesize\bfseries,
}

\end{MacroCode}
\end{option}
\end{option}


\begin{option}{multiexenumberfg}
\begin{option}{multiexepointsfg}
Colors of multiexe-environments:
\begin{MacroCode}{class}
\keys_define:nn {edu} {
  multiexenumberfg .tl_gset:N = \g_edu_multiexenumberfg_tl,
  multiexenumberfg .initial:n = black,
  multiexepointsfg .tl_gset:N = \g_edu_multiexepointsfg_tl,
  multiexepointsfg .initial:n = black,
}

\end{MacroCode}
\end{option}
\end{option}


\begin{option}{multiexenumberleft}
\begin{option}{multiexenumberright}
Colors of multiexe-environments:
\begin{MacroCode}{class}
\keys_define:nn {edu} {
  multiexenumberleft .tl_gset:N = \g_edu_multiexenumberleft_tl,
  multiexenumberleft .initial:n =,
  multiexenumberright .tl_gset:N = \g_edu_multiexenumberright_tl,
  multiexenumberright .initial:n = ),
}

\end{MacroCode}
\end{option}
\end{option}


\begin{option}{exepointslabel}
\begin{option}{subexepointslabel}
\begin{option}{multiexepointslabel}
Labels of points:
\begin{MacroCode}{class}
\keys_define:nn {edu} {
  exepointslabel .tl_gset:N = \g_edu_exepointslabel_tl,
  exepointslabel .initial:n = \,P,
  subexepointslabel .tl_gset:N = \g_edu_subexepointslabel_tl,
  subexepointslabel .initial:n = \,P,
  multiexepointslabel .tl_gset:N = \g_edu_multiexepointslabel_tl,
  multiexepointslabel .initial:n =,
}

\end{MacroCode}
\end{option}
\end{option}
\end{option}


\begin{option}{exepointsleft}
\begin{option}{exepointsright}
\begin{option}{subexepointsleft}
\begin{option}{subexepointsright}
\begin{option}{multiexepointsleft}
\begin{option}{multiexepointsright}
Delimiter of points:
\begin{MacroCode}{class}
\keys_define:nn {edu} {
  exepointsleft .tl_gset:N = \g_edu_exepointsleft_tl,
  exepointsleft .initial:n = [,
  exepointsright .tl_gset:N = \g_edu_exepointsright_tl,
  exepointsright .initial:n = ],
  subexepointsleft .tl_gset:N = \g_edu_subexepointsleft_tl,
  subexepointsleft .initial:n = [,
  subexepointsright .tl_gset:N = \g_edu_subexepointsright_tl,
  subexepointsright .initial:n = ],
  multiexepointsleft .tl_gset:N = \g_edu_multiexepointsleft_tl,
  multiexepointsleft .initial:n = [,
  multiexepointsright .tl_gset:N = \g_edu_multiexepointsright_tl,
  multiexepointsright .initial:n = ],
}

\end{MacroCode}
\end{option}
\end{option}
\end{option}
\end{option}
\end{option}
\end{option}


\begin{option}{exepointsachieved}
\begin{option}{subexepointsachieved}
\begin{option}{exepointsachievedspace}
\begin{option}{subexepointsachievedspace}
\begin{option}{exepointsachievedsep}
\begin{option}{subexepointsachievedsep}
Achieved points:
\begin{MacroCode}{class}
\keys_define:nn {edu} {
  exepointsachieved .bool_gset:N = \g_edu_exepointsachieved_bool,
  exepointsachieved .initial:n = true,
  subexepointsachieved .bool_gset:N = \g_edu_subexepointsachieved_bool,
  subexepointsachieved .initial:n = false,
  exepointsachievedspace .dim_gset:N = \g_edu_exepointsachievedspace_dim,
  exepointsachievedspace .initial:n = 0.6cm,
  subexepointsachievedspace .dim_gset:N = \g_edu_subexepointsachievedspace_dim,
  subexepointsachievedspace .initial:n = 0.6cm,
  exepointsachievedsep .tl_gset:N = \g_edu_exepointsachievedsep_tl,
  exepointsachievedsep .initial:n = /\,,
  subexepointsachievedsep .tl_gset:N = \g_edu_subexepointsachievedsep_tl,
  subexepointsachievedsep .initial:n = /\,,
}

\end{MacroCode}
\end{option}
\end{option}
\end{option}
\end{option}
\end{option}
\end{option}


\begin{option}{exebeforeskip}
\begin{option}{exeafterskip}
\begin{option}{subexebeforeskip}
\begin{option}{subexeafterskip}
\begin{option}{arraybeforeskip}
\begin{option}{arrayafterskip}
Skips before and after (sub-)exercises:
\begin{MacroCode}{class}
\keys_define:nn {edu} {
  exebeforeskip .dim_gset:N = \g_edu_exebeforeskip_dim,
  exeafterskip .dim_gset:N = \g_edu_exeafterskip_dim,
  subexebeforeskip .dim_gset:N = \g_edu_subexebeforeskip_dim,
  subexeafterskip .dim_gset:N = \g_edu_subexeafterskip_dim,
  arraybeforeskip .dim_gset:N = \g_edu_arraybeforeskip_dim,
  arrayafterskip .dim_gset:N = \g_edu_arrayafterskip_dim,
}

\__edu_keys_initial:n {
  exebeforeskip = 2.25ex,
  exeafterskip = 0.5ex,
  subexebeforeskip = 1.5ex,
  subexeafterskip = 0.5ex,
  arraybeforeskip = 0.5\baselineskip,
  arrayafterskip= 0.5\baselineskip
}

\end{MacroCode}
\end{option}
\end{option}
\end{option}
\end{option}
\end{option}
\end{option}


\subsubsection{Lösungen}

\begin{option}{sollabel}
\begin{option}{subsollabel}
Labels of (sub-)solutions:
\begin{MacroCode}{class}
\keys_define:nn {edu} {
  sollabel .tl_gset:N = \g_edu_sollabel_tl,
  sollabel .initial:n = {L\"osung},
  subsollabel .tl_gset:N = \g_edu_subsollabel_tl,
  subsollabel .initial:n = ,
}

\end{MacroCode}
\end{option}
\end{option}

\begin{option}{solnumberstyle}
\begin{option}{solnumberseparator}
\begin{option}{sollabelstyle}
\begin{option}{solstyle}
Style of solutions:
\begin{MacroCode}{class}
\keys_define:nn {edu} {
  solnumberstyle .tl_gset:N = \g_edu_solnumberstyle_tl,
  solnumberstyle .initial:n = \footnotesize,
  solnumberseparator .tl_gset:N = \g_edu_solnumberseparator_tl,
  solnumberseparator .initial:n = .,
  sollabelstyle .tl_gset:N = \g_edu_sollabelstyle_tl,
  sollabelstyle .initial:n = \large\bfseries\sffamily,
  solstyle .tl_gset:N = \g_edu_solstyle_tl,
  solstyle .initial:n = \large\sffamily,
}

\end{MacroCode}
\end{option}
\end{option}
\end{option}
\end{option}

\begin{option}{solnumberfg}
\begin{option}{solnumberbg}
\begin{option}{sollabelfg}
\begin{option}{sollabelbg}
\begin{option}{solfg}
\begin{option}{solbg}
Colors of solutions:
\begin{MacroCode}{class}
\keys_define:nn {edu} {
  solnumberfg .tl_gset:N = \g_edu_solnumberfg_tl,
  solnumberfg .initial:n = white,
  solnumberbg .tl_gset:N = \g_edu_solnumberbg_tl,
  solnumberbg .initial:n = black,
  sollabelfg .tl_gset:N = \g_edu_sollabelfg_tl,
  sollabelfg .initial:n = black,
  sollabelbg .tl_gset:N = \g_edu_sollabelbg_tl,
  sollabelbg .initial:n = white,
  solfg .tl_gset:N = \g_edu_solfg_tl,
  solfg .initial:n = black,
  solbg .tl_gset:N = \g_edu_solbg_tl,
  solbg .initial:n = white,
}

\end{MacroCode}
\end{option}
\end{option}
\end{option}
\end{option}
\end{option}
\end{option}

\begin{option}{subsolnumberstyle}
\begin{option}{subsolnumberseparator}
\begin{option}{subsollabelstyle}
\begin{option}{subsolstyle}
Style of subsolutions:
\begin{MacroCode}{class}
\keys_define:nn {edu} {
  subsolnumberstyle .tl_gset:N = \g_edu_subsolnumberstyle_tl,
  subsolnumberstyle .initial:n = \scriptsize,
  subsolnumberseparator .tl_gset:N = \g_edu_subsolnumberseparator_tl,
  subsolnumberseparator .initial:n = ,
  subsollabelstyle .tl_gset:N = \g_edu_subsollabelstyle_tl,
  subsollabelstyle .initial:n = \bfseries\sffamily,
  subsolstyle .tl_gset:N = \g_edu_subsolstyle_tl,
  subsolstyle .initial:n = \sffamily,
}

\end{MacroCode}
\end{option}
\end{option}
\end{option}
\end{option}

\begin{option}{subsolnumberfg}
\begin{option}{subsolnumberbg}
\begin{option}{subsollabelfg}
\begin{option}{subsollabelbg}
\begin{option}{subsolfg}
\begin{option}{subsolbg}
Colors of subsolutions:
\begin{MacroCode}{class}
\keys_define:nn {edu} {
  subsolnumberfg .tl_gset:N = \g_edu_subsolnumberfg_tl,
  subsolnumberfg .initial:n = white,
  subsolnumberbg .tl_gset:N = \g_edu_subsolnumberbg_tl,
  subsolnumberbg .initial:n = black,
  subsollabelfg .tl_gset:N = \g_edu_subsollabelfg_tl,
  subsollabelfg .initial:n = black,
  subsollabelbg .tl_gset:N = \g_edu_subsollabelbg_tl,
  subsollabelbg .initial:n = white,
  subsolfg .tl_gset:N = \g_edu_subsolfg_tl,
  subsolfg .initial:n = black,
  subsolbg .tl_gset:N = \g_edu_subsolbg_tl,
  subsolbg .initial:n = white,
}

\end{MacroCode}
\end{option}
\end{option}
\end{option}
\end{option}
\end{option}
\end{option}

\begin{option}{solbeforeskip}
\begin{option}{solafterskip}
\begin{option}{subsolbeforeskip}
\begin{option}{subsolafterskip}
Skips before and after (sub-)solutions:
\begin{MacroCode}{class}
\keys_define:nn {edu} {
  solbeforeskip .dim_gset:N = \g_edu_solbeforeskip_dim,
  solafterskip .dim_gset:N = \g_edu_solafterskip_dim,
  subsolbeforeskip .dim_gset:N = \g_edu_subsolbeforeskip_dim,
  subsolafterskip .dim_gset:N = \g_edu_subsolafterskip_dim,
}

\__edu_keys_initial:n {
  solbeforeskip = 2.25ex,
  solafterskip = 0.5ex,
  subsolbeforeskip = 1.5ex,
  subsolafterskip = 0.5ex,
}

\end{MacroCode}
\end{option}
\end{option}
\end{option}
\end{option}



\subsubsection{Lösungen in Aufgaben}

\begin{option}{showresults}
\begin{option}{resultfg}
\begin{option}{resultrule}
\begin{option}{resultrulelength}
\begin{option}{resultrulevoffset}
\begin{option}{resultboxwidth}
\begin{option}{resultboxheight}
\begin{option}{resultrulevoffset}
Options for typesetting results:
\begin{MacroCode}{class}
\keys_define:nn {edu} {
  showresults .bool_gset:N = \g_edu_showresults_bool,
  showresults .initial:n = false,
  resultfg .tl_gset:N = \g_edu_resultfg_tl,
  resultfg .initial:n = gray,
  resultrule .dim_gset:N = \g_edu_resultrule_dim,
  resultrule .initial:n = 0.4pt,
  resultrulelength .dim_gset:N = \g_edu_resultrulelength_dim,
  resultrulelength .initial:n = 2cm,
  resultrulevoffset .dim_gset:N = \g_edu_resultrulevoffset_dim,
  resultrulevoffset .initial:n = -4pt,
  resultboxwidth .dim_gset:N = \g_edu_resultboxwidth_dim,
  resultboxwidth .initial:n = 2cm,
  resultboxheight .dim_gset:N = \g_edu_resultboxheight_dim,
  resultboxheight .initial:n = 0.65cm,
  resultrulevoffset .dim_gset:N = \g_edu_resultboxvoffset_dim,
  resultrulevoffset .initial:n = -5pt,
}

\end{MacroCode}
\end{option}
\end{option}
\end{option}
\end{option}
\end{option}
\end{option}
\end{option}
\end{option}


\subsubsection{Fragen}

\begin{option}{questlabel}
\begin{option}{questpointslabel}
Label of questions:
\begin{MacroCode}{class}
\keys_define:nn {edu} {
  questlabel .tl_gset:N = \g_edu_questlabel_tl,
  questlabel .initial:n = Frage,
  questpointslabel .tl_gset:N = \g_edu_questpointslabel_tl,
  questpointslabel .initial:n = \,P,
}

\end{MacroCode}
\end{option}
\end{option}

\begin{option}{questlabelstyle}
\begin{option}{queststyle}
\begin{option}{questpointsstyle}
Style of questions:
\begin{MacroCode}{class}
\keys_define:nn {edu} {
  questlabelstyle .tl_gset:N = \g_edu_questlabelstyle_tl,
  questlabelstyle .initial:n = \sffamily\bfseries\small,
  queststyle .tl_gset:N = \g_edu_queststyle_tl,
  queststyle .initial:n = \sffamily\small,
  questpointsstyle .tl_gset:N = \g_edu_questpointsstyle_tl,
  questpointsstyle .initial:n = \sffamily\small,
}

\end{MacroCode}
\end{option}
\end{option}
\end{option}

\begin{option}{questlabelfg}
\begin{option}{questmclabelfg}
Colors of questions:
\begin{MacroCode}{class}
\keys_define:nn {edu} {
  questlabelfg .tl_gset:N = \g_edu_questlabelfg_tl,
  questlabelfg .initial:n = black,
  questmclabelfg .tl_gset:N = \g_edu_questmclabelfg_tl,
  questmclabelfg .initial:n = black,
}

\end{MacroCode}
\end{option}
\end{option}

\begin{option}{questpointsleft}
\begin{option}{questpointsright}
Delimiter of question points:
\begin{MacroCode}{class}
\keys_define:nn {edu} {
  questpointsleft .tl_gset:N = \g_edu_questpointsleft_tl,
  questpointsleft .initial:n = [,
  questpointsright .tl_gset:N = \g_edu_questpointsright_tl,
  questpointsright .initial:n = ],
}

\end{MacroCode}
\end{option}
\end{option}

\begin{option}{questpointssep}
\begin{option}{questsep}
Space before and after question points:
\begin{MacroCode}{class}
\keys_define:nn {edu} {
  questpointssep .dim_gset:N = \g_edu_questpointssep_dim,
  questpointssep .initial:n = 0.25em,
  questsep .dim_gset:N = \g_edu_questsep_dim,
  questsep .initial:n = 0.5em,
}

\__edu_keys_initial:n {questpointssep = 0.25em}
\__edu_keys_initial:n {questsep = 0.5em}

\end{MacroCode}
\end{option}
\end{option}

\begin{option}{questbeforeskip}
\begin{option}{questafterskip}
Skips before and after questions:
\begin{MacroCode}{class}
\keys_define:nn {edu} {
  questbeforeskip .dim_gset:N = \g_edu_questbeforeskip_dim,
  questbeforeskip .initial:n = 1ex,
  questafterskip .dim_gset:N = \g_edu_questafterskip_dim,
  questafterskip .initial:n = 0ex,
}

\__edu_keys_initial:n {questbeforeskip = 1ex}
\__edu_keys_initial:n {questafterskip = 0ex}

\end{MacroCode}
\end{option}
\end{option}


\subsubsection{Notizen}


\begin{option}{shownotes}
\begin{option}{notetstyle}
\begin{option}{notehrule}
\begin{option}{notetfg}
\begin{option}{notehrfg}
Skips before and after questions:
\begin{MacroCode}{class}
\keys_define:nn {edu} {
  shownotes .bool_gset:N = \g_edu_shownotes_bool,
  shownotes .initial:n = true,
  notetstyle .tl_gset:N = \g_edu_notetstyle_tl,
  notetstyle .initial:n = \sffamily,
  notehrule .dim_gset:N = \g_edu_notehrule_dim,
  notehrule .initial:n = 0.4pt,
  notetfg .tl_gset:N = \g_edu_notetfg_tl,
  notetfg .initial:n = wuRed,
  notehrfg .tl_gset:N = \g_edu_notehrfg_tl,
  notehrfg .initial:n = wuRed,
}

\end{MacroCode}
\end{option}
\end{option}
\end{option}
\end{option}
\end{option}




\subsubsection{Unterrichtsablauf}

First, options of the timetable:

\begin{option}{ttendtime}
Display ending time of lesson:
\begin{MacroCode}{class}
\keys_define:nn {edu} {
  ttendtime .bool_gset:N = \g_edu_ttendtime_bool,
  ttendtime .initial:n = true,
}

\end{MacroCode}
\end{option}

\begin{option}{tttimelabel}
\begin{option}{ttstagelabel}
\begin{option}{ttactivitylabel}
\begin{option}{ttmethodlabel}
\begin{option}{ttmedialabel}
Column labels:
\begin{MacroCode}{class}
\keys_define:nn {edu} {
  tttimelabel .tl_gset:N = \g_edu_tttimelabel_tl,
  tttimelabel .initial:n = Zeit,
  ttstagelabel .tl_gset:N = \g_edu_ttstagelabel_tl,
  ttstagelabel .initial:n = Phase,
  ttactivitylabel .tl_gset:N = \g_edu_ttactivitylabel_tl,
  ttactivitylabel .initial:n = Ablauf,
  ttmethodlabel .tl_gset:N = \g_edu_ttmethodlabel_tl,
  ttmethodlabel .initial:n = Methoden,
  ttmedialabel .tl_gset:N = \g_edu_ttmedialabel_tl,
  ttmedialabel .initial:n = Medien/Material,
}

\end{MacroCode}
\end{option}
\end{option}
\end{option}
\end{option}
\end{option}


\begin{option}{tttimewidth}
\begin{option}{ttstagewidth}
\begin{option}{ttmethodwidth}
\begin{option}{ttmediawidth}
Column widths (normal):
\begin{MacroCode}{class}
\keys_define:nn {edu} {
  tttimewidth .dim_gset:N = \g_edu_tttimewidth_dim,
  tttimewidth .initial:n = 1cm,
  ttstagewidth .dim_gset:N = \g_edu_ttstagewidth_dim,
  ttstagewidth .initial:n = 2.25cm,
  ttmethodwidth .dim_gset:N = \g_edu_ttmethodwidth_dim,
  ttmethodwidth .initial:n = 2cm,
  ttmediawidth .dim_gset:N = \g_edu_ttmediawidth_dim,
  ttmediawidth .initial:n = 2.75cm,
}

\end{MacroCode}
\end{option}
\end{option}
\end{option}
\end{option}


\begin{option}{tttimewidthlscape}
\begin{option}{ttstagewidthlscape}
\begin{option}{ttmethodwidthlscape}
\begin{option}{ttmediawidthlscape}
Column widths (landscape):
\begin{MacroCode}{class}
\keys_define:nn {edu} {
  tttimewidthlscape .dim_gset:N = \g_edu_tttimewidthlscape_dim,
  tttimewidthlscape .initial:n = 1cm,
  ttstagewidthlscape .dim_gset:N = \g_edu_ttstagewidthlscape_dim,
  ttstagewidthlscape .initial:n = 3.5cm,
  ttmethodwidthlscape .dim_gset:N = \g_edu_ttmethodwidthlscape_dim,
  ttmethodwidthlscape .initial:n = 3.5cm,
  ttmediawidthlscape .dim_gset:N = \g_edu_ttmediawidthlscape_dim,
  ttmediawidthlscape .initial:n = 3.5cm,
}

\end{MacroCode}
\end{option}
\end{option}
\end{option}
\end{option}


\begin{option}{ttshowtime}
\begin{option}{ttentrytimelabel}
Various options:
\begin{MacroCode}{class}
\keys_define:nn {edu} {
  ttshowtime .bool_gset:N = \g_edu_ttshowtime_bool,
  ttshowtime .initial:n = true,
  ttentrytimelabel .tl_gset:N = \g_edu_ttentrytimelabel_tl,
  ttentrytimelabel .initial:n = ,
}

\end{MacroCode}
\end{option}
\end{option}


Options of sequence-lists:

\begin{option}{ttshowtime}
\begin{option}{ttentrytimelabel}
Labels of teachers and students:
\begin{MacroCode}{class}
\keys_define:nn {edu} {
  seqteacherlabel .tl_gset:N = \g_edu_seqteacherlabel_tl,
  seqteacherlabel .initial:n = L,
  seqstudentlabel .tl_gset:N = \g_edu_seqstudentlabel_tl,
  seqstudentlabel .initial:n = S,
}

\end{MacroCode}
\end{option}
\end{option}

\begin{option}{seqteacherstyle}
\begin{option}{seqetudentstyle}
Style of labels:
\begin{MacroCode}{class}
\keys_define:nn {edu} {
  seqteacherstyle .tl_gset:N = \g_edu_seqteacherstyle_tl,
  seqteacherstyle .initial:n = \sffamily\bfseries,
  seqetudentstyle .tl_gset:N = \g_edu_seqetudentstyle_tl,
  seqetudentstyle .initial:n = \sffamily\bfseries,
}

\end{MacroCode}
\end{option}
\end{option}

\begin{option}{seqteacherfg}
\begin{option}{seqstudentfg}
Colors of labels:
\begin{MacroCode}{class}
\keys_define:nn {edu} {
  seqteacherfg .tl_gset:N = \g_edu_seqteacherfg_tl,
  seqteacherfg .initial:n = black,
  seqstudentfg .tl_gset:N = \g_edu_seqstudentfg_tl,
  seqstudentfg .initial:n = black,
}

\end{MacroCode}
\end{option}
\end{option}


\subsubsection{Tafelbild}

\begin{option}{bbstyle}
\begin{option}{bbfontsize}
\begin{option}{bbbaselineoffset}
\begin{option}{bbheight}
General options of blackboard:
\begin{MacroCode}{class}
\keys_define:nn {edu} {
  bbstyle .tl_gset:N = \g_edu_bbstyle_tl,
  bbstyle .initial:n = \sffamily,
  bbfontsize .dim_gset:N = \g_edu_bbfontsize_dim,
  bbfontsize .initial:n = 8pt,
  bbbaselineoffset .dim_gset:N = \g_edu_bbbaselineoffset_dim,
  bbbaselineoffset .initial:n = 1pt,
  bbheight .dim_gset:N = \g_edu_bbheight_dim,
  bbheight .initial:n = 0.25\linewidth - 2\fboxsep,
}

\__edu_keys_initial:n {bbheight = 0.25\linewidth - 2\fboxsep}

\end{MacroCode}
\end{option}
\end{option}
\end{option}
\end{option}

\begin{option}{bbsectionstyle}
\begin{option}{bbsubsectionstyle}
\begin{option}{bbsubsubsectionstyle}
Title style options of blackboard:
\begin{MacroCode}{class}
\keys_define:nn {edu} {
  bbsectionstyle .tl_gset:N = \g_edu_bbsectionstyle_tl,
  bbsectionstyle .initial:n = \normalsize\scshape\color{wuRed},
  bbsubsectionstyle .tl_gset:N = \g_edu_bbsubsectionstyle_tl,
  bbsubsectionstyle .initial:n = \small\scshape,
  bbsubsubsectionstyle .tl_gset:N = \g_edu_bbsubsubsectionstyle_tl,
  bbsubsubsectionstyle .initial:n = \ulined,
}

\end{MacroCode}
\end{option}
\end{option}
\end{option}


\subsubsection{Mathematik}

\begin{option}{commasep}
\begin{option}{amsoptions}
Basic math options:
\begin{MacroCode}{class}
\keys_define:nn {edu} {
  commasep .bool_gset:N = \g_edu_commasep_bool,    % Comma as separator
  commasep .initial:n = true,
  amsoptions .tl_gset:N = \g_edu_amsoptions_tl,  % Pass options to amsmath-package
  amsoptions .initial:n = intlimits,
}

\end{MacroCode}
\end{option}
\end{option}

\begin{option}{amsthm}
\begin{option}{framedthm}
\begin{option}{thmbox}
\begin{option}{thmimpnumbered}
\begin{option}{thmunimpnumbered}
Basic math options:
\begin{MacroCode}{class}
\keys_define:nn {edu} {
  amsthm .bool_gset:N = \g_edu_amsthm_bool,                      % Predefine amsthm theorems
  amsthm .initial:n = true,
  framedthm .bool_gset:N = \g_edu_framedthm_bool,                % Predefine thmbox theorems
  framedthm .initial:n = true,
  thmbox .bool_gset:N = \g_edu_thmbox_bool,                      % Predefine framed theorems
  thmbox .initial:n = false,
  thmimpnumbered .bool_gset:N = \g_edu_thmimpnumbered_bool,      % Number important theorems
  thmimpnumbered .initial:n = false,
  thmunimpnumbered .bool_gset:N = \g_edu_thmunimpnumbered_bool,  % Number unimportant theorems
  thmunimpnumbered .initial:n = false,
}

\end{MacroCode}
\end{option}
\end{option}
\end{option}
\end{option}
\end{option}





\begin{option}{thmlabelfg}
\begin{option}{thmframestyle}
\begin{option}{thmframefg}
\begin{option}{thmframebg}

Basic style options of theorem-like environments:
\begin{MacroCode}{class}
\keys_define:nn {edu} {
  thmlabelfg .tl_gset:N = \g_edu_thmlabelfg_tl,        % Color of theorem labels
  thmlabelfg .initial:n = black,
  thmframefg .tl_gset:N = \g_edu_thmframefg_tl,        % Color of framed theorem frame.
  thmframefg .initial:n = wuDarkerGray,
  thmframebg .tl_gset:N = \g_edu_thmframebg_tl,        % Color of framed theorem background.
  thmframebg .initial:n = wuLightGray,
}

\end{MacroCode}
\end{option}
\end{option}
\end{option}
\end{option}


\begin{option}{thmimplabelstyle}
\begin{option}{thmimpnotestyle}
\begin{option}{thmimpbodystyle}
Style of important theorems:
\begin{MacroCode}{class}
\keys_define:nn {edu} {
  thmimplabelstyle .tl_gset:N = \g_edu_thmimplabelstyle_tl,  % Style of heads
  thmimplabelstyle .initial:n = \sffamily\bfseries,
  thmimpnotestyle .tl_gset:N = \g_edu_thmimpnotestyle_tl,  % Style of notes
  thmimpnotestyle .initial:n = \sffamily\bfseries,
  thmimpbodystyle .tl_gset:N = \g_edu_thmimpbodystyle_tl,  % Style of bodies
  thmimpbodystyle .initial:n =,
}

\end{MacroCode}
\end{option}
\end{option}
\end{option}


\begin{option}{thmunimplabelstyle}
\begin{option}{thmunimpnotestyle}
\begin{option}{thmunimpbodystyle}
Style of unimportant theorems:
\begin{MacroCode}{class}
\keys_define:nn {edu} {
  thmunimplabelstyle .tl_gset:N = \g_edu_thmunimplabelstyle_tl,  % Style of heads
  thmunimplabelstyle .initial:n = \sffamily\bfseries,
  thmunimpnotestyle .tl_gset:N = \g_edu_thmunimpnotestyle_tl,  % Style of notes
  thmunimpnotestyle .initial:n = \sffamily,
  thmunimpbodystyle .tl_gset:N = \g_edu_thmunimpbodystyle_tl,  % Style of bodies
  thmunimpbodystyle .initial:n =,
}

\end{MacroCode}
\end{option}
\end{option}
\end{option}

\begin{option}{thmdefinitionlabel}
\begin{option}{thmdefitheolabel}
\begin{option}{thmexamplelabel}
\begin{option}{thmexampleexelabel}
\begin{option}{thmhintlabel}
\begin{option}{thmremarklabel}
\begin{option}{thmsolutionlabel}
\begin{option}{thmtheoremlabel}
Labels of theorems:
\begin{MacroCode}{class}
\keys_define:nn {edu} {
  thmdefinitionlabel .tl_gset:N = \g_edu_thmdefinitionlabel_tl, 
  thmdefinitionlabel .initial:n = Definition,
  thmdefitheolabel .tl_gset:N = \g_edu_thmdefitheolabel_tl,
  thmdefitheolabel .initial:n = Definition/Satz,
  thmexamplelabel .tl_gset:N = \g_edu_thmexamplelabel_tl,
  thmexamplelabel .initial:n = Beispiel,
  thmexampleexelabel .tl_gset:N = \g_edu_thmexampleexelabel_tl,
  thmexampleexelabel .initial:n = Beispielaufgabe,
  thmhintlabel .tl_gset:N = \g_edu_thmhintlabel_tl,
  thmhintlabel .initial:n = Hinweis,
  thmremarklabel .tl_gset:N = \g_edu_thmremarklabel_tl,
  thmremarklabel .initial:n = Bemerkung,
  thmsolutionlabel .tl_gset:N = \g_edu_thmsolutionlabel_tl,
  thmsolutionlabel .initial:n = L\"osung,
  thmtheoremlabel .tl_gset:N = \g_edu_thmtheoremlabel_tl,
  thmtheoremlabel .initial:n = Satz,
}

\end{MacroCode}
\end{option}
\end{option}
\end{option}
\end{option}
\end{option}
\end{option}
\end{option}
\end{option}

\subsubsection{Informatik}

\begin{option}{lstnumberfg}
\begin{option}{lstkeywordfg}
\begin{option}{lstrulefg}
Options of listings:
\begin{MacroCode}{class}
\keys_define:nn {edu} {
  lstnumberfg .tl_gset:N = \g_edu_lstnumberfg_tl,
  lstnumberfg .initial:n = black,
  lstkeywordfg .tl_gset:N = \g_edu_lstkeywordfg_tl,
  lstkeywordfg .initial:n = black,
  lstrulefg .tl_gset:N = \g_edu_lstrulefg_tl,
  lstrulefg .initial:n = gray,
}

\end{MacroCode}
\end{option}
\end{option}
\end{option}


\subsubsection{Kompatibilität}

Some important packages are incompatible. These options give the opportunity to choose between them.

\begin{option}{lstnumberfg}
\begin{option}{lstkeywordfg}
\begin{option}{lstrulefg}
Options of listings:
\begin{MacroCode}{class}
\tl_new:N \g_edu_struktexqtree_tl

\keys_define:nn {edu} {
  struktexqtree .choice_code:n = {
    \tl_gset:NV \g_edu_struktexqtree_tl \l_keys_choice_tl
  },
  struktexqtree .generate_choices:n = {  % option for choosing package
    struktex,
    tikz-qtree
  },
  struktexqtree .initial:n = struktex
}

\end{MacroCode}
\end{option}
\end{option}
\end{option}


\subsubsection{Undefinierte Optionen}

\begin{MacroCode}{class}
\DeclareOption*{%
  \PassOptionsToClass{\CurrentOption}{scrartcl}%
}

\end{MacroCode}


\subsubsection{Optionen verarbeiten}

\begin{MacroCode}{class}
\ProcessOptions              % LaTeX-Basics (for \PassOptionsToClass)
\ProcessKeysOptions{edu}  % l3keys2e options

\end{MacroCode}


\subsection{Basisklasse und Packages laden}

Informationen der einzelnen Packages sind der jeweiligen Dokumentation zu entnehmen.

\subsubsection{Basisklasse}

\begin{MacroCode}{class}
\LoadClass{scrartcl}

\end{MacroCode}


\subsubsection{Grundlegende Packages}

\begin{MacroCode}{class}
\RequirePackage[T1]{fontenc} 
\RequirePackage[utf8]{inputenc} 
\RequirePackage[ngerman]{babel}

\RequirePackage{calc}
\RequirePackage{etoolbox}

\end{MacroCode}


\subsubsection{Layout/Typographie}

\begin{MacroCode}{class}
\RequirePackage{booktabs}
\RequirePackage{ccicons}
\RequirePackage{enumitem}
\RequirePackage{eurosym}                                           % Euro-Sign
\RequirePackage{expdlist}
\RequirePackage{geometry}
\RequirePackage[notcomma, notperiod, notquote, notquery]{hanging}  % Hanging indention of multiexe
\RequirePackage{lastpage}                                          % Number of pages with 
\RequirePackage{multicol}
\RequirePackage{multirow}
\RequirePackage{pdflscape}
\RequirePackage{pifont}
\RequirePackage[newcommands]{ragged2e}
\RequirePackage{rotating}
\RequirePackage{scrpage2}
\RequirePackage{setspace}
\RequirePackage{siunitx}
\RequirePackage{tabularx}
\RequirePackage[nobottomtitles*]{titlesec}
\RequirePackage[normalem]{ulem}
\RequirePackage{xspace}

\end{MacroCode}


\subsubsection{Graphik}


\begin{MacroCode}{class}
\RequirePackage{graphicx}
\RequirePackage[svgnames, table]{xcolor}
\RequirePackage{subfig}
\RequirePackage{tikz}

\tl_if_eq:VnT \g_edu_struktexqtree_tl {tikz-qtree} {
  \ExplSyntaxOff
  \RequirePackage{tikz-qtree}
  \ExplSyntaxOn
}

\RequirePackage{tikzsymbols}

\end{MacroCode}


\subsubsection{Style- und Color-Themes}

\begin{MacroCode}{class}
\AtEndPreamble{
	\tl_if_empty:NF \g_edu_colortheme_tl {
	  \RequirePackage{edu-colors-\g_edu_colortheme_tl}
	}
	
	\tl_if_empty:NF \g_edu_styletheme_tl {
	  \RequirePackage{edu-styles-\g_edu_styletheme_tl}
	}
}
  
  

\end{MacroCode}


\subsubsection{Mathematik}

\begin{MacroCode}{class}
\RequirePackage[\g_edu_amsoptions_tl]{amsmath}
\RequirePackage{amssymb}
\RequirePackage{amsthm}
\RequirePackage{cancel}
\RequirePackage{esvect}
\RequirePackage{gauss}
\RequirePackage{polynom}
\RequirePackage{thmtools}
\RequirePackage[only, lightning]{stmaryrd}
\RequirePackage{xlop}


\end{MacroCode}


\subsubsection{Computer Science}

\begin{MacroCode}{class}
\RequirePackage{listings}

% menukeys needs to be loaded at the end. Otherwise there occurs an 
% option clash with newtxmath/amsmath, which could not be fixed.
\AtEndPreamble{
  \RequirePackage[os=win]{menukeys}
  \renewmenumacro{\menu}[>]{angularmenus}
  \renewmenumacro{\keys}{angularkeys}
}

\tl_if_eq:VnT \g_edu_struktexqtree_tl {struktex} {
  \RequirePackage{struktex}
}

\end{MacroCode}


\subsubsection{Sonstige}

\begin{MacroCode}{class}
\RequirePackage{bibgerm}
\RequirePackage{hyperref}

\end{MacroCode}


\subsection{Schriften}

Soll eine Overhead-Folie gesetzt werden (\opt{sfdefault}), wird serifenlose Schrift verwendet:

\begin{MacroCode}{class}
\bool_if:NT \g_edu_transparency_bool {
  \keys_set:nn {edu} {sfdefault = true}
}

\end{MacroCode}

\subsubsection{Schriftarten}

Über die Optionen \opt{palatino}, \opt{lato} und \opt{beramono} können diese als Standardschrift für Serifenschrift, serifenlose Schrift bzw. nichtproportionale Schrift gewählt werden:

\begin{MacroCode}{class}
\bool_if:NF \g_edu_sfdefault_bool {            % Serif fonts only if not sfdefault
	\tl_if_eq:VnT \g_edu_rmfont_tl {libertine} {
	  \RequirePackage[rm]{libertine}
	  \RequirePackage[libertine, varbb]{newtxmath}
	}
	
	\tl_if_eq:VnT \g_edu_rmfont_tl {palatino} {
	  \RequirePackage{mathpazo}
	}
}

\tl_if_eq:VnT \g_edu_sffont_tl {lato} {
  \RequirePackage[defaultsans]{lato}
}

\tl_if_eq:VnT \g_edu_sffont_tl {sourcesanspro} {
  \RequirePackage[scale=0.96]{sourcesanspro}
}

\tl_if_eq:VnT \g_edu_sffont_tl {roboto} {
  \RequirePackage[scaled=0.9]{roboto}    % Add scaled option after existing bug in roboto is fixed
}

\tl_if_eq:VnT \g_edu_ttfont_tl {beramono} {
  \RequirePackage[scaled=0.85]{beramono}
}

\tl_if_eq:VnT \g_edu_ttfont_tl {sourcecodepro} {
  \RequirePackage[scale=0.875]{sourcecodepro}
}

\end{MacroCode}

Mithilfe von \opt{sfdefault} kann \cs{sfdefault} als Standardschrift verwendet werden. Die Symbole für Formeln werden dann einzeln angegeben, da das Einbinden von  \textsf{cmbright} zu einem Fehler führt, wenn man innerhalb des Dokuments die Schriftgröße ändert, was bei \opt{transparency} der Fall ist (\Macro\KOMAoption{fontsize}{22pt}). Es sollten alle Mathesymbole (inkl. griechischer Buchstaben) serifenlos gesetzt werden. Nur große Operatoren (Summe, Integral) sind nicht serifenlos vorhanden.

\begin{MacroCode}{class}
\bool_if:NT \g_edu_sfdefault_bool {
  \renewcommand{\familydefault}{\sfdefault}
  \keys_set:nn {edu} {sfmath = true}
}

\bool_if:NT \g_edu_sfmath_bool {

  \RequirePackage[]{sansmath}                   % Default sans serif font as math font
  \sansmath
  
  \usepackage[bb=pazo, bbscaled=0.9]{mathalfa}  % Load blackboard font of palation
 
  % Load small and large greek letters of Iwona as there are
  % not many other sans serif greek letters
	\DeclareSymbolFont{Greekletters}{OT1}{iwona}{m}{n}
	\DeclareSymbolFont{greekletters}{OML}{iwona}{m}{it}
	
	\DeclareMathSymbol{\Delta}{\mathord}{Greekletters}{"01}
	\DeclareMathSymbol{\Theta}{\mathord}{Greekletters}{"02}
	\DeclareMathSymbol{\Lambda}{\mathord}{Greekletters}{"03}
	\DeclareMathSymbol{\Xi}{\mathord}{Greekletters}{"04}
	\DeclareMathSymbol{\Pi}{\mathord}{Greekletters}{"05}
	\DeclareMathSymbol{\Sigma}{\mathord}{Greekletters}{"06}
	\DeclareMathSymbol{\Upsilon}{\mathord}{Greekletters}{"07}
	\DeclareMathSymbol{\Phi}{\mathord}{Greekletters}{"08}
	\DeclareMathSymbol{\Psi}{\mathord}{Greekletters}{"09}
	\DeclareMathSymbol{\Omega}{\mathord}{Greekletters}{"0A}
	
	\DeclareMathSymbol{\alpha}{\mathord}{greekletters}{"0B}
	\DeclareMathSymbol{\beta}{\mathord}{greekletters}{"0C}
	\DeclareMathSymbol{\gamma}{\mathord}{greekletters}{"0D}
	\DeclareMathSymbol{\delta}{\mathord}{greekletters}{"0E}
	\DeclareMathSymbol{\epsilon}{\mathord}{greekletters}{"0F}
	\DeclareMathSymbol{\zeta}{\mathord}{greekletters}{"10}
	\DeclareMathSymbol{\eta}{\mathord}{greekletters}{"11}
	\DeclareMathSymbol{\theta}{\mathord}{greekletters}{"12}
	\DeclareMathSymbol{\iota}{\mathord}{greekletters}{"13}
	\DeclareMathSymbol{\kappa}{\mathord}{greekletters}{"14}
	\DeclareMathSymbol{\lambda}{\mathord}{greekletters}{"15}
	\DeclareMathSymbol{\mu}{\mathord}{greekletters}{"16}
	\DeclareMathSymbol{\nu}{\mathord}{greekletters}{"17}
	\DeclareMathSymbol{\xi}{\mathord}{greekletters}{"18}
	\DeclareMathSymbol{\pi}{\mathord}{greekletters}{"19}
	\DeclareMathSymbol{\rho}{\mathord}{greekletters}{"1A}
	\DeclareMathSymbol{\sigma}{\mathord}{greekletters}{"1B}
	\DeclareMathSymbol{\tau}{\mathord}{greekletters}{"1C}
	\DeclareMathSymbol{\upsilon}{\mathord}{greekletters}{"1D}
	\DeclareMathSymbol{\phi}{\mathord}{greekletters}{"1E}
	\DeclareMathSymbol{\chi}{\mathord}{greekletters}{"1F}
	\DeclareMathSymbol{\psi}{\mathord}{greekletters}{"20}
	\DeclareMathSymbol{\omega}{\mathord}{greekletters}{"21}
	\DeclareMathSymbol{\varepsilon}{\mathord}{greekletters}{"22}
	\DeclareMathSymbol{\vartheta}{\mathord}{greekletters}{"23}
	\DeclareMathSymbol{\varpi}{\mathord}{greekletters}{"24}
	\DeclareMathSymbol{\varrho}{\mathord}{greekletters}{"25}
	\DeclareMathSymbol{\varsigma}{\mathord}{greekletters}{"26}
	\DeclareMathSymbol{\varphi}{\mathord}{greekletters}{"27}
}

\end{MacroCode}


\subsubsection{Schriftgröße}

Bei Verwendung der Option \opt{twuop} wird die Schriftgröße des gesamten Dokuments angepasst. Die Option \opt{transparency} wird gesondert verarbeitet.

\begin{MacroCode}{class}

\AtBeginDocument{
  \bool_if:NTF \g_edu_twoup_bool {
    \KOMAoptions{fontsize=\dim_use:N \g_edu_twoupfontsize_dim}
  }{
	  \KOMAoptions{fontsize=\dim_use:N \g_edu_fontsize_dim}
  }
}

\end{MacroCode}



\subsection{Metadaten}

\begin{macro*}{\g_edu_author_tl}
\begin{macro*}{\g_edu_class_tl}
\begin{macro*}{\g_edu_date_tl}
\begin{macro*}{\g_edu_email_tl}
\begin{macro*}{\g_edu_field_tl}
\begin{macro*}{\g_edu_group_tl}
\begin{macro*}{\g_edu_license_tl}
\begin{macro*}{\g_edu_subject_tl}
\begin{macro*}{\g_edu_subtitle_tl}
\begin{macro*}{\g_edu_title_tl}
\begin{macro*}{\g_edu_version_tl}
\begin{MacroCode}{class}

\tl_new:N \g_edu_author_tl
\tl_new:N \g_edu_class_tl
\tl_new:N \g_edu_date_tl
\tl_new:N \g_edu_email_tl
\tl_new:N \g_edu_field_tl
\tl_new:N \g_edu_group_tl
\tl_new:N \g_edu_license_tl
\tl_new:N \g_edu_subject_tl
\tl_new:N \g_edu_subtitle_tl
\tl_new:N \g_edu_title_tl
\tl_new:N \g_edu_version_tl

\end{MacroCode}
\end{macro*}
\end{macro*}
\end{macro*}
\end{macro*}
\end{macro*}
\end{macro*}
\end{macro*}
\end{macro*}
\end{macro*}
\end{macro*}
\end{macro*}

\begin{macro*}{\g_edu_authorshort_tl}
\begin{macro*}{\g_edu_classshort_tl}
\begin{macro*}{\g_edu_dateshort_tl}
\begin{macro*}{\g_edu_emailshort_tl}
\begin{macro*}{\g_edu_fieldshort_tl}
\begin{macro*}{\g_edu_groupshort_tl}
\begin{macro*}{\g_edu_licenseshort_tl}
\begin{macro*}{\g_edu_subjectshort_tl}
\begin{macro*}{\g_edu_subtitleshort_tl}
\begin{macro*}{\g_edu_titleshort_tl}
\begin{macro*}{\g_edu_versionshort_tl}
\begin{MacroCode}{class}

\tl_new:N \g_edu_authorshort_tl
\tl_new:N \g_edu_classshort_tl
\tl_new:N \g_edu_dateshort_tl
\tl_new:N \g_edu_emailshort_tl
\tl_new:N \g_edu_fieldshort_tl
\tl_new:N \g_edu_groupshort_tl
\tl_new:N \g_edu_licenseshort_tl
\tl_new:N \g_edu_subjectshort_tl
\tl_new:N \g_edu_subtitleshort_tl
\tl_new:N \g_edu_titleshort_tl
\tl_new:N \g_edu_versionshort_tl

\end{MacroCode}
\end{macro*}
\end{macro*}
\end{macro*}
\end{macro*}
\end{macro*}
\end{macro*}
\end{macro*}
\end{macro*}
\end{macro*}
\end{macro*}
\end{macro*}


\begin{macro}{\author}[2]{[<author short>]}{<author>}
\begin{macro}{\class}[2]{[<class short>]}{<class>}
\begin{macro}{\date}[2]{[<date short>]}{<date>}
\begin{macro}{\email}[2]{[<email short>]}{<email>}
\begin{macro}{\field}[2]{[<field short>]}{<field>}
\begin{macro}{\group}[2]{[<group short>]}{<group>}
\begin{macro}{\license}[2]{[<license short>]}{<license>}
\begin{macro}{\subject}[2]{[<subject short>]}{<subject>}
\begin{macro}{\subtitle}[2]{[<subtitle short>]}{<subtitle>}
\begin{macro}{\title}[2]{[<title short>]}{<title>}
\begin{macro}{\version}[2]{[<version short>]}{<version>}
\begin{MacroCode}{class}
\DeclareDocumentCommand \author {o m} {
  \tl_gset:Nn \g_edu_author_tl {#2}
  \IfNoValueTF{#1} {
    \tl_gset:Nn \g_edu_authorshort_tl {#2}
  }{
    \tl_gset:Nn \g_edu_authorshort_tl {#1}
  }
}

\DeclareDocumentCommand \class {o m} {
  \tl_gset:Nn \g_edu_class_tl {#2}
  \IfNoValueTF{#1} {
    \tl_gset:Nn \g_edu_classshort_tl {#2}
  }{
    \tl_gset:Nn \g_edu_classshort_tl {#1}
  }
}

\DeclareDocumentCommand \date {o m} {
  \tl_gset:Nn \g_edu_date_tl {#2}
  \IfNoValueTF{#1} {
    \tl_gset:Nn \g_edu_dateshort_tl {#2}
  }{
    \tl_gset:Nn \g_edu_dateshort_tl {#1}
  }
}

\DeclareDocumentCommand \email {o m} {
  \tl_gset:Nn \g_edu_email_tl {#2}
  \IfNoValueTF{#1} {
    \tl_gset:Nn \g_edu_emailshort_tl {#2}
  }{
    \tl_gset:Nn \g_edu_emailshort_tl {#1}
  }
}

\DeclareDocumentCommand \field {o m} {
  \tl_gset:Nn \g_edu_field_tl {#2}
  \IfNoValueTF{#1} {
    \tl_gset:Nn \g_edu_fieldshort_tl {#2}
  }{
    \tl_gset:Nn \g_edu_fieldshort_tl {#1}
  }
}

\DeclareDocumentCommand \group {o m} {
  \tl_gset:Nn \g_edu_group_tl {#2}
  \IfNoValueTF{#1} {
    \tl_gset:Nn \g_edu_groupshort_tl {#2}
  }{
    \tl_gset:Nn \g_edu_groupshort_tl {#1}
  }
}

\DeclareDocumentCommand \license {o m} {
  \tl_gset:Nn \g_edu_license_tl {#2}
  \IfNoValueTF{#1} {
    \tl_gset:Nn \g_edu_licenseshort_tl {#2}
  }{
    \tl_gset:Nn \g_edu_licenseshort_tl {#1}
  }
}

\DeclareDocumentCommand \subject {o m} {
  \tl_gset:Nn \g_edu_subject_tl {#2}
  \IfNoValueTF{#1} {
    \tl_gset:Nn \g_edu_subjectshort_tl {#2}
  }{
    \tl_gset:Nn \g_edu_subjectshort_tl {#1}
  }
}

\DeclareDocumentCommand \subtitle {o m} {
  \tl_gset:Nn \g_edu_subtitle_tl {#2}
  \IfNoValueTF{#1} {
    \tl_gset:Nn \g_edu_subtitleshort_tl {#2}
  }{
    \tl_gset:Nn \g_edu_subtitleshort_tl {#1}
  }
}

\DeclareDocumentCommand \title {o m} {
  \tl_gset:Nn \g_edu_title_tl {#2}
  \IfNoValueTF{#1} {
    \tl_gset:Nn \g_edu_titleshort_tl {#2}
  }{
    \tl_gset:Nn \g_edu_titleshort_tl {#1}
  }
}

\DeclareDocumentCommand \version {o m} {
  \tl_gset:Nn \g_edu_version_tl {#2}
  \IfNoValueTF{#1} {
    \tl_gset:Nn \g_edu_versionshort_tl {#2}
  }{
    \tl_gset:Nn \g_edu_versionshort_tl {#1}
  }
}

\end{MacroCode}
\end{macro}
\end{macro}
\end{macro}
\end{macro}
\end{macro}
\end{macro}
\end{macro}
\end{macro}
\end{macro}
\end{macro}
\end{macro}


\begin{macro}{\printauthor}
\begin{macro}{\printauthor*}
\begin{macro}{\printclass}
\begin{macro}{\printclass*}
\begin{macro}{\printdate}
\begin{macro}{\printdate*}
\begin{macro}{\printemail}
\begin{macro}{\printemail*}
\begin{macro}{\printfield}
\begin{macro}{\printfield*}
\begin{macro}{\printgroup}
\begin{macro}{\printgroup*}
\begin{macro}{\printlicense}
\begin{macro}{\printlicense*}
\begin{macro}{\printsubject}
\begin{macro}{\printsubject*}
\begin{macro}{\printsubtitle}
\begin{macro}{\printsubtitle*}
\begin{macro}{\printtitle}
\begin{macro}{\printtitle*}
\begin{macro}{\printversion}
\begin{macro}{\printversion*}
\begin{MacroCode}{class}
\DeclareDocumentCommand \printauthor { s } {
  \IfBooleanTF{#1} {
    \g_edu_authorshort_tl\xspace
  }{
    \g_edu_author_tl\xspace
  }
}

\DeclareDocumentCommand \printclass { s } {
  \IfBooleanTF{#1} {
    \g_edu_classshort_tl\xspace
  }{
    \g_edu_class_tl\xspace
  }
}

\DeclareDocumentCommand \printdate { s } {
  \IfBooleanTF{#1} {
    \g_edu_dateshort_tl\xspace
  }{
    \g_edu_date_tl\xspace
  }
}

\DeclareDocumentCommand \printemail { s } {
  \IfBooleanTF{#1} {
    \g_edu_emailshort_tl\xspace
  }{
    \g_edu_email_tl\xspace
  }
}

\DeclareDocumentCommand \printfield { s } {
  \IfBooleanTF{#1} {
    \g_edu_fieldshort_tl\xspace
  }{
    \g_edu_field_tl\xspace
  }
}

\DeclareDocumentCommand \printgroup { s } {
  \IfBooleanTF{#1} {
    \g_edu_groupshort_tl\xspace
  }{
    \g_edu_group_tl\xspace
  }
}

\DeclareDocumentCommand \printlicense { s } {
  \IfBooleanTF{#1} {
    \g_edu_licenseshort_tl\xspace
  }{
    \g_edu_license_tl\xspace
  }
}

\DeclareDocumentCommand \printsubject { s } {
  \IfBooleanTF{#1} {
    \g_edu_subjectshort_tl\xspace
  }{
    \g_edu_subject_tl\xspace
  }
}

\DeclareDocumentCommand \printsubtitle { s } {
  \IfBooleanTF{#1} {
    \g_edu_subtitleshort_tl\xspace
  }{
    \g_edu_subtitle_tl\xspace
  }
}

\DeclareDocumentCommand \printtitle { s } {
  \IfBooleanTF{#1} {
    \g_edu_titleshort_tl\xspace
  }{
    \g_edu_title_tl\xspace
  }
}

\DeclareDocumentCommand \printversion { s } {
  \IfBooleanTF{#1} {
    \g_edu_versionshort_tl\xspace
  }{
    \g_edu_version_tl\xspace
  }
}

\end{MacroCode}
\end{macro}
\end{macro}
\end{macro}
\end{macro}
\end{macro}
\end{macro}
\end{macro}
\end{macro}
\end{macro}
\end{macro}
\end{macro}
\end{macro}
\end{macro}
\end{macro}
\end{macro}
\end{macro}
\end{macro}
\end{macro}
\end{macro}
\end{macro}
\end{macro}
\end{macro}


\subsection{Hyperref}


\begin{macro*}{\g__edu_pdftitletemp_tl}
\begin{macro*}{\g__edu_pdfsubjecttemp_tl}
First, document title and subject are created.
\begin{MacroCode}{class}
\tl_new:N \g__edu_pdftitletemp_tl
\tl_gset:Nn \g__edu_pdftitletemp_tl {\g_edu_title_tl}

\tl_new:N \g__edu_pdfsubjecttemp_tl

\AtEndPreamble{
	
	\tl_if_empty:NF \g_edu_subtitle_tl {
	  \tl_gput_right:Nn \g__edu_pdftitletemp_tl {
	    \ --~\g_edu_subtitle_tl
	  }
	}
	
	\tl_if_empty:NF \g_edu_subject_tl {
	  \tl_gput_right:Nn \g__edu_pdfsubjecttemp_tl {
	    \g_edu_subject_tl
	  }
	}
	
	\bool_if:nT {!\tl_if_empty_p:N \g_edu_subject_tl && !\tl_if_empty_p:N \g_edu_field_tl} {
	  \tl_gput_right:Nn \g__edu_pdfsubjecttemp_tl {
	    :~
	  }
	}
	
	\tl_if_empty:NF \g_edu_field_tl {
	  \tl_gput_right:Nn \g__edu_pdfsubjecttemp_tl {
	    \g_edu_field_tl
	  }
	}
	
	\tl_if_empty:NF \g_edu_class_tl {
	  \tl_gput_right:Nn \g__edu_pdfsubjecttemp_tl {
	    \ (\g_edu_class_tl)
	  }
	}

\end{MacroCode}
\end{macro*}
\end{macro*}

Then, \pkg{hyperref} is configured:

\begin{MacroCode}{class}
  \hypersetup{
    breaklinks=false,
    linkcolor=\g_edu_linkfg_tl,
    citecolor=\g_edu_linkfg_tl,
    filecolor=\g_edu_linkfg_tl,
    urlcolor=\g_edu_linkfg_tl,
    linkbordercolor=\tl_use:N \g_edu_linkborderfg_tl,
    citebordercolor=\tl_use:N \g_edu_linkborderfg_tl,
    filebordercolor=\tl_use:N \g_edu_linkborderfg_tl,
    urlbordercolor=\tl_use:N \g_edu_linkborderfg_tl,
    pdftitle={\g__edu_pdftitletemp_tl},
    pdfauthor={\g_edu_author_tl},
    pdfsubject={\g__edu_pdfsubjecttemp_tl},
    linktocpage=true,
  }%
  
  \bool_if:NTF \g_edu_colorlinks_bool {
    \hypersetup{colorlinks=true}
  }{
    \hypersetup{colorlinks=false}
  }
} % AtEndPreamble

\end{MacroCode}


\subsection{Layout}

\subsubsection{Allgemeine Einstellungen}

\begin{MacroCode}{class}
\setlength{\columnsep}{0.75cm}

\end{MacroCode}

\subsubsection{Listen}

\begin{macro*}{\g__edu_listsep_dim}
\begin{macro*}{\g__edu_listmargin_dim}
\begin{macro*}{\g__edu_listarraysep_dim}
\begin{macro*}{\g__edu_listarraymargin_dim}
\begin{macro*}{\g__edu_listlabelwidth_dim}
\begin{macro*}{\g__edu_hangindention_dim}
Define new lengths:
\begin{MacroCode}{class}

\dim_new:N \g__edu_listsep_dim
\dim_new:N \g__edu_listmargin_dim
\dim_new:N \g__edu_listarraysep_dim
\dim_new:N \g__edu_listarraymargin_dim

\dim_new:N \g__edu_listlabelwidth_dim

\dim_new:N \g__edu_hangindention_dim

\end{MacroCode}
\end{macro*}
\end{macro*}
\end{macro*}
\end{macro*}
\end{macro*}
\end{macro*}


Configure list-settings:
\begin{MacroCode}{class}
\AtBeginDocument{
  \dim_gset:Nn \g__edu_listsep_dim {\g_edu_listarraysep_dim}
  \dim_gset:Nn \g__edu_listmargin_dim {\g_edu_listarraymargin_dim}
  \dim_gset:Nn \g__edu_listlabelwidth_dim {1.2em}
}

\setlist{%
  partopsep=0ex,
  topsep=0.5\baselineskip - \parskip,
  itemsep=0.5\baselineskip - \parskip,
  parsep=\parskip,
  labelsep=\g__edu_listsep_dim,
  leftmargin=\g__edu_listlabelwidth_dim + \g__edu_listsep_dim + \g__edu_listmargin_dim
}

\setlist[2]{topsep=0.5\baselineskip - \parskip}
\setlist[3]{topsep=0.5\baselineskip - \parskip}
\setlist[4]{topsep=0.5\baselineskip - \parskip}
\setlist[5]{topsep=0.5\baselineskip - \parskip}

\setlist[itemize, 1]{label=\color{\g_edu_itemizefg_tl}\rule[0.3ex]{0.8ex}{0.8ex}}
\setlist[itemize, 2]{label=\color{\g_edu_itemizefg_tl}\rule[0.5ex]{0.8ex}{0.4ex}}
\setlist[itemize, 3]{label=\color{\g_edu_itemizefg_tl}\rule[0.55ex]{0.8ex}{0.2ex}}

\setlist[enumerate, 1]{label=\color{\g_edu_enumeratefg_tl}\textsf{\arabic*.}}
\setlist[enumerate, 2]{label=\color{\g_edu_enumeratefg_tl}\textsf{\alph*.)}}
\setlist[enumerate, 3]{label=\color{\g_edu_enumeratefg_tl}\textsf{\roman*.}}

\setlist[description]{labelsep=0.25em, font=\color{\g_edu_descriptionfg_tl}}

\end{MacroCode}


\begin{environment}{edulist}[2]{[<itemsep>]}{<item>}
Liste, welche in Teilaufgaben, Lösungen, Fragen und Antworten verwendet wird.
\begin{MacroCode}{class}
\newenvironment{edulist}[2][0.5\baselineskip]{
  \dim_gset:Nn \g__edu_listarraysep_dim {\g_edu_listarraysep_dim}
  \dim_gset:Nn \g__edu_listarraymargin_dim {\g_edu_listarraymargin_dim}
%  
  \vspace{0.5\baselineskip}
  \begin{list}{\sffamily #2}{%
    \setlength{\partopsep}{0ex}%
    \setlength{\topsep}{0pt}%
    \setlength{\parsep}{\parskip}%
    \setlength{\itemsep}{#1 - \parskip}%
    \setlength{\labelsep}{\g__edu_listarraysep_dim}%
    \setlength{\leftmargin}{%
      \g__edu_listlabelwidth_dim + \g__edu_listarraysep_dim + \g__edu_listarraymargin_dim%
    }%
  }
}{
  \end{list}%
  \vspace{0.5\baselineskip}
}

\end{MacroCode}
\end{environment}

\begin{environment}{itemizet}
Liste ohne oberen und unteren Abstand. Geeignet für Aufzählungen in Tabellen. \begin{MacroCode}{class}
\newenvironment{itemizet}{%
  \@minipagetrue
%  \compress
  \begin{itemize}[nosep, leftmargin=1em]
}{%
  \vspace{-\baselineskip}
  \end{itemize}
}

%  \makeatletter
%\newcommand{\compress}{\@minipagetrue}
%  \makeatother

%    \end{macrocode}

\end{MacroCode}
\end{environment}


\subsubsection{Tabellen}

Spaltentyp für zentrierte Spalten mit fester Breite:
\begin{MacroCode}{class}
\newcolumntype{C}[1]{>{\centering\arraybackslash\hspace{0pt}}p{#1}}

\end{MacroCode}

Spaltentyp für \env{multiexearray}.
\begin{MacroCode}{class}
\newcolumntype{e}{%
  >{%
    {\sffamily\multiexelabelboxed}%
      \hspace{\g__edu_listarraysep_dim}\hangpara{\g__edu_hangindention_dim}{1}\raggedright\arraybackslash%
     }%
     X%
}%

\end{MacroCode}

Spaltentyp für \env{multiexearray*}.
\begin{MacroCode}{class}
\newcolumntype{E}{%
  >{%
    {\sffamily\multiexelabelboxed}%
      \hspace{\g__edu_listarraysep_dim}$\displaystyle%
     }%
     X%
    <{$}%
}%

\end{MacroCode}

Spaltentyp für \env{questmcarray}.
\begin{MacroCode}{class}
\newcolumntype{q}{%
  >{%
    {\sffamily%
      \makebox[\g__edu_listlabelwidth_dim][r]{%
        \color{\g_edu_questmclabelfg_tl}%
        $\square$%
      }%
    }%
    \hspace{\g__edu_listarraysep_dim}%
    \hangpara{\g__edu_hangindention_dim}{1}\raggedright\arraybackslash%
   }%
   X%
}%

\end{MacroCode}

Spaltentyp für \env{questmcarray*}.
\begin{MacroCode}{class}
\newcolumntype{Q}{%
  >{%
    {\sffamily%
      \makebox[\g__edu_listlabelwidth_dim][r]{%
        \color{\g_edu_questmclabelfg_tl}%
        $\square$%
      }%
    }%
    \hspace{\g__edu_listarraysep_dim}$\displaystyle%
  }%
  X%
  <{$}%
}%

\end{MacroCode}

Spaltentyp für \env{questmcarrayalph}.
\begin{MacroCode}{class}
\newcolumntype{a}{%
  >{%
    {\sffamily\g__edu_questmclabelboxed_tl}%
      \hspace{\g__edu_listarraysep_dim}\hangpara{\g__edu_hangindention_dim}{1}\raggedright\arraybackslash%
     }%
     X%
}%

\end{MacroCode}

Spaltentyp für \env{questmcarrayalph*}.
\begin{MacroCode}{class}
\newcolumntype{A}{%
  >{%
    {\sffamily\g__edu_questmclabelboxed_tl}%
      \hspace{\g__edu_listarraysep_dim}$\displaystyle%
     }%
     X%
    <{$}%
}%

\end{MacroCode}


\begin{macro}{\arraysetup}[2]{[<extrarowheight>]}{<m>}
Setup der wichtigsten Parameter und Erzeugung des Abstands vor Tabellen bei \env{multiexearray}, \env{multiexearray*}, \env{questmcarray}, \env{questmcarray*}, \env{questmcarrayalph} und \env{questmcarrayalph*}.
\begin{MacroCode}{class}

\DeclareDocumentCommand \arraysetup { O{0ex} m } {%
  \par
  \parskipreduce
  
  \dim_gset:Nn \g__edu_listarraysep_dim {\g_edu_listarraysep_dim}
  \dim_gset:Nn \g__edu_listarraymargin_dim {\g_edu_listarraymargin_dim}
  
  \setlength{\extrarowheight}{#1}
%  \renewcommand{\arraystretch}{1.25} % War aktiviert. Warum?!
  \setlength{\tabcolsep}{\g__edu_listarraymargin_dim}
  \dim_gset:Nn \g__edu_hangindention_dim {
    \g__edu_listlabelwidth_dim+\g__edu_listarraysep_dim
  }
%  
  \str_if_eq:nnTF {#2} {m} {%
      \renewcommand{\tabularxcolumn}[1]{m{##1}}
  }{
      \renewcommand{\tabularxcolumn}[1]{p{##1}}
  }%
%  
  \vspace{\g_edu_arraybeforeskip_dim}
  \vspace{-#1}
}

\end{MacroCode}
\end{macro}


\begin{macro}{\arraycleanup}
Deaktivieren der wichtigsten Parameter und Erzeugung des Abstands nach Tabellen bei \env{multiexearray}, \env{multiexearray*}, \env{questmcarray}, \env{questmcarray*}, \env{questmcarrayalph} und \env{questmcarrayalph*}.
\begin{MacroCode}{class}
\DeclareDocumentCommand \arraycleanup { } {%
  \vspace{\g_edu_arrayafterskip_dim}
  \parskipreduce
  \par
}

\end{MacroCode}
\end{macro}


\subsubsection{Sonstige Kommandos/Umgebungen}

\begin{macro}{\fpbox}[2]{[<width>]}{<content>}
Umrahmte Box mit variabler Breite (standardmäßig \cs{linewidth}).
\begin{MacroCode}{class}
\DeclareDocumentCommand \fpbox { O{\linewidth} m } {
  \noindent\fbox{\parbox{#1}{#2}}
}

\end{MacroCode}
\end{macro}

\begin{macro}{\__edu_moveright:n}[1]{<token list>}
\begin{environment}{citequote}[3]{[<author>]}{[<key>]}{[<reference>]}
Umgebung für Zitate, bei der optional ein Author, eine Quelle und dann ebenfalls optional eine genauere Referenz (z.\,B Seite) angegeben werden kann.
\begin{MacroCode}{class}
\tl_new:N \l__edu_citetext_tl

% Adapted from TeXbook
\cs_new:Npn \__edu_moveright:n #1 {
  \group_begin:
    \leavevmode\unskip\nobreak\hfil\penalty50\hskip2em
  \hbox{}\nobreak\hfil #1%
  \parfillskip=0pt \finalhyphendemerits=0 \endgraf
  \group_end:
}

\DeclareDocumentEnvironment {citequote} { O{} O{} O{} } {
  \par
  \vspace{0.5\baselineskip}
  \parskipreduce
  \begin{addmargin}{3em}
    \begin{em}
}{
    \bool_if:nT {!\tl_if_empty_p:n {#1} || !\tl_if_empty_p:n {#2} || !\tl_if_empty_p:n {#3}} {
      \tl_clear:N \l__edu_citetext_tl
      \upshape
	    \tl_if_empty:nF {#1} {
	      \tl_put_right:Nn \l__edu_citetext_tl {\textsc{#1}\xspace}
	    }
      \bool_if:nT {!\tl_if_empty_p:n {#2} || !\tl_if_empty_p:n {#3}} {
		    \tl_if_empty:nTF {#3} {
		      \tl_put_right:Nn \l__edu_citetext_tl {\cite{#2}}
		    }{
		      \tl_put_right:Nn \l__edu_citetext_tl {\cite[#3]{#2}}
		    }
	    }
	    \__edu_moveright:n \l__edu_citetext_tl
    }
    \end{em}      
  \end{addmargin}
  \par
  \vspace{0.5\baselineskip}
  \parskipreduce
}

\end{MacroCode}
\end{environment}
\end{macro}



\subsubsection{Seitenränder}

\begin{macro}{\setgeometry}
Makro zum aktualisieren der Seitenränder, falls eine der Optionen \opt{top} etc. geändert wurde. Ruft sich anschließend selbst auf.
\begin{MacroCode}{class}
\DeclareDocumentCommand \setgeometry { } {
  \bool_if:NTF \g_edu_twoup_bool {
    \geometry{a4paper,%
      top=\g_edu_twouptop_dim, bottom=\g_edu_twoupbottom_dim,%
      left=\g_edu_twoupleft_dim, right=\g_edu_twoupright_dim%
    }
  }{
    \geometry{a4paper,%
      top=\g_edu_top_dim, bottom=\g_edu_bottom_dim,%
      left=\g_edu_left_dim, right=\g_edu_right_dim%
    }
  }
  
  \bool_if:NTF \g_edu_footer_bool {
    \bool_if:NTF \g_edu_twoup_bool {
      \setlength{\footskip}{\g_edu_twoupfootskip_dim}
    }{
      \setlength{\footskip}{\g_edu_footskip_dim}
    }
    \geometry{includefoot}
  }{
    \setlength{\footskip}{0pt}
  }
  
  \bool_if:NT \g_edu_transparency_bool {
    \keys_set:nn {edu}{footer=false}
    \geometry{a4paper,%
      top=\g_edu_transparencytop_dim, bottom=\g_edu_transparencybottom_dim,%
      left=\g_edu_transparencyleft_dim, right=\g_edu_transparencyright_dim%
    }
  }
  \bool_if:NT \g_edu_glue_bool {
    \geometry{a4paper,%
      top=\g_edu_gluetop_dim, bottom=\g_edu_gluebottom_dim,%
      left=\g_edu_glueleft_dim, right=\g_edu_glueright_dim%
    }
  }
}

\AtEndPreamble{
  \setgeometry
}

\end{MacroCode}
\end{macro}


\subsubsection{Abschnitte (Parts, Sections, Subsection, etc.)}

\minisec{part}

\begin{macro*}{\g__edu_partbefore_dim}
\begin{macro*}{\g__edu_partafter_dim}
\begin{macro*}{\g__edu_partnumbertempwidth_dim}
\begin{macro*}{\g__edu_partnumberminwidth_dim}
First, some new lengths:
\begin{MacroCode}{class}
\dim_new:N \g__edu_partbefore_dim
\dim_new:N \g__edu_partafter_dim
\dim_gset:Nn \g__edu_partbefore_dim {3.5ex}
\dim_gset:Nn \g__edu_partafter_dim {1.25ex}

\dim_new:N \g__edu_partnumbertempwidth_dim
\dim_new:N \g__edu_partnumberminwidth_dim
\dim_gset:Nn \g__edu_partnumberminwidth_dim {0.9em}

\end{MacroCode}
\end{macro*}
\end{macro*}
\end{macro*}
\end{macro*}

Now, changing \cs{part} using titlesec:
\begin{MacroCode}{class}
\titleformat{\part}
  [hang]                                                      % Shape
  {\sffamily\bfseries\LARGE\color{\g_edu_partfg_tl}}          % Format
  {\setlength{\fboxsep}{0.125em}%
    \raisebox{0.1ex}{%
      \g_edu_partnumbersize_tl\colorbox{\g_edu_partnumberbg_tl}{%
        \settowidth{\g__edu_partnumbertempwidth_dim}{\thepart}%
        \dim_compare:nNnTF
        {\g__edu_partnumbertempwidth_dim} < {\g__edu_partnumberminwidth_dim} {
          \makebox[\g__edu_partnumberminwidth_dim]{%
            \textcolor{\g_edu_partnumberfg_tl}{\thepart}%
          }
        }{
            \textcolor{\g_edu_partnumberfg_tl}{\thepart}%
        }
      }
    }
  }                                  % Label
  {0.5em}                            % Sep
  {}                                 % Before
  {}                                 % After
  
\titlespacing*{\part}
  {0pt}                              % Left
  {\g__edu_partbefore_dim - \parskip}  % Beforesep
  {\g__edu_partafter_dim - \parskip}   % Aftersep

\end{MacroCode}


\minisec{section}

\begin{macro*}{\g__edu_sectionbefore_dim}
\begin{macro*}{\g__edu_sectionafter_dim}
First, some new lengths:
\begin{MacroCode}{class}
\dim_new:N \g__edu_sectionbefore_dim
\dim_new:N \g__edu_sectionafter_dim
\dim_gset:Nn \g__edu_sectionbefore_dim {2.5ex}
\dim_gset:Nn \g__edu_sectionafter_dim {0.75ex}

\end{MacroCode}
\end{macro*}
\end{macro*}

Now, changing \cs{section} using titlesec:
\begin{MacroCode}{class}
\titleformat{\section}
  [hang]                                                % Shape
  {\sffamily\bfseries\Large\color{\g_edu_sectionfg_tl}} % Format
  {\setlength{\fboxsep}{0.125em}%
    \raisebox{0.2ex}{%
      \g_edu_sectionnumbersize_tl%
      \colorbox{\g_edu_sectionnumberbg_tl}{%
        \textcolor{\g_edu_sectionnumberfg_tl}{\thesection}%
      }
    }
  }                                  % Label
  {0.5em}                            % Sep
  {}                                 % Before
  {}                                 % After
  
\titlespacing*{\section}
  {0pt}                                       % Left
  {\g__edu_sectionbefore_dim - \parskip}        % Beforesep
  {\g__edu_sectionafter_dim - \parskip}         % Aftersep
  
\end{MacroCode}


\minisec{subsection}

\begin{macro*}{\g__edu_subsectionbefore_dim}
\begin{macro*}{\g__edu_subsectionafter_dim}
First, some new lengths:
\begin{MacroCode}{class}
\dim_new:N \g__edu_subsectionbefore_dim
\dim_new:N \g__edu_subsectionafter_dim
\dim_gset:Nn \g__edu_subsectionbefore_dim {1.75ex}
\dim_gset:Nn \g__edu_subsectionafter_dim {0.75ex}

\end{MacroCode}
\end{macro*}
\end{macro*}

Now, changing \cs{subsection} using titlesec:
\begin{MacroCode}{class}
\titleformat{\subsection}
  [hang]                                                   % Shape
  {\sffamily\bfseries\large\color{\g_edu_subsectionfg_tl}} % Format
  {\setlength{\fboxsep}{0.125em}%
    \raisebox{0.15ex}{%
      \g_edu_subsectionnumbersize_tl%
      \colorbox{\g_edu_subsectionnumberbg_tl}{%
        \textcolor{\g_edu_subsectionnumberfg_tl}{\thesubsection}%
      }
    }
  }                                  % Label
  {0.5em}                             % Sep
  {}                                  % Before
  {}                                  % After
  
\titlespacing*{\subsection}
  {0pt}                                        % Left
  {\g__edu_subsectionbefore_dim - \parskip}      % Beforesep
  {\g__edu_subsectionafter_dim - \parskip}       % Aftersep

\end{MacroCode}


% TODO Later
\minisec{subsubsection}

Changing \cs{subsubsection} using titlesec:
\begin{MacroCode}{class}
\titleformat{\subsubsection}
  [hang]                               % Shape
  {\sffamily\bfseries\small}           % Format
  {\thesubsubsection}                  % Label
  {0.5em}                              % Sep
  {}                                   % Before
  {}                                   % After
  
\titlespacing*{\subsubsection}
  {0pt}                                % Left
  {1ex}                                % Beforesep
  {0.1ex}                              % Aftersep

\end{MacroCode}


\subsubsection{Absatzauszeichnung}

Es können Einzug (\opt{parindent}) und Abstand zwischen Absätzen (\opt{parskip}) kombiniert werden:

\begin{macro}{\setpar}
Setzt die Absatzauszeichnung.
\begin{MacroCode}{class}
\DeclareDocumentCommand \setpar { } {
  \bool_if:NTF \g_edu_parindent_bool {
    \bool_if:NT \g_edu_parskip_bool {
      \KOMAoptions{parskip=half}
    }
    \setlength{\parindent}{1em} % Important: After \KOMAoptions{parskip=half}
  }{
    \bool_if:NT \g_edu_parskip_bool {
      \KOMAoptions{parskip=half}
    }
    \setlength{\parindent}{0em}
  }
}

\AtBeginDocument{%
  \setpar%
}

\end{MacroCode}
\end{macro}


\begin{macro}{\parskipreduce}
An vielen Stellen muss der Absatzabstand nachträglich entfernt werden. Hierzu dient dieses Makro.
\begin{MacroCode}{class}
\DeclareDocumentCommand \parskipreduce { } {%
  \vspace{-\parskip}
}

\end{MacroCode}
\end{macro}


\subsubsection{Inhaltsverzeichnis}

Bei der Verwendung von Parts als höchste Gliederungsebene wird die Überschrift des Inhaltsverzeichnisses vergrößert.
\begin{MacroCode}{class}
\bool_if:NT \g_edu_parts_bool {
  \AtBeginDocument{
    \tl_gput_left:Nn \contentsname {\LARGE}
  }
}

\end{MacroCode}

Die Farbe und Schriftgröße von Parts im Inhaltsverzeichnis wird vergrößert.
\begin{MacroCode}{class}
\addtokomafont{partentry}{%  
  \color{wuDarkRed}\Large
}

\end{MacroCode}


\subsubsection{Titel}

Bei der Option \opt{parts} wird der Titel in der Schriftgröße \texttt{huge} gesetzt:
\begin{MacroCode}{class}
\AtEndPreamble{
  \bool_if:NT \g_edu_parts_bool {
    \tl_gput_right:Nn \g_edu_titlestyle_tl {
      \huge
    }
    \tl_gput_right:Nn \g_edu_subtitlestyle_tl {
      \LARGE
    }
  }
}

\end{MacroCode}


\begin{macro*}{\g__edu_titletext_tl}
\begin{macro}{\maketitle}
\begin{macro}{\maketitle*}
\begin{MacroCode}{class}
\tl_new:N \g__edu_titletext_tl

\DeclareDocumentCommand \maketitle {s} {
	
	\IfBooleanTF{#1} {
		  
		  
		\setlength{\parfillskip}{0em}
	  \par
	  \vspace{0.75ex}
	  \parskipreduce
	  \group_begin:
	    \g_edu_titlestyle_tl%
	    \color{\g_edu_titlefg_tl}%
	    \noindent\g_edu_title_tl%
	  \group_end:
	  \hfill%
	  \tl_if_empty:NTF \g_edu_group_tl {
	    \ ~
	  }{
	    \group_begin:
	      \setlength{\fboxsep}{0.25em}
	      \g_edu_groupstyle_tl%
	      \colorbox{\g_edu_groupbg_tl}{%
	        \textcolor{\g_edu_groupfg_tl}{\g_edu_group_tl}%
	      }%
	    \group_end:
	  }
	  \par
	  \setlength{\parfillskip}{1em plus 1fil}
	  
	  \tl_if_empty:NF \g_edu_subtitle_tl {
	    \parskipreduce
	    \group_begin:
	      \g_edu_subtitlestyle_tl
	      \color{\g_edu_titlefg_tl}\noindent\g_edu_subtitle_tl\par
	    \group_end:
	  }
		
	  \bool_if:NT \g_edu_transparency_bool {
	    \KOMAoption{fontsize}{\dim_use:N \g_edu_transparencyfontsize_dim}%
	  }

	}{ % star
		\tl_if_empty:NF \g_edu_title_tl {
		  \tl_put_right:Nn \g__edu_titletext_tl {
		    \group_begin:
		      \g_edu_titlestyle_tl \color{\g_edu_titlefg_tl} \noindent \g_edu_title_tl
		    \group_end:
		    \par\parskipreduce
		  }
		}
		
		\tl_if_empty:NF \g_edu_subtitle_tl {
		  \tl_put_right:Nn \g__edu_titletext_tl {
		    \vspace{1ex}
		    \group_begin:
		      \g_edu_subtitlestyle_tl \color{\g_edu_titlefg_tl} \noindent \g_edu_subtitle_tl
		    \group_end:
		    \par\parskipreduce
		  }
		}
		
		\bool_if:nT {!\tl_if_empty_p:N \g_edu_subject_tl || !\tl_if_empty_p:N \g_edu_field_tl || !\tl_if_empty_p:N \g_edu_class_tl} {
		  \tl_put_right:Nn \g__edu_titletext_tl {
		    \vspace{4ex}
		  }
		}
		
		\tl_if_empty:NF \g_edu_subject_tl {
		  \tl_put_right:Nn \g__edu_titletext_tl {
		    \group_begin:
		      \g_edu_subjectstyle_tl \noindent \g_edu_subject_tl
		    \group_end:
		  }
		}
		
		\bool_if:nT {!\tl_if_empty_p:N \g_edu_subject_tl && !\tl_if_empty_p:N \g_edu_field_tl} {
		  \tl_put_right:Nn \g__edu_titletext_tl {
		    \g_edu_subjectstyle_tl :~
		  }
		}
		
		\tl_if_empty:NF \g_edu_field_tl {
		  \tl_put_right:Nn \g__edu_titletext_tl {
		    \group_begin:
		      \g_edu_fieldstyle_tl \noindent \g_edu_field_tl
		    \group_end:
		  }
		}
		
		\bool_if:nT {!\tl_if_empty_p:N \g_edu_subject_tl || !\tl_if_empty_p:N \g_edu_field_tl} {
		  \tl_put_right:Nn \g__edu_titletext_tl {
		    \par\parskipreduce
		  }
		}
		
		\tl_if_empty:NF \g_edu_class_tl {
		  \tl_put_right:Nn \g__edu_titletext_tl {
		    \group_begin:
		      \g_edu_classstyle_tl \noindent \g_edu_class_tl
		    \group_end:
	      \par\parskipreduce
		  }
		}
		
		\bool_if:nT {!\tl_if_empty_p:N \g_edu_author_tl || !\tl_if_empty_p:N \g_edu_email_tl} {
		  \tl_put_right:Nn \g__edu_titletext_tl {
		    \vspace{3ex}
		  }
		}
		
		\tl_if_empty:NF \g_edu_author_tl {
		  \tl_put_right:Nn \g__edu_titletext_tl {
		    \group_begin:
		      \g_edu_authorstyle_tl \noindent \g_edu_author_tl
		    \group_end:
	      \par\parskipreduce
		  }
		}
		
		\tl_if_empty:NF \g_edu_email_tl {
		  \tl_put_right:Nn \g__edu_titletext_tl {
		    \group_begin:
		      \g_edu_emailstyle_tl \noindent \href{mailto:\g_edu_email_tl}{\nolinkurl{\g_edu_email_tl}}
		    \group_end:
	      \par\parskipreduce
		  }
		}
		
		\bool_if:nT {!\tl_if_empty_p:N \g_edu_date_tl || !\tl_if_empty_p:N \g_edu_version_tl || !\tl_if_empty_p:N \g_edu_license_tl} {
		  \tl_put_right:Nn \g__edu_titletext_tl {
		    \vspace{4ex}
		  }
		}
		
		\tl_if_empty:NF \g_edu_date_tl {
		  \tl_put_right:Nn \g__edu_titletext_tl {
		    \group_begin:
		      \g_edu_datestyle_tl \noindent \g_edu_date_tl
		    \group_end:
		  }
		}
		
		\bool_if:nT {!\tl_if_empty_p:N \g_edu_date_tl && !\tl_if_empty_p:N \g_edu_version_tl} {
		  \tl_put_right:Nn \g__edu_titletext_tl {
		    \g_edu_datestyle_tl,~
		  }
		}
		
		\tl_if_empty:NF \g_edu_version_tl {
		  \tl_put_right:Nn \g__edu_titletext_tl {
		    \group_begin:
		      \g_edu_versionstyle_tl \noindent \g_edu_version_tl
		    \group_end:
		  }
		}
		
		\bool_if:nT {!\tl_if_empty_p:N \g_edu_date_tl || !\tl_if_empty_p:N \g_edu_version_tl} {
		  \tl_put_right:Nn \g__edu_titletext_tl {
		    \par\parskipreduce
		  }
		}
		
		\tl_if_empty:NF \g_edu_license_tl {
		  \tl_put_right:Nn \g__edu_titletext_tl {
		    \group_begin:
		      \g_edu_licensestyle_tl \noindent \g_edu_license_tl
		    \group_end:
		    \par\parskipreduce
		  }
		}
		
	  \vspace*{1cm}
	  \begin{center}
	    \g__edu_titletext_tl
	  \end{center}
	  
	  \skip_vertical:N \g_edu_titleskip_skip
	
	  \bool_if:NT \g_edu_transparency_bool {
	    \KOMAoption{fontsize}{\dim_use:N \g_edu_transparencyfontsize_dim}%
	  }
	} % no star
}

\end{MacroCode}
\end{macro}
\end{macro}
\end{macro*}


\subsubsection{Intro}

\begin{macro}{\intro}
Dient zur Einleitung eines Textes vor den eigentlichen Aufgaben. Erzeugt lediglich einen Abstand zwischen dem Titel (durch \cs{maketitle}) und dem nachfolgenden Fließtext.
\begin{MacroCode}{class}
\DeclareDocumentCommand \intro { } {
  \vspace{2ex}
  \parskipreduce
}

\end{MacroCode}
\end{macro}


\subsubsection{Abstract (Zusammenfassung)}

Die Absatzauszeichnung muss für den Abstract gesondert angewendet werden:
\begin{MacroCode}{class}
\tl_gput_right:Nn \quotation {
  \setpar
}

\bool_if:NF \g_edu_parindent_bool {	
	\tl_gput_right:Nn \quotation {
	  \noindent\ignorespaces
	}
}

\AtEndEnvironment{abstract}{\vspace{3em}}

\end{MacroCode}


\subsubsection{Kopfzeile}

\begin{macro}{\makeheader}
Dieses Makro erstellt in Abhängig der relevanten Optionen die Kopfzeile der ersten Seite.
\begin{MacroCode}{class}
\DeclareDocumentCommand \makeheader { } {
  \group_begin:
    \sffamily\small
    \group_begin:
      \noindent\g_edu_subjectshort_tl \hfill \g_edu_dateshort_tl
    \group_end:\\
    \group_begin:
      \noindent\g_edu_fieldshort_tl \hfill \g_edu_classshort_tl
    \group_end:\\
  \group_end:
  \noindent\rule[0.5em]{\textwidth}{\g_edu_headerrulewidth_dim}
}



\end{MacroCode}
\end{macro}


\subsubsection{Fußzeile}

Die Fußzeilen werden durch das Package \pkg{scrheadings} realisiert. Dieses wird zuerst aktiviert und die standard Kopf- und Fußzeilen gelöscht:
\begin{MacroCode}{class}
\AtBeginDocument {

	\pagestyle{scrheadings}
	\clearscrheadings
	\clearscrheadfoot

  \bool_if:NT \g_edu_footer_bool {
    
\end{MacroCode}

Abhängig von den getroffenen oder nicht getroffenen Angaben von \cs{author}, \cs{field} etc. wird der Inhalt der inneren Seite der Fußzeile (bei einseitigem Satz links) im Makro \cs{__edu_footertext} erzeugt und an der entsprechenden Position der Fußzeile eingefügt:
\begin{macro*}{\l__edu_footertext_tl}
\begin{MacroCode}{class}
  \tl_new:N \l__edu_footertext_tl
  
  \tl_if_empty:NF \g_edu_licenseshort_tl {
    \tl_put_right:Nn \l__edu_footertext_tl {
      \g_edu_licenseshort_tl\hspace{0.4em}
    }
  }
  
  \tl_if_empty:NF \g_edu_authorshort_tl {
    \tl_put_right:Nn \l__edu_footertext_tl {
      \textsc{\g_edu_authorshort_tl}\hspace{0.4em}
    }
  }
  
  \bool_if:nT {
    !\tl_if_empty_p:N \g_edu_fieldshort_tl || 
    !\tl_if_empty_p:N \g_edu_titleshort_tl
  }{
    \tl_put_right:Nn \l__edu_footertext_tl {
      \textbar\hspace{0.4em}
    }
  }  
  
  \tl_if_empty:NF \g_edu_fieldshort_tl {
    \tl_put_right:Nn \l__edu_footertext_tl {
      \g_edu_fieldshort_tl
    }
  }
  
  \bool_if:nT {
    !\tl_if_empty_p:N \g_edu_fieldshort_tl && 
    !\tl_if_empty_p:N \g_edu_titleshort_tl
  }{
    \tl_put_right:Nn \l__edu_footertext_tl {
      :\
    }
  }
  
  \tl_if_empty:NF \g_edu_titleshort_tl {
    \tl_put_right:Nn \l__edu_footertext_tl {
      \g_edu_titleshort_tl
    }
  }
  
  \tl_if_empty:NF \g_edu_versionshort_tl {
    \tl_put_right:Nn \l__edu_footertext_tl {
      \quad(\g_edu_versionshort_tl)
    }
  }
  
  \refoot[]{\l__edu_footertext_tl}  % gerade rechts
  \lofoot[]{\l__edu_footertext_tl}  % ungerade links

\end{MacroCode}
\end{macro*}

Die andere Seite der Fußzeile wird mit der Seitenanzahl (\cs{pagemark}) und -- je nach Wert der Option \opt{pagecount} -- zusätzlich der Anzahl der Seiten (\cs{pageref{LastPage}}) versehen:
\begin{macro*}{\l__edu_groupfooter_tl}
\begin{MacroCode}{class}
  
  \tl_new:N \l__edu_groupfooter_tl
  \tl_set:Nn \l__edu_groupfooter_tl {
    \tl_if_empty:NTF \g_edu_group_tl {
	    \ ~
	  }{
      \setlength{\fboxsep}{0.25em}
      \g_edu_groupstyle_tl\small%
      \colorbox{black}{%
        \textcolor{\g_edu_groupfg_tl}{\g_edu_groupshort_tl}%
      }%
    }
    \quad
  }
  
  \bool_if:NT \g_edu_pagecount_bool {
    \rofoot[]{\l__edu_groupfooter_tl\small\normalfont\sffamily%
      \pagemark/\pageref*{LastPage}}         % gerade rechts
    \lefoot[]{\l__edu_groupfooter_tl\small\normalfont\sffamily%
      \pagemark/\pageref*{LastPage}}         % ungerade links
  }{
    \lefoot[]{\l__edu_groupfooter_tl\pagemark}                    % gerade links
    \rofoot[]{\l__edu_groupfooter_tl\pagemark}                    % ungerade rechts
  }

\end{MacroCode}
\end{macro*}


Kopfzeilen bleiben leer:
\begin{MacroCode}{class}
  \lehead[]{}  % gerade links
  \rehead[]{}  % gerade rechts
  \lohead[]{}  % ungerade links
  \rohead[]{}  % ungerade rechts

\end{MacroCode}

Die Formatierung von \cs{pagemark} und der Fußzeile wird durch \pkg{scrheadings} vorgenommen und muss gesondert vorgenommen werden:
\begin{MacroCode}{class}
  \setkomafont{pagenumber}{
    \small\normalfont\sffamily
  }

  \setkomafont{pagefoot}{
    \footnotesize\normalfont\sffamily
  }
  }{ % \bool_if:NT \g_edu_footer_bool
    \pagestyle{empty}
  }
} % \AtBeginDocument

\end{MacroCode}


\subsubsection{Typographie}


\begin{macro}{\textsfbf}[1]{<text>}
\begin{macro}{\cemph}[1]{<text>}
\begin{macro}{\csfemph}[1]{<text>}
\begin{macro}{\uline}
Formatierungskommandos:
\begin{MacroCode}{class}
\DeclareDocumentCommand \textsfbf { m } {
  \text{\sffamily\bfseries #1}
}

\DeclareDocumentCommand \cemph { m } {
  \textcolor{\g_edu_cemphfg_tl}{#1}
}

\DeclareDocumentCommand \csfemph { m } {
  \textcolor{\g_edu_cemphfg_tl}{\sffamily #1}
}

% Create command \ulined working as a underlined switch (ulem)
\useunder{\uline}{\ulined}{}

\end{MacroCode}
\end{macro}
\end{macro}
\end{macro}
\end{macro}


\begin{macro}{\textrightarrow}
\begin{macro}{\textRightarrow}
\begin{MacroCode}{class}
\DeclareDocumentCommand \textrightarrow { } {$\rightarrow$\xspace}
\DeclareDocumentCommand \textRightarrow { } {$\Rightarrow$\xspace}

\end{MacroCode}
\end{macro}
\end{macro}

Das Eurozeichen der Tastatur als \cs{euro}-Makro auffassen, dass mithilfe des Packages \textsf{eurosym} das Eurosymbols setzt:
\begin{MacroCode}{class}
\DeclareUnicodeCharacter{20AC}{\euro}

\end{MacroCode}

\begin{macro}{\today*}
Analog zu \cs{today} stellt die hier definierte Sternvariante das aktuelle Datum an. Es wird jedoch im Format T.M.JJJJ (bzw. D.M.YYYY) dargestellt.
\begin{MacroCode}{class}
\AfterPreamble{
  \cs_new_eq:NN \__edu_todaytemp \today
	\DeclareDocumentCommand \today { s } {
	  \IfBooleanTF{#1} {
	    \the\day.\the\month.\the\year
	  }{
	    \__edu_todaytemp
	  }
	}
}

\end{MacroCode}
\end{macro}


\begin{macro}{\headrule}
Das folgende Makro erzeugt eine \cs{midrule} des \pkg{booktabs}-Packages in der Stärke \cs{heavyrulethick} zum optisch auffallenderen Trennen des Kopfes einer Tabelle und deren Inhalt:
\begin{MacroCode}{class}
\DeclareExpandableDocumentCommand \headrule { } {
  \midrule[\heavyrulewidth]
}

\end{MacroCode}
\end{macro}


\subsubsection{Einheiten}

Konfiguration des Packages \pkg{siunitx} nach deutscher Konvention:
\begin{MacroCode}{class}
\sisetup{%
  locale=DE,%
  per-mode=fraction,%
  list-final-separator={ und },%
  list-pair-separator={ und },%
  list-separator={; },%
  range-phrase={ bis }
}

\end{MacroCode}

Häufig genutzte Einheiten werden definiert:
\begin{MacroCode}{class}
\DeclareSIUnit \scm{\square\centi\metre}
\DeclareSIUnit \sm{\square\metre}
\DeclareSIUnit \skm{\square\kilo\metre}

\DeclareSIUnit \qcm{\square\centi\metre}
\DeclareSIUnit \qm{\square\metre}
\DeclareSIUnit \qkm{\square\kilo\metre}

\DeclareSIUnit \ccm{\cubic\centi\metre}
\DeclareSIUnit \cm{\cubic\metre}
\DeclareSIUnit \ckm{\cubic\kilo\metre}

\DeclareSIUnit \cmps{\centi\metre\per\second}
\DeclareSIUnit \mps{\metre\per\second}
\DeclareSIUnit \kmps{\kilo\metre\per\second}

\DeclareSIUnit \cmph{\centi\metre\per\hour}
\DeclareSIUnit \mph{\metre\per\hour}
\DeclareSIUnit \kmph{\kilo\metre\per\hour}

\end{MacroCode}


\subsection{Abkürzungen, Symbole etc.}

\subsubsection{Abkürzungen} 

\begin{macro}{\dh}
\begin{macro}{\Dh}
\begin{macro}{\so}
\begin{macro}{\So}
\begin{macro}{\su}
\begin{macro}{\Su}
\begin{macro}{\ua}
\begin{macro}{\Ua}
\begin{macro}{\uU}
\begin{macro}{\UU}
\begin{macro}{\zB}
\begin{macro}{\ZB}
Deutsche Abkürzungen:
\begin{MacroCode}{class}
\DeclareDocumentCommand \dh { } {d.\,h.\xspace}
\DeclareDocumentCommand \Dh { } {D.\,h.\xspace}
\DeclareDocumentCommand \so { } {s.\,o.\xspace}
\DeclareDocumentCommand \So { } {S.\,o.\xspace}
\DeclareDocumentCommand \su { } {s.\,u.\xspace}
\DeclareDocumentCommand \Su { } {S.\,u.\xspace}
\DeclareDocumentCommand \ua { } {u.\,a.\xspace}
\DeclareDocumentCommand \Ua { } {U.\,a.\xspace}
\DeclareDocumentCommand \uU { } {u.\,U.\xspace}
\DeclareDocumentCommand \UU { } {U.\,U.\xspace}
\DeclareDocumentCommand \zB { } {z.\,B.\xspace}
\DeclareDocumentCommand \ZB { } {Z.\,B.\xspace}

\end{MacroCode}
\end{macro}
\end{macro}
\end{macro}
\end{macro}
\end{macro}
\end{macro}
\end{macro}
\end{macro}
\end{macro}
\end{macro}
\end{macro}
\end{macro}


\subsubsection{Siehe Abschnitte, siehe Abbildungen, etc.}

\begin{macro}{\see}
\begin{macro}{\seea}
\begin{macro}{\seee}
\begin{macro}{\seef}
\begin{macro}{\seel}
\begin{macro}{\seer}
\begin{macro}{\sees}
\begin{macro}{\seesol}
Geklammerte 'Siehe'-Verweise auf Abschnitte, Abbildungen, etc:
\begin{MacroCode}{class}
\DeclareDocumentCommand \see { m } {%
  \g_edu_seeleft_tl%
  \g_edu_seelabel_tl\g_edu_seelabelsep_tl\ref{#1}%
  \g_edu_seeright_tl%
}

\DeclareDocumentCommand \seea { m } {%
  \g_edu_seeleft_tl%
  \g_edu_seelabel_tl\g_edu_seelabelsep_tl%
  \g_edu_seeappendixlabel_tl\g_edu_seerefsep_tl\ref{#1}%
  \g_edu_seeright_tl%
}

\DeclareDocumentCommand \seee { m } {%
  \g_edu_seeleft_tl%
  \g_edu_seelabel_tl\g_edu_seelabelsep_tl%
  \g_edu_seeexerciselabel_tl\g_edu_seerefsep_tl\ref{#1}%
  \g_edu_seeright_tl%
}

\DeclareDocumentCommand \seef { m } {%
  \g_edu_seeleft_tl%
  \g_edu_seelabel_tl\g_edu_seelabelsep_tl%
  \g_edu_seefigurelabel_tl\g_edu_seerefsep_tl\ref{#1}%
  \g_edu_seeright_tl%
}

\DeclareDocumentCommand \seel { m } {%
  \g_edu_seeleft_tl%
  \g_edu_seelabel_tl\g_edu_seelabelsep_tl%
  \g_edu_seelistinglabel_tl\g_edu_seerefsep_tl\ref{#1}%
  \g_edu_seeright_tl%
}

\DeclareDocumentCommand \seer { m } {%
  \g_edu_seeleft_tl%
  \g_edu_seelabel_tl\g_edu_seelabelsep_tl%
  \ref{#1}%
  \g_edu_seeright_tl%
}

\DeclareDocumentCommand \sees { m } {%
  \g_edu_seeleft_tl%
  \g_edu_seelabel_tl\g_edu_seelabelsep_tl%
  \g_edu_seesectionlabel_tl\g_edu_seerefsep_tl\ref{#1}%
  \g_edu_seeright_tl%
}

\DeclareDocumentCommand \seesol { m } {%
  \g_edu_seeleft_tl%
  \g_edu_seelabel_tl\g_edu_seelabelsep_tl%
  \g_edu_seesolutionlabel_tl\g_edu_seerefsep_tl\ref{#1}%
  \g_edu_seeright_tl%
}

\end{MacroCode}
\end{macro}
\end{macro}
\end{macro}
\end{macro}
\end{macro}
\end{macro}
\end{macro}
\end{macro}


\subsubsection{Aufgaben angeben: S.\,X, Nr.\,Y}

\begin{macro}{\pgno}[2]{[<pagenumber>]}{[<no>]}
Zur Angabe von Aufgaben im Format S.\,X, Nr.\,Y.
\begin{MacroCode}{class}

\DeclareDocumentCommand \pgno { O{} O{} } {
  \tl_if_empty:nF {#1} {
    \g_edu_pglabel_tl\,#1\xspace
  }
  
  \bool_if:nT {!\tl_if_empty_p:n {#1} && !\tl_if_empty_p:n {#2}} {
    ,\,
  }
  
  \tl_if_empty:nF {#2} {
    \g_edu_nolabel_tl\,#2\xspace
  }
}

\end{MacroCode}
\end{macro}


\subsubsection{Kasten \emph{Bitte wenden}}

\begin{macro}{\pto}
Setzt einen kleinen Kasten mit der Aufforderung zum Wenden des Blattes in die rechte untere Ecke.
\begin{MacroCode}{class}
\DeclareDocumentCommand \pto {  } {%
	\vfill
	\par
	\setlength{\parfillskip}{0pt}
	\hfill\fbox{\g_edu_ptolabelstyle_tl \g_edu_ptolabel_tl \g_ptosymbolstyle_tl \g_ptosymbol_tl}
	\par
	\setlength{\parfillskip}{1em plus 1fil}
}

\end{MacroCode}
\end{macro}


\subsubsection{Creative Commons Lizenz}

\begin{macro}{\__edu_ccScale:n}[1]{<scale factor>}
First, the control sequence \cs{__edu_ccScale} is declared to scale the  icons:
\begin{MacroCode}{class}
\cs_new:Npn \__edu_ccScale:n #1 {
  \scalebox{\fp_use:N \g_edu_ccscale_fp}{#1}
}

\end{MacroCode}
\end{macro}


\begin{macro*}{\__edu_ccLogo}
\begin{macro*}{\__edu_ccAttribution}
\begin{macro*}{\__edu_ccShareAlike}
\begin{macro*}{\__edu_ccNoDerivatives}
\begin{macro*}{\__edu_ccNonCommercial}
\begin{macro*}{\__edu_ccNonCommercialEU}
\begin{macro*}{\__edu_ccNonCommercialJP}
\begin{macro*}{\__edu_ccZero}
\begin{macro*}{\__edu_ccPublicDomain}
\begin{macro*}{\__edu_ccSampling}
\begin{macro*}{\__edu_ccShare}
\begin{macro*}{\__edu_ccRemix}
\begin{macro*}{\__edu_ccCopy}
\begin{macro*}{\__edu_ccby}
\begin{macro*}{\__edu_ccbysa}
\begin{macro*}{\__edu_ccbynd}
\begin{macro*}{\__edu_ccbync}
\begin{macro*}{\__edu_ccbynceu}
\begin{macro*}{\__edu_ccbyncjp}
\begin{macro*}{\__edu_ccbyncsa}
\begin{macro*}{\__edu_ccbyncsaeu}
\begin{macro*}{\__edu_ccbyncsajp}
\begin{macro*}{\__edu_ccbyncnd}
\begin{macro*}{\__edu_ccbyncndeu}
\begin{macro*}{\__edu_ccbyncndjp}
\begin{macro*}{\__edu_cczero}
\begin{macro*}{\__edu_ccpd}
Copies of all icons are created:
\begin{MacroCode}{class}
\cs_new_eq:NN \__edu_ccLogo \ccLogo
\cs_new_eq:NN \__edu_ccAttribution \ccAttribution
\cs_new_eq:NN \__edu_ccShareAlike \ccShareAlike
\cs_new_eq:NN \__edu_ccNoDerivatives \ccNoDerivatives
\cs_new_eq:NN \__edu_ccNonCommercial \ccNonCommercial
\cs_new_eq:NN \__edu_ccNonCommercialEU \ccNonCommercialEU
\cs_new_eq:NN \__edu_ccNonCommercialJP \ccNonCommercialJP
\cs_new_eq:NN \__edu_ccZero \ccZero
\cs_new_eq:NN \__edu_ccPublicDomain \ccPublicDomain
\cs_new_eq:NN \__edu_ccSampling \ccSampling
\cs_new_eq:NN \__edu_ccShare \ccShare
\cs_new_eq:NN \__edu_ccRemix \ccRemix
\cs_new_eq:NN \__edu_ccCopy \ccCopy

\cs_new_eq:NN \__edu_ccby \ccby
\cs_new_eq:NN \__edu_ccbysa \ccbysa
\cs_new_eq:NN \__edu_ccbynd \ccbynd
\cs_new_eq:NN \__edu_ccbync \ccbync
\cs_new_eq:NN \__edu_ccbynceu \ccbynceu
\cs_new_eq:NN \__edu_ccbyncjp \ccbyncjp
\cs_new_eq:NN \__edu_ccbyncsa \ccbyncsa
\cs_new_eq:NN \__edu_ccbyncsaeu \ccbyncsaeu
\cs_new_eq:NN \__edu_ccbyncsajp \ccbyncsajp
\cs_new_eq:NN \__edu_ccbyncnd \ccbyncnd
\cs_new_eq:NN \__edu_ccbyncndeu \ccbyncndeu
\cs_new_eq:NN \__edu_ccbyncndjp \ccbyncndjp
\cs_new_eq:NN \__edu_cczero \cczero
\cs_new_eq:NN \__edu_ccpd \ccpd

\end{MacroCode}
\end{macro*}
\end{macro*}
\end{macro*}
\end{macro*}
\end{macro*}
\end{macro*}
\end{macro*}
\end{macro*}
\end{macro*}
\end{macro*}
\end{macro*}
\end{macro*}
\end{macro*}
\end{macro*}
\end{macro*}
\end{macro*}
\end{macro*}
\end{macro*}
\end{macro*}
\end{macro*}
\end{macro*}
\end{macro*}
\end{macro*}
\end{macro*}
\end{macro*}
\end{macro*}
\end{macro*}


\begin{macro}{\ccLogo}
\begin{macro}{\ccAttribution}
\begin{macro}{\ccShareAlike}
\begin{macro}{\ccNoDerivatives}
\begin{macro}{\ccNonCommercial}
\begin{macro}{\ccNonCommercialEU}
\begin{macro}{\ccNonCommercialJP}
\begin{macro}{\ccZero}
\begin{macro}{\ccPublicDomain}
\begin{macro}{\ccSampling}
\begin{macro}{\ccShare}
\begin{macro}{\ccRemix}
\begin{macro}{\ccCopy}
User commands with scaled (starred) versions are created:
\begin{MacroCode}{class}
\DeclareDocumentCommand \ccLogo { s } {
  \IfBooleanTF {#1} {
    \__edu_ccScale:n{\__edu_ccLogo}
  }{
    \__edu_ccLogo
  }
}

\DeclareDocumentCommand \ccAttribution { s } {
  \IfBooleanTF {#1} {
    \__edu_ccScale:n{\__edu_ccAttribution}
  }{
    \__edu_ccAttribution
  }
}

\DeclareDocumentCommand \ccShareAlike { s } {
  \IfBooleanTF {#1} {
    \__edu_ccScale:n{\__edu_ccShareAlike}
  }{
    \__edu_ccShareAlike
  }
}

\DeclareDocumentCommand \ccNoDerivatives { s } {
  \IfBooleanTF {#1} {
    \__edu_ccScale:n{\__edu_ccNoDerivatives}
  }{
    \__edu_ccNoDerivatives
  }
}

\DeclareDocumentCommand \ccNonCommercial { s } {
  \IfBooleanTF {#1} {
    \__edu_ccScale:n{\__edu_ccNonCommercial}
  }{
    \__edu_ccNonCommercial
  }
}

\DeclareDocumentCommand \ccNonCommercialEU { s } {
  \IfBooleanTF {#1} {
    \__edu_ccScale:n{\__edu_ccNonCommercialEU}
  }{
    \__edu_ccNonCommercialEU
  }
}

\DeclareDocumentCommand \ccNonCommercialJP { s } {
  \IfBooleanTF {#1} {
    \__edu_ccScale:n{\__edu_ccNonCommercialJP}
  }{
    \__edu_ccNonCommercialJP
  }
}

\DeclareDocumentCommand \ccZero { s } {
  \IfBooleanTF {#1} {
    \__edu_ccScale:n{\__edu_ccZero}
  }{
    \__edu_ccZero
  }
}

\DeclareDocumentCommand \ccPublicDomain { s } {
  \IfBooleanTF {#1} {
    \__edu_ccScale:n{\__edu_ccPublicDomain}
  }{
    \__edu_ccPublicDomain
  }
}

\DeclareDocumentCommand \ccSampling { s } {
  \IfBooleanTF {#1} {
    \__edu_ccScale:n{\__edu_ccSampling}
  }{
    \__edu_ccSampling
  }
}

\DeclareDocumentCommand \ccShare { s } {
  \IfBooleanTF {#1} {
    \__edu_ccScale:n{\__edu_ccShare}
  }{
    \__edu_ccShare
  }
}

\DeclareDocumentCommand \ccRemix { s } {
  \IfBooleanTF {#1} {
    \__edu_ccScale:n{\__edu_ccRemix}
  }{
    \__edu_ccRemix
  }
}

\DeclareDocumentCommand \ccCopy { s } {
  \IfBooleanTF {#1} {
    \__edu_ccScale:n{\__edu_ccCopy}
  }{
    \__edu_ccCopy
  }
}

\end{MacroCode}
\end{macro}
\end{macro}
\end{macro}
\end{macro}
\end{macro}
\end{macro}
\end{macro}
\end{macro}
\end{macro}
\end{macro}
\end{macro}
\end{macro}
\end{macro}


\begin{macro}{\ccby*}
\begin{macro}{\ccbysa*}
\begin{macro}{\ccbynd*}
\begin{macro}{\ccbync*}
\begin{macro}{\ccbynceu*}
\begin{macro}{\ccbyncjp*}
\begin{macro}{\ccbyncsa*}
\begin{macro}{\ccbyncsaeu*}
\begin{macro}{\ccbyncsajp*}
\begin{macro}{\ccbyncnd*}
\begin{macro}{\ccbyncndeu*}
\begin{macro}{\ccbyncndjp*}
\begin{macro}{\cczero*}
\begin{macro}{\ccpd*}
Skalierte Symbole für Create Commons Lizenz:
\begin{MacroCode}{class}
\DeclareDocumentCommand \ccby { s } {
  \IfBooleanTF {#1} {
    \__edu_ccScale:n{\__edu_ccby}
  }{
    \__edu_ccby
  }
}

\DeclareDocumentCommand \ccbysa { s } {
  \IfBooleanTF {#1} {
    \__edu_ccScale:n{\__edu_ccbysa}
  }{
    \__edu_ccbysa
  }
}

\DeclareDocumentCommand \ccbynd { s } {
  \IfBooleanTF {#1} {
    \__edu_ccScale:n{\__edu_ccbynd}
  }{
    \__edu_ccbynd
  }
}

\DeclareDocumentCommand \ccbync { s } {
  \IfBooleanTF {#1} {
    \__edu_ccScale:n{\__edu_ccbync}
  }{
    \__edu_ccbync
  }
}

\DeclareDocumentCommand \ccbynceu { s } {
  \IfBooleanTF {#1} {
    \__edu_ccScale:n{\__edu_ccbynceu}
  }{
    \__edu_ccbynceu
  }
}

\DeclareDocumentCommand \ccbyncjp { s } {
  \IfBooleanTF {#1} {
    \__edu_ccScale:n{\__edu_ccbyncjp}
  }{
    \__edu_ccbyncjp
  }
}

\DeclareDocumentCommand \ccbyncsa { s } {
  \IfBooleanTF {#1} {
    \__edu_ccScale:n{\__edu_ccbyncsa}
  }{
    \__edu_ccbyncsa
  }
}

\DeclareDocumentCommand \ccbyncsaeu { s } {
  \IfBooleanTF {#1} {
    \__edu_ccScale:n{\__edu_ccbyncsaeu}
  }{
    \__edu_ccbyncsaeu
  }
}

\DeclareDocumentCommand \ccbyncsajp { s } {
  \IfBooleanTF {#1} {
    \__edu_ccScale:n{\__edu_ccbyncsajp}
  }{
    \__edu_ccbyncsajp
  }
}

\DeclareDocumentCommand \ccbyncnd { s } {
  \IfBooleanTF {#1} {
    \__edu_ccScale:n{\__edu_ccbyncnd}
  }{
    \__edu_ccbyncnd
  }
}

\DeclareDocumentCommand \ccbyncndeu { s } {
  \IfBooleanTF {#1} {
    \__edu_ccScale:n{\__edu_ccbyncndeu}
  }{
    \__edu_ccbyncndeu
  }
}

\DeclareDocumentCommand \ccbyncndjp { s } {
  \IfBooleanTF {#1} {
    \__edu_ccScale:n{\__edu_ccbyncndjp}
  }{
    \__edu_ccbyncndjp
  }
}

\DeclareDocumentCommand \cczero { s } {
  \IfBooleanTF {#1} {
    \__edu_ccScale:n{\__edu_cczero}
  }{
    \__edu_cczero
  }
}

\DeclareDocumentCommand \ccpd { s } {
  \IfBooleanTF {#1} {
    \__edu_ccScale:n{\__edu_ccpd}
  }{
    \__edu_ccpd
  }
}

\end{MacroCode}
\end{macro}
\end{macro}
\end{macro}
\end{macro}
\end{macro}
\end{macro}
\end{macro}
\end{macro}
\end{macro}
\end{macro}
\end{macro}
\end{macro}
\end{macro}
\end{macro}


\subsubsection{Symbole für Unterrichtsablauf}

\begin{macro}{\action}
\begin{macro}{\speech}
Symbole für Handlung oder Sprache -- erstellt mit \pkg{tikz}.
\begin{MacroCode}{class}
\ExplSyntaxOff % Important! TikZ has problems with expl3. Use \@usetl instead.
\DeclareDocumentCommand \action { } {%
  \raisebox{0.5ex}{%
    \resizebox{1em}{!}{%
      \tikz \draw[\@usetl{actionfg}{tl}, ->, line width=2.5pt] (0,0) .. controls (0.1, 0.1) ..  (0.5,0);%
    }%
  }\xspace%
}

\DeclareDocumentCommand \speech { }{%
  \raisebox{0.5ex}{%
    \resizebox{1em}{\heightof{L}}{%
      \tikz \node[\@usetl{speechfg}{tl}, draw, fill, ellipse callout, callout relative pointer={(0.25,-0.25)}, callout pointer arc=40, xscale=-1] {\phantom{aa}};%
    }%
  }\xspace%
}

\ExplSyntaxOn

\end{MacroCode}
\end{macro}
\end{macro}


\subsection{Grafik}

\subsubsection{Vordefinierte Farben}

Definition der verwendeten Farben:
\begin{MacroCode}{class}
\definecolor{wuDarkRed}{HTML}{910000}
\definecolor{wuSemiDarkRed}{HTML}{a00000}
\definecolor{wuRed}{HTML}{F22222}

\definecolor{wuDarkerGray}{RGB}{75, 75, 75}  
\definecolor{wuDarkGray}{RGB}{160, 160, 160}  
\definecolor{wuGray}{RGB}{210, 210, 210}  
\definecolor{wuLightGray}{RGB}{227, 227, 227}  
\definecolor{wuLighterGray}{RGB}{235, 235, 235}  

\definecolor{wuBlue}{HTML}{3851FF}
\definecolor{wuGreen}{HTML}{009E0A}
\definecolor{wuOrange}{HTML}{FF8D29}
\definecolor{wuPink}{HTML}{FF5BF8}
\definecolor{wuViolet}{HTML}{9F32CC}
\definecolor{wuTurquoise}{HTML}{3E9DA6}
\definecolor{wuBrown}{HTML}{885704}

\end{MacroCode}


\subsubsection{Grafikpfad}

\begin{MacroCode}{class}
\AtBeginDocument{
  \graphicspath{\g_edu_graphicspath_tl}  
}

\end{MacroCode}


\subsubsection{TikZ}

Laden zusätzlicher \pkg{tikz}-Bibliotheken:
\begin{MacroCode}{class}
\ExplSyntaxOff  % Important! \usetikzlibrary doesn't work within \ExplSyntax*.
\usetikzlibrary{calc}
\usetikzlibrary{fadings}
\usetikzlibrary{patterns}  
\usetikzlibrary{positioning}  
\usetikzlibrary{shapes.callouts}

\end{MacroCode}

Konfiguration von \pkg{tikz}:
\begin{MacroCode}{class}
\tikzset{>=stealth}
\tikzset{font=\small}
\tikzset{execute at end picture={\renewcommand{\tikzScale}{1.0}}}

\end{MacroCode}

Der folgende Befehl behebt vom Adobe Reader falsch angezeigte Farben mit opacity-Option. Weshalb ist mir nicht bekannt.
\begin{MacroCode}{class}
\pdfpageattr{/Group << /S /Transparency /I true /CS /DeviceRGB>>}

\ExplSyntaxOn

\end{MacroCode}

Vordefinierte Werte, die innerhalb von Grafiken das Erstellen von Graphen verwendet werden können.

\begin{macro}{\g_edu_tikzscale_fp}
\begin{macro}{\tikzScale}
\begin{macro}{\tikzXStart}
\begin{macro}{\tikzXEnd}
\begin{macro}{\tikzYStart}
\begin{macro}{\tikzYEnd}
\begin{MacroCode}{class}
% TODO: pgfplots?
\fp_new:N \g_edu_tikzscale_fp
\fp_gset:Nn \g_edu_tikzscale_fp {1}

\DeclareDocumentCommand \tikzXStart { } {-3}
\DeclareDocumentCommand \tikzXEnd { } {3}
\DeclareDocumentCommand \tikzYStart { } {-3}
\DeclareDocumentCommand \tikzYEnd { } {3}

\end{MacroCode}
\end{macro}
\end{macro}
\end{macro}
\end{macro}
\end{macro}
\end{macro}


\begin{macro}{\tikzscale}[1]{<scale>}
\begin{macro}{\tikzscalefactor}
Wird innerhalb einer TikZ-Grafik das Makro \cs{tikzscalefactor} als 
Skalierungsfaktor verwendet, kann die Grafik durch Aufruf 
dieses Befehls skaliert werden.
\begin{MacroCode}{class}
\DeclareDocumentCommand \tikzscale { m } {
  \fp_gset:Nn \g_edu_tikzscale_fp {#1}
}

\DeclareExpandableDocumentCommand \tikzscalefactor { } {
  \fp_use:N \g_edu_tikzscale_fp
}

\end{MacroCode}
\end{macro}
\end{macro}

\begin{macro}{\tikzinput}[2]{[<scale>]}{<pgf-file>}
\begin{macro}{\tikzinput*}[2]{[<scale>]}{<pgf-file>}
Zum Laden von PGF-Dateien mit TikZ-Code aus dem Ordner/Pfad \cs{g_edu_tikzpath} kann dieses Makro verwendet werden. Die Sternvariante zentriert die Graphik durch Schachtelung in eine \env*{center}-Umgebung.
\begin{MacroCode}{class}
\DeclareDocumentCommand \tikzinput { s O{1.0} m } {%
  \fp_gset:Nn \g_edu_tikzscale_fp {#2}%
  \IfBooleanT {#1} {
    \begin{center}
  }
  \input{\g_edu_tikzpath_tl#3}%
  \IfBooleanT {#1} {
    \end{center}
  }
  \fp_set:Nn \g_edu_tikzscale_fp {1.0}%
}

\end{MacroCode}
\end{macro}
\end{macro}


\subsection{Einrichtung der \textsf{edu}-Klasse}

\begin{macro}{\edusetup}[1]{<key>=<val> list}
\begin{macro}{\eduoption}[2]{<key>}{<val>}
Wrapper für \cs{setkeys} zum Setzen der Optionen.
\begin{MacroCode}{class}
\DeclareDocumentCommand  \edusetup { m } {
  \keys_set:nn {edu} {#1}
}

\DeclareDocumentCommand  \eduoption { m m } {
  \keys_set:nn {edu} {#1=#2}
}

\end{MacroCode}
\end{macro}
\end{macro}


\subsection{Überschriften für Aufgaben etc.}

Die Überschriften der Aufgaben werden durch das Package \pkg{titlesec} implementiert. Exemplarisch wird an dieser Stelle die Abhandlung von Aufgaben besprochen. Die Überschriften für Lösungen usw. werden analog implementiert.


\subsubsection{Counter und Hilfsmakros}

Zum Zählen der Aufgabennummern werden zuerst Zähler definiert. Anschließend Hilfs-Makros erstellt, welche im späteren Verlauf zur Anzeige der Punkte und zum Zusammensetzen der Überschriften dienen.

\begin{macro}{\l__edu_exepoints_tl}
\begin{macro}{\l__edu_exelabeltext_tl}
\begin{macro}{\l__edu_subexelabeltext_tl}
\begin{MacroCode}{class}
\newcounter{exercisecounter}
\newcounter{subexercisecounter}[exercisecounter]
\newcounter{multiexecounter}

\renewcommand{\thesubexercisecounter}{\theexercisecounter.\arabic{subexercisecounter}}

\tl_new:N \l__edu_exepoints_tl        % For optional number points
\tl_new:N \l__edu_exelabeltext_tl     % Concatenation of the exercise-label
\tl_new:N \l__edu_subexelabeltext_tl  % Concatenation of the subexercise-label

\end{MacroCode}
\end{macro}
\end{macro}
\end{macro}

\begin{macro}{\l__edu_sollabeltext_tl}
\begin{macro}{\l__edu_subsollabeltext_tl}
\begin{MacroCode}{class}
\newcounter{solutioncounter}
\newcounter{subsolutioncounter}[solutioncounter]
\newcounter{multisolcounter}

\renewcommand{\thesubsolutioncounter}{\thesolutioncounter.\arabic{subsolutioncounter}}

\tl_new:N \l__edu_sollabeltext_tl     % Concatenation of the exercise-label
\tl_new:N \l__edu_subsollabeltext_tl  % Concatenation of the subexercise-label

\end{MacroCode}
\end{macro}
\end{macro}

\begin{macro}{\rstexe}
\begin{macro}{\rstsubexe}
\begin{macro}{\rstmultiexe}
Mit den folgenden Makros können Counter für Aufgaben manuell zurückgesetzt werden:
\begin{MacroCode}{class}
\DeclareDocumentCommand \rstexe { } {\setcounter{exercisecounter}{0}}
\DeclareDocumentCommand \rstsubexe { } {\setcounter{subexercisecounter}{0}}
\DeclareDocumentCommand \rstmultiexe { } {\setcounter{multiexecounter}{0}}

\end{MacroCode}
\end{macro}
\end{macro}
\end{macro}


\subsubsection{Überschriften-Klassen und Benutzermakros}

Nun werden die Überschriften-Klassen für Aufgaben (\cs{exercise}) und Unteraufgaben (\cs{subexercise}) erzeugt. Für beide Klassen werden außerdem Makros erstellt (\cs{exe} und \cs{subexe}), welche für den Benutzer zur Erzeugung einer (Unter-) Aufgabenüberschrift dienen. Durch sie werden die Punktzahlen unter Berücksichtigung der Label \opt{exepointslabel} und \opt{subexepointslabel} erzeugt und der Text der Überschriften \cs{exetext} bzw. \cs{subexelabeltext} unter Berücksichtugung der entsprechenden Parameter zusammengesetzt. Abschließend wird in \cs{exe} und \cs{subexe} durch den Aufruf von \cs{exercise} bzw. \cs{subexercise} die entsprechende Überschrift erzeugt.

Sowohl \cs{exercise} als auch \cs{subexercise} setzen den Counter für Teilaufgaben \cs{multiexecounter} auf Null.

\begin{macro}{\exe}[2]{[<number points>]}{<title>}
\begin{MacroCode}{class}
\DeclareDocumentCommand \exe { o m }{
  \refstepcounter{exercisecounter}
  
  \tl_clear:N \l__edu_exelabeltext_tl
  
  \IfNoValueTF {#1} {
    \tl_clear:N \l__edu_exepoints_tl
  }{
    \bool_if:NTF \g_edu_exepointsachieved_bool {
	    \tl_set:Nn \l__edu_exepoints_tl {
	      \g_edu_exepointsleft_tl
	        \hspace{\g_edu_exepointsachievedspace_dim}
	        \g_edu_exepointsachievedsep_tl
	        #1\g_edu_exepointslabel_tl
	      \g_edu_exepointsright_tl
	    }
	  }{
      \tl_set:Nn \l__edu_exepoints_tl {
	      \g_edu_exepointsleft_tl
	        #1\g_edu_exepointslabel_tl
	      \g_edu_exepointsright_tl
	    }
	  }
  }

  \tl_if_empty:nTF {#2} {
    \tl_if_empty:NTF \g_edu_exelabel_tl {
      \tl_clear:N \l__edu_exelabeltext_tl
    }{
      \tl_set:Nn \l__edu_exelabeltext_tl {
        \hspace{0.1em}\g_edu_exelabel_tl
      }
    }
  }{
    \tl_if_empty:NTF \g_edu_exelabel_tl {
      \tl_set:Nn \l__edu_exelabeltext_tl {
        \hspace{0.3em}
      }
    }{
      \tl_set:Nn \l__edu_exelabeltext_tl {
        \hspace{0.1em}\g_edu_exelabel_tl{:}\hspace{0.3em}
      }
    }
  }
    
  \setlength{\parfillskip}{0em} % Damit Punktzahl bei parkip wirklich rechtsbündig ist. Sonst rechts Abstand in letzter (einziger) Zeile.
  \vspace{\g_edu_exebeforeskip_dim}%
  \parskipreduce%
  {\g_edu_exelabelstyle_tl%
    \setlength{\fboxsep}{0.125em}%
    \raisebox{0.2ex}{%
      \colorbox{\g_edu_exenumberbg_tl}{%
        \textcolor{\g_edu_exenumberfg_tl}{%
          \g_edu_exenumberstyle_tl\theexercisecounter\g_edu_exenumberseparator_tl%
        }%
      }%
    }%
    \hspace{0.1em}%
    \colorbox{\g_edu_exelabelbg_tl}{%
      \textcolor{\g_edu_exelabelfg_tl}{\l__edu_exelabeltext_tl}%
    }%
  }%
  {
    \colorbox{\g_edu_exebg_tl}{%
      \textcolor{\g_edu_exefg_tl}{\g_edu_exestyle_tl #2}%
    }%
  }%
  \hfill%
  {\g_edu_exepointsstyle_tl%
    \textcolor{\g_edu_exepointsfg_tl}{%
      \l__edu_exepoints_tl%
    }%
  }%
  \nopagebreak\@afterheading%
  \vspace{\g_edu_exeafterskip_dim}%
  \parskipreduce%
  \par % Wichtig, damit \setlength{\parfillskip}{0em} wirkt (s. o.).
  \setlength{\parfillskip}{1em plus 1fil}
  
  \bool_if:NT \g_edu_exetoc_bool {
  \addcontentsline{toc}{section}{\theexercisecounter\g_edu_exenumberseparator_tl~\l__edu_exelabeltext_tl #2}
  }
  
  \setcounter{subexercisecounter}{0}
  \setcounter{multiexecounter}{0}
  \setcounter{question}{0}
  
%  \__edu_remove_multiexespace
%
%  \peek_meaning_ignore_spaces:NTF {L} {
%    \bool_if:NTF \g_edu_parskip {
%      \vspace{0.5\baselineskip}
%    }{
%      \vspace{-0.375\baselineskip}
%    }
%    Ja
%  }{
%    Nein
%  }
}

\end{MacroCode}
\end{macro}

\begin{macro}{\subexe}[2]{[<number points>]}{<title>}
\begin{MacroCode}{class}
\DeclareDocumentCommand \subexe { o m }{
  \refstepcounter{subexercisecounter}
  
  \tl_clear:N \l__edu_subexelabeltext_tl
  
  \IfNoValueTF {#1} {
    \tl_clear:N \l__edu_exepoints_tl
  }{
    \bool_if:NTF \g_edu_subexepointsachieved_bool {
	    \tl_set:Nn \l__edu_exepoints_tl {
	      \g_edu_subexepointsleft_tl
	        \hspace{\g_edu_subexepointsachievedspace_dim}
	        \g_edu_subexepointsachievedsep_tl
	        #1\g_edu_subexepointslabel_tl
	      \g_edu_subexepointsright_tl
	    }
	  }{
      \tl_set:Nn \l__edu_exepoints_tl {
	      \g_edu_subexepointsleft_tl
	        #1\g_edu_subexepointslabel_tl
	      \g_edu_subexepointsright_tl
	    }
	  }
  }

  \tl_if_empty:nTF {#2} {
    \tl_if_empty:NTF \g_edu_subexelabel_tl {
      \tl_clear:N \l__edu_subexelabeltext_tl
    }{
      \tl_set:Nn \l__edu_subexelabeltext_tl {
        \hspace{0.1em}\g_edu_subexelabel_tl
      }
    }
  }{
    \tl_if_empty:NTF \g_edu_subexelabel_tl {
      \tl_set:Nn \l__edu_subexelabeltext_tl {
        \hspace{0.3em}
      }
    }{
      \tl_set:Nn \l__edu_subexelabeltext_tl {
        \hspace{0.1em}\g_edu_subexelabel_tl{:}\hspace{0.3em}
      }
    }
  }
    
  \setlength{\parfillskip}{0em}
  \vspace{\g_edu_subexebeforeskip_dim}%
  \parskipreduce%
  {\g_edu_subexelabelstyle_tl%
    \setlength{\fboxsep}{0.125em}
    \raisebox{0.2ex}{%
      \colorbox{\g_edu_subexenumberbg_tl}{%
        \textcolor{\g_edu_subexenumberfg_tl}{%
          \g_edu_subexenumberstyle_tl%
          \thesubexercisecounter%
          \g_edu_subexenumberseparator_tl%
        }%
      }%
    }%
    \colorbox{\g_edu_subexelabelbg_tl}{%
      \textcolor{\g_edu_subexelabelfg_tl}{\l__edu_subexelabeltext_tl}%
    }%
  }%
  {%
    \colorbox{\g_edu_subexebg_tl}{%
      \textcolor{\g_edu_subexefg_tl}{\g_edu_subexestyle_tl #2}%
    }%
  }%
  \hfill%
  {\g_edu_subexepointsstyle_tl%
    \textcolor{\g_edu_subexepointsfg_tl}{%
      \l__edu_exepoints_tl%
    }%
  }%
  \nopagebreak\@afterheading
  \vspace{\g_edu_subexeafterskip_dim}%
  \parskipreduce%
  \par
  \setlength{\parfillskip}{1em plus 1fil}
  
  \bool_if:NT \g_edu_exetoc_bool {
  \addcontentsline{toc}{subsection}{\thesubexercisecounter\g_edu_subexenumberseparator_tl~\l__edu_subexelabeltext_tl #2}
  }
  
  \setcounter{multiexecounter}{0}
  \setcounter{question}{0}
}

\end{MacroCode}
\end{macro}
Die Implementierung der Überschriften für Lösungen erfolgt analog:

\begin{macro}{\sol}[1]{<title>}
\begin{MacroCode}{class}
\DeclareDocumentCommand \sol { m }{
  \refstepcounter{solutioncounter}
  
  \tl_clear:N \l__edu_sollabeltext_tl

  \tl_if_empty:nTF {#1} {
    \tl_if_empty:NTF \g_edu_sollabel_tl {
      \tl_clear:N \l__edu_sollabeltext_tl
    }{
      \tl_set:Nn \l__edu_sollabeltext_tl {
        \hspace{0.1em}\g_edu_sollabel_tl
      }
    }
  }{
    \tl_if_empty:NTF \g_edu_sollabel_tl {
      \tl_set:Nn \l__edu_sollabeltext_tl {
        \hspace{0.3em}
      }
    }{
      \tl_set:Nn \l__edu_sollabeltext_tl {
        \hspace{0.1em}\g_edu_sollabel_tl{:}\hspace{0.3em}
      }
    }
  }
  
  \vspace{\g_edu_solbeforeskip_dim}%
  \parskipreduce%
  {\g_edu_sollabelstyle_tl%
    \setlength{\fboxsep}{0.125em}%
    \raisebox{0.2ex}{%
      \colorbox{\g_edu_solnumberbg_tl}{%
        \textcolor{\g_edu_solnumberfg_tl}{%
          \g_edu_solnumberstyle_tl\thesolutioncounter\g_edu_solnumberseparator_tl%
        }%
      }%
    }%
    \hspace{0.1em}%
    \colorbox{\g_edu_sollabelbg_tl}{%
      \textcolor{\g_edu_sollabelfg_tl}{\l__edu_sollabeltext_tl}%
    }%
  }%
  {%
    \colorbox{\g_edu_solbg_tl}{%
      \textcolor{\g_edu_solfg_tl}{\g_edu_solstyle_tl #1}%
    }%
  }%
  \nopagebreak\@afterheading
  \vspace{\g_edu_solafterskip_dim}%
  \parskipreduce%
  
  \bool_if:NT \g_edu_exetoc_bool {
  \addcontentsline{toc}{section}{\thesolutioncounter\g_edu_solnumberseparator_tl~\l__edu_sollabeltext_tl #1}
  }
    
  \setcounter{subsolutioncounter}{0} 
  \setcounter{multiexecounter}{0}
}

\end{MacroCode}
\end{macro}

\begin{macro}{\subsol}[1]{<title>}
\begin{MacroCode}{class}
\DeclareDocumentCommand \subsol { m }{
  \refstepcounter{subsolutioncounter}
  
  \tl_clear:N \l__edu_subsollabeltext_tl

  \tl_if_empty:nTF {#1} {
    \tl_if_empty:NTF \g_edu_subsollabel_tl {
      \tl_clear:N \l__edu_subsollabeltext_tl
    }{
      \tl_set:Nn \l__edu_subsollabeltext_tl {
        \hspace{0.1em}\g_edu_subsollabel_tl
      }
    }
  }{
    \tl_if_empty:NTF \g_edu_subsollabel_tl {
      \tl_set:Nn \l__edu_subsollabeltext_tl {
        \hspace{0.3em}
      }
    }{
      \tl_set:Nn \l__edu_subsollabeltext_tl {
        \hspace{0.1em}\g_edu_subsollabel_tl{:}\hspace{0.3em}
      }
    }
  }
  
  \vspace{\g_edu_subsolbeforeskip_dim}%
  \parskipreduce%
  {\g_edu_subsollabelstyle_tl%
    \setlength{\fboxsep}{0.125em}
    \raisebox{0.2ex}{%
      \colorbox{\g_edu_subsolnumberbg_tl}{%
        \textcolor{\g_edu_subsolnumberfg_tl}{%
          \g_edu_subsolnumberstyle_tl%
          \thesubsolutioncounter%
          \g_edu_subsolnumberseparator_tl%
        }%
      }%
    }%
    \colorbox{\g_edu_subsollabelbg_tl}{%
      \textcolor{\g_edu_subsollabelfg_tl}{\l__edu_subsollabeltext_tl}%
    }%
  }%
  {%
    \colorbox{\g_edu_subsolbg_tl}{%
      \textcolor{\g_edu_subsolfg_tl}{\g_edu_subsolstyle_tl #1}%
    }%
  }%
  \nopagebreak\@afterheading
  \vspace{\g_edu_subsolafterskip_dim}%
  \parskipreduce%
  
  \bool_if:NT \g_edu_exetoc_bool {
  \addcontentsline{toc}{subsection}{\thesolutioncounter.\thesubsolutioncounter\g_edu_subsolnumberseparator_tl~\l__edu_subsollabeltext_tl #1}
  }
  
  \setcounter{multiexecounter}{0}
}

\end{MacroCode}
\end{macro}


\subsection{Teilaufgaben-Umgebungen}

\subsubsection{Nummerierung und Punkte der Teilaufgaben}

\begin{macro}{\multiexelabel}
\begin{macro}{\multiexelabelboxed}
Diese Makros liefern die aktuelle Nummerierung der Teilaufgabe in Kleinbuchstaben mit anschließender Klammer (a), b),\dots) und erhöhen den Counter um Eins. \cs{multiexelabelboxed} setzt die Buchstaben zusätzlich rechtsbündig in eine Box.
\begin{MacroCode}{class}
\DeclareDocumentCommand \multiexelabel { } {%
  \stepcounter{multiexecounter}%
  {\g_edu_multiexenumberstyle_tl%
    \textcolor{\g_edu_multiexenumberfg_tl}{%
      \g_edu_multiexenumberleft_tl\alph{multiexecounter}\g_edu_multiexenumberright_tl%
    }%
  }
}

\newcommand{\multiexelabelboxed}{%
  \stepcounter{multiexecounter}%    War \refstepcounter. Dies erzeugt vertikalen Abstand.
  \setlength{\fboxsep}{0pt}%
  \makebox[\g__edu_listlabelwidth_dim][r]{%
    \g_edu_multiexenumberstyle_tl%
    \textcolor{\g_edu_multiexenumberfg_tl}{%
      \g_edu_multiexenumberleft_tl\alph{multiexecounter}\g_edu_multiexenumberright_tl
    }%
  }%
}

\end{MacroCode}
\end{macro}
\end{macro}


\begin{macro}{\points}
\begin{macro}{\points*}
Setzt die Punktzahl der aktuellen Teilaufgabe zusammen. Die Sternvariante setzt die Punktzahl am rechten Rand der aktuellen Zeile.
\begin{MacroCode}{class}
\DeclareDocumentCommand \points { s m } {%
  \IfBooleanT{#1} {
	   \hspace*{0pt}\hfill\hspace{0.5em}%
	}
  ~\nolinebreak%  
  \text{%
    \g_edu_multiexepointsstyle_tl%
    \color{\g_edu_multiexepointsfg_tl}%
    \g_edu_multiexepointsleft_tl%
      #2\g_edu_multiexepointslabel_tl%
    \g_edu_multiexepointsright_tl%
   }%
}

\end{MacroCode}
\end{macro}
\end{macro}


\begin{macro}{\res}[2]{[<alternative text>]}{<text>}
Zeigt Ergebnisse in Abhängigkeit der Option \opt{showresults} an.
\begin{MacroCode}{class}
\DeclareDocumentCommand \res { O{} m} {%
  \bool_if:NTF \g_edu_showresults_bool {
    \textcolor{\g_edu_resultfg_tl}{#2}%
  }{#1}%
}

\end{MacroCode}
\end{macro}


\begin{macro}{\resr}[3]{[<rule length>]}{<text>}{[<vertical offset>]}
Zeigt Ergebnisse in Abhängigkeit der Option \opt{showresults} an. Ansonsten wird ein horizontale Linie beliebiger Länge angezeigt.
\begin{MacroCode}{class}
\DeclareDocumentCommand \resr { O{\g_edu_resultrulelength_dim} m O{\g_edu_resultrulevoffset_dim} } {%
  \bool_if:NTF \g_edu_showresults_bool {%
    \textcolor{\g_edu_resultfg_tl}{#2}%
  }{%
    \rule[-\g_edu_resultrule_dim+#3]{#1}{\g_edu_resultrule_dim}%
  }%
}

\end{MacroCode}
\end{macro}


\begin{macro}{\resb}[4]{[<box width>]}{[<box height>]}{<text>}{[<vertical offset>]}
Zeigt Ergebnisse in Abhängigkeit der Option \opt{showresults} an. Ansonsten wird ein horizontale Linie beliebiger Länge angezeigt.
\begin{MacroCode}{class}
\DeclareDocumentCommand \resb { O{\g_edu_resultboxwidth_dim} O{\g_edu_resultboxheight_dim} m O{\g_edu_resultboxvoffset_dim} } {%
  \bool_if:NTF \g_edu_showresults_bool {%
    \textcolor{\g_edu_resultfg_tl}{#3}%
  }{%
    \tikz[baseline=-#4] \draw (0,0) rectangle (#1, #2);
  }%
}

\end{MacroCode}
\end{macro}


\subsubsection{Aufgabenliste (einspaltiger Satz von Teilaufgaben)}

%\begin{macro}{\__edu_remove_multiexespace}
%\begin{MacroCode}{class}
%\cs_new:Npn \__edu_remove_multiexespace {
%  \peek_meaning_ignore_spaces:NTF L {
%    \bool_if:NTF \g_edu_parskip_bool {
%      \vspace{0.5\baselineskip}
%    }{
%      \vspace{-0.375\baselineskip}
%    }
%    Ja
%  }{
%    Nein
%  }
%}
%
%
%\end{MacroCode}
%\end{macro}

\begin{environment}{multiexelist}[1]{[<itemsep>]}
\begin{environment}{multiexelist*}[1]{[<itemsep>]}
Der einspaltige Satz von Teilaufgaben geschieht mithilfe einer angepassten Liste. Die normale Variante sollte verwendet werden, wenn vor den Teilaufgaben Fließtext steht. Folgen die Teilaufgaben direkt auf eine Aufgabenüberschrift (\cs{exe}, \cs{subexe}, \dots), sollte die Sternvariante verwendet werden, da hierbei der obere Abstand angepasst werden muss.
\begin{MacroCode}{class}
\DeclareDocumentEnvironment {multiexelist} { s O{0.5\baselineskip} } {
  \parskipreduce
  \IfBooleanT {#1} {
    \bool_if:NTF \g_edu_parskip_bool {
      \vspace{0.125\baselineskip}
    }{
      \vspace{-0.375\baselineskip}
    }
  }
  \begin{edulist}[#2]{\multiexelabelboxed}
}{
  \end{edulist}
  \parskipreduce
}

% Hack, because xparse doesn't generate starred environments:
\cs_new:cpn {multiexelist*} {\multiexelist*}
\cs_new_eq:cN {endmultiexelist*} \endmultiexelist

\end{MacroCode}
\end{environment}
\end{environment}


\subsubsection{Aufgabentabelle (mehrspaltiger Satz von Teilaufgaben)}

\begin{environment}{multiexearray}[2]{[<extrarowheight>]}{<number cols>}
Der mehrspaltige Satz von Teilaufgaben geschieht mithilfe einer \cs*{tabularx}-Tabelle und einem angepassten Spaltentyp \texttt{A}.

Die Zeilenhöhe kann optional verändert werden, wobei \cs{extrarowheight} verwendet wird. Anschließend werden alle für den Satz der Tabelle nötigen Längen gesetzt bzw. berechnet und der neue Spaltentyp definiert. Vor jede Spalte wird die Nummer der aktuellen Teilaufgabe durch \cs{multiexelabelboxed} gesetzt. Die Box wird benötigt, damit die Nummerierung rechtsbündig erfolgt. Da dieser Umgebung dem Satz von Fließtext dient, werden ab der zweiten Zeile alle Zeilen eingerückt, damit die Nummerierung links übersteht.

Die Sternvariante versetzt alle Zellen für Teilaufgaben in den Mathematik-Modus.
\begin{MacroCode}{class}
\DeclareDocumentEnvironment {multiexearray} { s O{0.25\baselineskip} m } {
  \IfBooleanTF {#1} {
	  \arraysetup[#2]{m}
    \noindent\tabularx{\textwidth}{*{#3}{E}}
	}{%
	  \arraysetup[#2]{}
	  \noindent\tabularx{\textwidth}{*{#3}{e}}%
  }
}{
  \endtabularx
  \arraycleanup
}

% Hack, because xparse doesn't generate starred environments:
\cs_new:cpn {multiexearray*} {\multiexearray*}
\cs_new_eq:cN {endmultiexearray*} \endmultiexearray

\end{MacroCode}
\end{environment}


\begin{macro}{\ls}
Abkürzung für vergrößerten vertikalen Abstand in Tabellenzeile.
\begin{MacroCode}{class}
\DeclareExpandableDocumentCommand \ls { } {\addlinespace[1.5ex]}

\end{MacroCode}
\end{macro}


\subsection{Fragen}

\subsubsection{Zähler, Längen und Überschriften}

\begin{macro}{\l__edu_questmclabelwidth_dim}
\begin{macro}{\l__edu_questpoints_tl}
Zuerst werden benötigte Zähler und Längen definiert und die Nummerierung und Beschriftung der Fragen als Titel definiert:
\begin{MacroCode}{class}
\newcounter{question}
\newcounter{@questmccounter}

\dim_new:N \l__edu_questmclabelwidth_dim
\dim_set:Nn \l__edu_questmclabelwidth_dim {0.9em}

\tl_new:N \l__edu_questpoints_tl

\end{MacroCode}
\end{macro}
\end{macro}


\begin{macro}{\quest}[2]{<number points>}{<text>}

Erzeugt eine neue Frage unter Berücksichtigung der relevanten Optionen.
\begin{MacroCode}{class}
\DeclareDocumentCommand \quest { o m } {%
  
  \refstepcounter{question}
  
  \tl_set:Nn \l__edu_questpoints_tl {#1}
  
  \setcounter{@questmccounter}{0}
  
  \vspace{\g_edu_questbeforeskip_dim}
  {\g_edu_questlabelstyle_tl%
    \textcolor{\g_edu_questlabelfg_tl}{%
      \thequestion.\,\g_edu_questlabel_tl%
    }%
  }%
  \IfNoValueTF {#1} {
    \hspace{\g_edu_questsep_dim}{\g_edu_queststyle_tl #2}
  }{%
    \hspace{\g_edu_questpointssep_dim}%
    {%
      \g_edu_questpointsstyle_tl\g_edu_questpointsleft_tl%
      \l__edu_questpoints_tl\g_edu_questpointslabel_tl%
      \g_edu_questpointsright_tl%
    }%
    \hspace{\g_edu_questsep_dim}{\g_edu_queststyle_tl #2}
  }
  \nopagebreak\@afterheading
  \vspace{\g_edu_questafterskip_dim}
}

\end{MacroCode}
\end{macro}


\subsubsection{Makros für Nummerierung bei alphabetischem Multiple Choice}

\begin{macro}{\g__edu_questmclabelboxed_tl}
\begin{MacroCode}{class}
\tl_new:N \g__edu_questmclabelboxed_tl
\tl_gset:Nn \g__edu_questmclabelboxed_tl {
  \stepcounter{@questmccounter}%    War \refstepcounter, erzeugt aber vert. Abstand.
  \setlength{\fboxsep}{0pt}%
  \makebox[\g__edu_listlabelwidth_dim][r]{%
    \raisebox{-0.125em}{%
      \color{\g_edu_questmclabelfg_tl}%
      \framebox[\l__edu_questmclabelwidth_dim][c]{%
        \rule{0pt}{\l__edu_questmclabelwidth_dim}%
        \raisebox{0.2em}{\footnotesize\alph{@questmccounter}}%
      }%
    }%
  }%
}

\end{MacroCode}
\end{macro}



\subsubsection{Umgebungen für Antworten}


\begin{macro}{\questblank}[1]{[<vert. space>]}
Erzeugt eine neue Frage unter Berücksichtigung der relevanten Optionen mit anschließendem Freiraum.
\begin{MacroCode}{class}
\DeclareDocumentCommand \questblank { O{3cm} } {%
  \vspace{#1}
  \parskipreduce
}

\end{MacroCode}
\end{macro}


\begin{macro}{\questtextblank}[3]{[<lineskip>]}{<number lines>}{<linewidth>}
Erzeugt eine neue Frage unter Berücksichtigung der relevanten Optionen mit anschließenden horizontalen Linien variabler Breite, die zusätzlichen freien Raum daneben ermöglichen.
\begin{MacroCode}{class}
\DeclareDocumentCommand \questtextblank { O{0.75cm} m m } {%
  \par\vspace{1ex}
  \prg_replicate:nn {#2} {
    \par
    \vspace{-\baselineskip}
    \parskipreduce
    \vspace{#1}
    \noindent\rule{#3}{0.4pt}
  }
}

\end{MacroCode}
\end{macro}



\begin{macro}{\questtext}[2]{[<lineskip>]}{<number lines>}
Erzeugt eine neue Frage unter Berücksichtigung der relevanten Optionen mit anschließenden horizontalen Linien.
\begin{MacroCode}{class}
\DeclareDocumentCommand \questtext { O{0.75cm} m } {%
  \questtextblank[#1]{#2}{\linewidth}
}

\end{MacroCode}
\end{macro}


\begin{environment}{questmclist}[1]{[<itemsep>]}
Erzeugt eine neue Frage unter Berücksichtigung der relevanten Optionen mit anschließender Multiple-Choice-Aufzählung.
\begin{MacroCode}{class}
\DeclareDocumentEnvironment {questmclist} { O{0.5\baselineskip} } {%
  \parskipreduce%
  \vspace{-0.15\baselineskip}%
  \begin{edulist}[#1]{\color{\g_edu_questmclabelfg_tl}$\square$}
}{%
  \end{edulist}
  \parskipreduce
}

\end{MacroCode}
\end{environment}



\begin{environment}{questmclistalph}[1]{[<itemsep>]}
\begin{MacroCode}{class}
\DeclareDocumentEnvironment {questmclistalph} { O{0.5\baselineskip} } {%
  \parskipreduce%
  \vspace{-0.25\baselineskip}%
  \begin{edulist}[#1]{\g__edu_questmclabelboxed_tl}
}{%
  \end{edulist}%
  \parskipreduce
}

\end{MacroCode}
\end{environment}


\begin{environment}{questmcarray}[2]{[<extrarowheight>]}{<number columns>}
\begin{environment}{questmcarray*}[2]{[<extrarowheight>]}{<number columns>}
Erzeugt eine neue Frage unter Berücksichtigung der relevanten Optionen mit anschließendem Multiple-Choice-Array. Die Sternvariante versetzt die Zellen in den Mathematik-Modus
\begin{MacroCode}{class}
\DeclareDocumentEnvironment {questmcarray} { s O{0.25\baselineskip} m } {%
  \IfBooleanTF {#1} {
    \arraysetup[#2]{m}
    \noindent\tabularx{\textwidth}{*{#3}{Q}}%
  }{
    \arraysetup[#2]{}
    \noindent\tabularx{\textwidth}{*{#3}{q}}%
  }
}{%
  \endtabularx
  \arraycleanup
}

% Hack, because xparse doesn't generate starred environments:
\cs_new:cpn {questmcarray*} {\questmcarray*}
\cs_new_eq:cN {endquestmcarray*} \endquestmcarray

\end{MacroCode}
\end{environment}
\end{environment}


\begin{environment}{questmcarrayalph}[2]{[<extrarowheight>]}{<number columns>}
\begin{environment}{questmcarrayalph*}[2]{[<extrarowheight>]}{<number columns>}
Erzeugt eine neue Frage unter Berücksichtigung der relevanten Optionen mit anschließendem Multiple-Choice-Array in alphabetischer Nummerierung. Die Sternvariante versetzt die Zellen in den Mathematik-Modus
\begin{MacroCode}{class}
\DeclareDocumentEnvironment {questmcarrayalph} { s O{0.25\baselineskip} m } {%
  \IfBooleanTF {#1} {
    \arraysetup[#2]{m}
    \noindent\tabularx{\textwidth}{*{#3}{A}}%
  }{
    \arraysetup[#2]{}
    \noindent\tabularx{\textwidth}{*{#3}{a}}%
  }
}{%
  \endtabularx
  \arraycleanup
}

% Hack, because xparse doesn't generate starred environments:
\cs_new:cpn {questmcarrayalph*} {\questmcarrayalph*}
\cs_new_eq:cN {endquestmcarrayalph*} \endquestmcarrayalph

\end{MacroCode}
\end{environment}
\end{environment}


\subsection{Mehrspaltiges Layout}

\begin{macro*}{\g__edu_colone_dim}
\begin{macro*}{\g__edu_coltwo_dim}
\begin{macro*}{\g__edu_colalign_tl}
Zuerst werden benötigte Längen definiert und initialisiert:
\begin{MacroCode}{class}
\dim_new:N \g__edu_colone_dim
\dim_new:N \g__edu_coltwo_dim

\tl_new:N \g__edu_colalign_tl
\tl_gset:Nn \g__edu_colalign_tl {t}

\end{MacroCode}
\end{macro*}
\end{macro*}
\end{macro*}


Der Abstand zwischen den Spalten und dem umgebenden Fließtext entspricht 
dem Absatzabstand:
\begin{MacroCode}{class}
\setlength{\multicolsep}{0.5\baselineskip}

\end{MacroCode}


\begin{environment}{multicols2r}
Abkürzendes Makro für gleichmäßiges, zweispaltiges Layout ohne bündigen unteren Rand über das Package \pkg{multicol}.
\begin{MacroCode}{class}
\DeclareDocumentEnvironment {multicols2r} { } {%
  \begin{multicols}{2}\raggedcolumns
}{%
  \end{multicols}%
}

\end{MacroCode}
\end{environment}


\begin{environment}{multicols3r}
Abkürzendes Makro für gleichmäßiges, dreispaltiges Layout ohne bündigen unteren Rand über das Package \pkg{multicol}.
\begin{MacroCode}{class}
\DeclareDocumentEnvironment {multicols3r} { } {%
  \begin{multicols}{3}\raggedcolumns
}{%
  \end{multicols}%
}

\end{MacroCode}
\end{environment}


\begin{environment}{cols2}[1]{[<width of left column>]}
\begin{environment}{cols2*}[1]{[<width of left column>]}
Abkürzendes Makro für zweispaltiges Layout. Das optionale Argument bestimmt die Breite der linken Spalte. Die Sternvariante zentriert die beiden Spalten vertikal.

Die Sternvariante wird explizit deklariert, da der verwendete Hack hier nicht funktioniert. Evtl. wegen der Zahl im Namen des Environments? Bei \env{cols} funktioniert es.
\begin{MacroCode}{class}
\DeclareDocumentEnvironment {cols2} { O{0.5\linewidth} } {%
  \par
  \setlength{\parfillskip}{0em}  
  \dim_gset:Nn \g__edu_colone_dim {#1 - 0.5\columnsep}
  \dim_gset:Nn \g__edu_coltwo_dim {\linewidth - \g__edu_colone_dim - \columnsep}
%
  \parskipreduce
  \noindent\begin{minipage}[\g__edu_colalign_tl]{\dim_use:N\g__edu_colone_dim}%
  \setpar
  \parskipreduce
}{%
  \end{minipage}%
  \par
  \setlength{\parfillskip}{1em plus 1fil}
  \vspace{1.5\multicolsep}
  \parskipreduce
}

% Declared explicit, because hack doesn't work here. Because of the number?
\DeclareDocumentEnvironment {cols2*} { O{0.5\linewidth} } {%
  \tl_gset:Nn \g__edu_colalign_tl {c}
  \vspace{1.5\multicolsep} % Erhöht, da bei [t] minipage vorher keinen normalen baselineskip hat.
  \begin{cols2}[#1]
}{%
  \end{cols2}%
  \tl_gset:Nn \g__edu_colalign_tl {t}
}

\end{MacroCode}
\end{environment}
\end{environment}


\begin{environment}{cols}[1]{[<number columns>]}
\begin{environment}{cols*}[1]{[<width of left column>]}
Abkürzendes Makro für mehrspaltiges Layout. Die Sternvariante zentriert die beiden Spalten vertikal. Das optionale Argument gibt die Anzahl der Spalten an. Standardwert ist hierbei 2. D. h. \Macro\begin{cols} entspricht \Macro\begin{cols}[2] und \Macro\begin{cols2}.
\begin{MacroCode}{class}
\DeclareDocumentEnvironment {cols} {s O{2} } {%
  \par
  \setlength{\parfillskip}{0em}
%
  % Bei der Berechnung wichtig: Bei Multiplikation einer Länge mit einer
  % Ganzzahl muss zuerst die Länge (hier \columnsep) angegeben werden.
  \dim_gset:Nn \g__edu_colone_dim {(\linewidth - \columnsep * (#2 - 1))/#2}
  \dim_gset_eq:NN \g__edu_coltwo_dim \g__edu_colone_dim
%
  \parskipreduce
  \vspace{-0.15\baselineskip}
  \IfBooleanT {#1} {
    \tl_gset:Nn \g__edu_colalign_tl {c}
    \vspace{1.2\multicolsep} % Erhöht, da bei [t] minipage vorher keinen normalen baselineskip hat.
  }
  \noindent\begin{minipage}[\g__edu_colalign_tl]{\g__edu_colone_dim}%
  \setpar
  \parskipreduce
}{%
  \end{minipage}%
  \par
  \setlength{\parfillskip}{1em plus 1fil}
  \parskipreduce
  \vspace{0.75\baselineskip}
  \IfBooleanT {#1} {
    \tl_gset:Nn \g__edu_colalign_tl {t}
  }  
}

% Hack, because xparse doesn't generate starred environments:
\cs_new:cpn {cols*} {\cols*}
\cs_new:cpn {endcols*} {\endcols}

\end{MacroCode}
\end{environment}
\end{environment}


\begin{macro}{\colbreak}
Durch \cs{colbreak} wird die erste \env*{minipage} beendet und in die zweite \env*{minipage} gewechselt.
\begin{MacroCode}{class}
\DeclareDocumentCommand \colbreak { } {%
  \end{minipage}\hfill%
  \begin{minipage}[\g__edu_colalign_tl]{\g__edu_coltwo_dim}%
  \setpar
  \parskipreduce
}

\end{MacroCode}
\end{macro}


\begin{macro}{\mathreduce}
Verwendet man zu Beginn einer Spalte eine abgesetzte Gleichung,
wird ein fehlerhafter Abstand vor der Gleichung erzeugt. Dieser 
kann mit diesem Befehl beseitigt werden.
\begin{MacroCode}{class}
\DeclareDocumentCommand \mathreduce { } {%
  \vspace{-\abovedisplayskip}
  \vspace{-0.5\baselineskip}
}

\end{MacroCode}
\end{macro}


\begin{macro}{\g__edu_aligntemp_tl}
\begin{environment}{graphicscol}[4]{[<width of graphicscolumn>]}{<file>}{[<options of \cs*{includegraphics}>]}{[<l|r|c>]}
\begin{environment}{graphicscol*}[4]{[<width of graphicscolumn>]}{<file>}{[<options of \cs*{includegraphics}>]}{[<l|r|c>]}
Das folgende Makro erzeugt ein zweispaltiges Layout, wobei die rechte 
Spalte (in der Sternversion die linke Spalte) eine beliebige Abbildung 
linksbündig (\texttt{c}), rechtsbündig (\texttt{r}) oder zentriert (\texttt{c}) darstellt.
Die Breite der rechten Spalte und Optionen für das verwendete \cs*{includegraphics} können angegeben werden.
\begin{MacroCode}{class}

\tl_new:N \g__edu_aligntemp_tl

\DeclareDocumentEnvironment {graphicscol} { s O{0.5\linewidth} m O{width=\linewidth} o } {%
  
  \IfValueTF {#5} {
    \tl_gset:Nn \g__edu_aligntemp_tl {#5}
  }{
	  \IfBooleanTF {#1} {
	    \tl_gset:Nn \g__edu_aligntemp_tl {l}
	  }{
	    \tl_gset:Nn \g__edu_aligntemp_tl {r}
	  }
  }
    
  \begin{cols2*}[#2]
  \IfBooleanT {#1} {
    \tl_if_in:NnT \g__edu_aligntemp_tl {r} {
      \hfill
    }
    \tl_if_in:NnT \g__edu_aligntemp_tl {c} {
  	  \begin{center}
	  }
	    \includegraphics[#4]{#3}
    \tl_if_in:NnT \g__edu_aligntemp_tl {c} {
	    \end{center}
	  }
    \tl_if_in:NnT \g__edu_aligntemp_tl {l} {
      \hfill
    }
    \colbreak
  }
}{%
  \IfBooleanF {#1} {
    \colbreak
	  \tl_if_in:NnT \g__edu_aligntemp_tl {r} {
	    \setlength{\parfillskip}{0em}
      \hfill
    }
    \tl_if_in:NnT \g__edu_aligntemp_tl {c} {
  	  \begin{center}
	  }
	    \includegraphics[#4]{#3}
    \tl_if_in:NnT \g__edu_aligntemp_tl {c} {
	    \end{center}
	  }
    \tl_if_in:NnT \g__edu_aligntemp_tl {l} {
      \hfill
    }
  }
  \end{cols2*}
}

% Hack, because xparse doesn't generate starred environments:
\cs_new:cpn {graphicscol*} {\graphicscol*}
\cs_new:cpn {endgraphicscol*} {\endgraphicscol}

\end{MacroCode}
\end{environment}
\end{environment}
\end{macro}


\begin{environment}{tikzcol}[3]{[<width of tikz-column>]}{<file>}{[<l|r|c>]}
\begin{environment}{tikzcol*}[3]{[<width of tikz-column>]}{<file>}{[<l|r|c>]}
Das folgende Makro erzeugt ein zweispaltiges Layout, wobei die rechte 
Spalte (in der Sternversion die linke Spalte) eine beliebige \pkg{tikz}-Grafik 
zentriert darstellt. Diese Grafik muss in einer PGF-Datei im vorgegebene \opt{tikzpath} vorliegen und wird mittels \cs{tikzinput*} eingebunden. Die Grafik wird wahlweise linksbündig (\texttt{c}), rechtsbündig (\texttt{r}) oder zentriert (\texttt{c}) gesetzt.
\begin{MacroCode}{class}

\DeclareDocumentEnvironment {tikzcol} { s O{0.5\linewidth} m o } {%
  
  \IfValueTF {#4} {
    \tl_gset:Nn \g__edu_aligntemp_tl {#4}
  }{
	  \IfBooleanTF {#1} {
	    \tl_gset:Nn \g__edu_aligntemp_tl {l}
	  }{
	    \tl_gset:Nn \g__edu_aligntemp_tl {r}
	  }
  }
    
  \begin{cols2*}[#2]
  \IfBooleanT {#1} {
    \tl_if_in:NnT \g__edu_aligntemp_tl {r} {
      \hfill
    }
    \tl_if_in:NnT \g__edu_aligntemp_tl {c} {
  	  \begin{center}
	  }
	    \tikzinput{#3}
    \tl_if_in:NnT \g__edu_aligntemp_tl {c} {
	    \end{center}
	  }
    \tl_if_in:NnT \g__edu_aligntemp_tl {l} {
      \hfill
    }
    \colbreak
  }
}{%
  \IfBooleanF {#1} {
    \colbreak
	  \tl_if_in:NnT \g__edu_aligntemp_tl {r} {
      \hfill
    }
    \tl_if_in:NnT \g__edu_aligntemp_tl {c} {
  	  \begin{center}
	  }
	    \tikzinput{#3}
    \tl_if_in:NnT \g__edu_aligntemp_tl {c} {
	    \end{center}
	  }
    \tl_if_in:NnT \g__edu_aligntemp_tl {l} {
      \hfill
    }
  }
  \end{cols2*}
}

% Hack, because xparse doesn't generate starred environments:
\cs_new:cpn {tikzcol*} {\tikzcol*}
\cs_new:cpn {endtikzcol*} {\endtikzcol}

\end{MacroCode}
\end{environment}
\end{environment}


\subsection{Notizen}

\begin{macro}{\notet}[1]{<text>}
Textnotizen:
\begin{MacroCode}{class}
\DeclareDocumentCommand \notet { m } {%
  \bool_if:NT \g_edu_shownotes_bool {
    {\g_edu_notetstyle_tl\textcolor{\g_edu_notetfg_tl}{#1}}\xspace%
  }
}
\end{MacroCode}
\end{macro}


\begin{macro}{\notehr}
Horizontale Linie zur Markierung.
\begin{MacroCode}{class}
\DeclareDocumentCommand \notehr { } {%
  \bool_if:NT \g_edu_shownotes_bool {
    \par
    {%
      \color{\g_edu_notehrfg_tl}
      \rule[0.75\baselineskip]{\linewidth}{\g_edu_notehrule_dim}%
      \vspace{-\baselineskip}%
    }%
  }%
}

\end{MacroCode}
\end{macro}


\subsection{Unterrichtsablauf}

\subsubsection{Hilfskommandos}

Zuerst einige Zähler:
\begin{MacroCode}{class}
\int_new:N \g__edu_ttminutesum_int
\int_gset:Nn \g__edu_ttminutesum_int {0}

\int_new:N \g__edu_tthour_int
\int_new:N \g__edu_ttminute_int

\end{MacroCode}


\begin{macro}{\edu_ttaddhours:n}[1]{<hours>}
\begin{macro}{\edu_ttaddminutes:n}[1]{<minutes>}
Die Uhrzeit zur Beginn der jeweiligen Phase soll berechnet werden.
 Die aktuelle Stunde und Minute wird in den Countern \cs{__edu_tthour} und \cs{__edu_tthour} gespeichert. Die folgenden Kommandos erlauben das Addieren von Stunden und Minuten zur aktuellen Zeit. Es sind jedoch nur Stundenwerte kleiner 24 und Minutenwerte kleiner 60 zulässig.
\begin{MacroCode}{class}
\cs_new:Npn \edu_ttaddhours:n #1 {
  \int_gadd:Nn \g__edu_tthour_int {#1}
  \int_compare:nNnT {\g__edu_tthour_int} > {23} {
    \int_gsub:Nn \g__edu_tthour_int {24}
  }
}

\cs_new:Npn \edu_ttaddminutes:n #1 {
  \int_gadd:Nn \g__edu_ttminute_int {#1}
  \int_compare:nNnT {\g__edu_ttminute_int} > {59} {
    \int_gadd:Nn \g__edu_tthour_int {1}
    \int_gsub:Nn \g__edu_ttminute_int {60}
  }
  \int_compare:nNnT {\g__edu_tthour_int} > {23} {
    \int_gsub:Nn \g__edu_tthour_int {24}
  }
}

\end{MacroCode}
\end{macro}
\end{macro}


\begin{macro}{\edu_ttentrytime:}
Mit diesem Makro kann die aktuelle Uhrzeit im Format hh:mm ausgegeben werden. Als Einheit wird die Option \opt{ttentrytimelabel} verwendet.
\begin{MacroCode}{class}
\cs_new:Npn \edu_ttentrytime: {  
  \int_compare:nNnTF {\g__edu_tthour_int} < {10} {
    0\int_to_arabic:n {\g__edu_tthour_int}
  }{
    \int_to_arabic:n {\g__edu_tthour_int}
  }
  :
  \int_compare:nNnTF {\g__edu_ttminute_int} < {10} {
    0\int_to_arabic:n {\g__edu_ttminute_int}
  }{
    \int_to_arabic:n {\g__edu_ttminute_int}
  }
  \tl_if_empty:NF \__edu_ttentryltimelabel {
    \,\g_edu_ttentrytimelabel_tl
  }
}

\end{MacroCode}
\end{macro}


\subsubsection{Tabelle}

Tabelle zur Planung des Unterrichtsablaufs.

\begin{macro}{\g__edu_ttable_tl}
Die Zeilen der Tabelle werden als \cs{__edu_ttable} zusammengesetzt.
\begin{MacroCode}{class}
\tl_new:N \g__edu_ttable_tl
\end{MacroCode}
\end{macro}

\begin{macro}{\ttstart}[2]{<begin hour>}{<begin minute>}
Dieser Befehl inizialisiert die Zeit-Variablen. Als Argumente müssen die aktuelle Stunde und Minuten angegeben werden, z.\,B. \Macro\begin{ttable}{9}{45} für 9:45\,Uhr.
\begin{MacroCode}{class}
\DeclareDocumentCommand \ttstart { m m } {
  \int_gset:Nn \g__edu_ttminutesum_int {0}
  \int_gset:Nn \g__edu_tthour_int {#1}
  \int_gset:Nn \g__edu_ttminute_int {#2}
  
  \tl_clear:N \g__edu_ttable_tl
}


\end{MacroCode}
\end{macro}


\begin{macro}{ttable}
\begin{macro}{ttable*}[1]{[<text before>]}
Erzeugt die Tabelle für den Unterrichtsablauf unter Verwendung diverser Optionen.

Die Sternvariante setzt die Tabelle im Querformat. Die Breiten der Spalten werden über eigene Optionen angegeben. Ein zusätzlicher, optionaler Parameter kann in der Sternvariante verwendet werden, um Text (z.\,B. eine Überschrift) über der Tabelle zu positionieren.
\begin{MacroCode}{class}
\DeclareDocumentCommand \ttable { s o } {%
  \IfBooleanTF {#1} {
    \landscape
    \IfValueT {#2} {
      #2
    }
    \begin{tabularx}{\linewidth}{%
      p{\g_edu_tttimewidthlscape_dim}%
      p{\g_edu_ttstagewidthlscape_dim}%
      X%
      p{\g_edu_ttmethodwidthlscape_dim}%
      p{\g_edu_ttmediawidthlscape_dim}%
    }
  }{
	  \begin{tabularx}{\linewidth}{%
	    p{\g_edu_tttimewidth_dim}%
	    p{\g_edu_ttstagewidth_dim}%
	    X%
	    p{\g_edu_ttmethodwidth_dim}%
	    p{\g_edu_ttmediawidth_dim}%
	  }
  }
  \bool_if:NTF \g_edu_ttendtime_bool {
    \toprule
  }{
    \midrule
  }
  \small\textsfbf{\g_edu_tttimelabel_tl} & 
  \small\textsfbf{\g_edu_ttstagelabel_tl} & 
  \small\textsfbf{\g_edu_ttactivitylabel_tl} &  
  \small\textsfbf{\g_edu_ttmethodlabel_tl} & 
  \small\textsfbf{\g_edu_ttmedialabel_tl} \\ 
  
  \bool_if:NTF \g_edu_ttendtime_bool {
    \headrule
  }{
    \midrule
  }
  
  \tl_use:N \g__edu_ttable_tl
  
  \bool_if:NT \g_edu_ttendtime_bool {
    \edu_ttentrytime: \\ \bottomrule
  }  

  \end{tabularx}
  \IfBooleanT {#1} {
    \endlandscape
  }
}

\end{MacroCode}
\end{macro}
\end{macro}


\begin{macro}{\ttentry}[5]{[<duration>]}{<stage>}{<activity>}{<methods>}{<media>}
\begin{macro}{\ttentry*}[5]{[<duration>]}{<stage>}{<activity>}{<methods>}{<media>}
Durch \cs{ttentry} wird eine Zeile des Unterrichtsablaufes angegeben. Die Parameter entsprechen den Spalten der Tabelle. Der erste Parameter -- die Dauer der Phase -- ist optional. Wird sie nicht angegeben, entfällt die Anzeige der Dauer/Uhrzeit.

Die Sternvariante setzt die Zelle direkt in eine Aufzählung vom Typ \env*{itemizet}.
\begin{MacroCode}{class}
\DeclareDocumentCommand \ttentry { s O{} m +m +m +m } {%
  \bool_if:nT {!\tl_if_empty_p:n {#2} && \bool_if_p:N \g_edu_ttshowtime_bool} {
    \tl_gput_right:Nx \g__edu_ttable_tl {\edu_ttentrytime:}
    \tl_gput_right:Nn \g__edu_ttable_tl {\newline}
    
    \edu_ttaddminutes:n {#2}
    \int_gadd:Nn \g__edu_ttminutesum_int {#2}
    
    \tl_gput_right:Nn \g__edu_ttable_tl {#2\textbar}
    \tl_gput_right:NV \g__edu_ttable_tl {\g__edu_ttminutesum_int}
  }%
  \tl_gput_right:Nn \g__edu_ttable_tl { & #3 & }
  \IfBooleanT {#1} {
    \tl_gput_right:Nn \g__edu_ttable_tl { \begin{itemizet} }
  }
  \tl_gput_right:Nn \g__edu_ttable_tl { #4 }
  \IfBooleanT {#1} {
    \tl_gput_right:Nn \g__edu_ttable_tl { \end{itemizet} }
  }
  \tl_gput_right:Nn \g__edu_ttable_tl {& #5 & #6 \tabularnewline \midrule
  }
}

\end{MacroCode}
\end{macro}
\end{macro}


\subsubsection{Ablaufliste}


\begin{macro}{\g_edu_teachera_tl}
\begin{macro}{\g_edu_teachers_tl}
\begin{macro}{\g_edu_studenta_tl}
\begin{macro}{\g_edu_students_tl}
Zuerst werden Makros erstellt, welche die Label erzeugen. Für Lehrer und Schüler, Handlungen und Gesprochenes:
\begin{MacroCode}{class}
\tl_new:N \g_edu_teachera_tl
\tl_new:N \g_edu_teachers_tl
\tl_new:N \g_edu_studenta_tl
\tl_new:N \g_edu_students_tl

\tl_gset:Nn \g_edu_teachera_tl {%
  \makebox[2em][l]{{\g_edu_seqteacherstyle_tl\color{\g_edu_seqteacherfg_tl} \g_edu_seqteacherlabel_tl} \action}%
}

\tl_gset:Nn \g_edu_teachers_tl {%
  \makebox[2em][l]{{\g_edu_seqteacherstyle_tl\color{\g_edu_seqteacherfg_tl} \g_edu_seqteacherlabel_tl} \speech}%
}

\tl_gset:Nn \g_edu_studenta_tl {%
  \makebox[2em][l]{{\g_edu_seqetudentstyle_tl\color{\g_edu_seqstudentfg_tl} \g_edu_seqstudentlabel_tl} \action}%
}

\tl_gset:Nn \g_edu_students_tl {%
  \makebox[2em][l]{{\g_edu_seqetudentstyle_tl\color{\g_edu_seqstudentfg_tl} \g_edu_seqstudentlabel_tl} \speech}%
}

\end{MacroCode}
\end{macro}
\end{macro}
\end{macro}
\end{macro}



\begin{macro}{\itemta}
\begin{macro}{\itemts}
\begin{macro}{\itemsa}
\begin{macro}{\itemss}
Damit diese analog zu \cs{item} verwendet werden können, müssen sie nun noch in ein solches gebettet werden. Dies kann nicht in einem Schritt geschehen (Warum?).
\begin{MacroCode}{class}
\DeclareDocumentCommand \itemta { } {\item[\g_edu_teachera_tl]}
\DeclareDocumentCommand \itemts { } {\item[\g_edu_teachers_tl]}
\DeclareDocumentCommand \itemsa { } {\item[\g_edu_studenta_tl]}
\DeclareDocumentCommand \itemss { } {\item[\g_edu_students_tl]}

\end{MacroCode}
\end{macro}
\end{macro}
\end{macro}
\end{macro}


\begin{environment}{sequence}
\begin{environment}{sequencet}
Abschließend wird eine angepasste Auflistung \env{sequence} definiert. Die Variante \env{sequencet} ist analog zu \env{itemizet} zur Verwendung in einer Ablauftabelle geeignet.
\begin{MacroCode}{class}
\DeclareDocumentEnvironment {sequence} { } {%
  \begin{itemize}[labelwidth=2em, leftmargin= 2em + \labelsep + \labelsep]
}{%
  \end{itemize}
}

\DeclareDocumentEnvironment {sequencet} { } {%
  \@minipagetrue%
  \setpar
  \begin{itemize}[labelwidth=2em, leftmargin= 2em + \labelsep + 0.1\labelsep, nosep]
}{%
  \vspace{-\baselineskip}
  \end{itemize}
}

\end{MacroCode}
\end{environment}
\end{environment}


\subsection{Tafelbild}

\begin{macro}{\g_edu_setbblengths_tl}
Die folgenden Längen werden für das Tafelbild angepasst.
\begin{MacroCode}{class}
\tl_new:N \g_edu_setbblengths_tl
\tl_gset:Nn \g_edu_setbblengths_tl {%
  \setlength{\columnsep}{0.5em}%
  \setlength{\jot}{0pt}%
  \setlength{\abovedisplayskip}{0.4ex}%
  \setlength{\belowdisplayskip}{0.4ex}%
}

\end{MacroCode}
\end{macro}


\begin{macro*}{\g__edu_bbbaselineskip_dim}
\begin{environment}{bbpart}[2]{[<vertical alignment>]}{<width>}
\begin{environment}{bbpart*}[2]{[<vertical alignment>]}{<width>}
Nun wird die \env{bbpart}-Umgebung definiert. Mit ihr ist es möglich, 
den Teil einer Tafel (als rechteckige Minipage) zu setzen. Das 
optionale Argument gibt die vertikale Zentrierung an, das benötigte 
Argument die Breite des Rechtecks. Die Höhe entspricht einem Viertel 
der Seitenbreite.

In der Sternvariante wird der Text horizontal zentriert.

Wichtig: Jede Zeile mit einem \% beenden, sonst entsteht ein ungewünschter Abstand zwischen den Tafelteilen.

Wichtig: Da die Definitionen von \cs*{NewEnviron} nicht geschachtelt werden 
können, müssen alle Varianten der Umgebung einzeln definiert werden. Dies 
führt leider zu Redundanzen und muss beim Verändern berücksichtigt werden.
\begin{MacroCode}{class}
\dim_new:N \g__edu_bbbaselineskip_dim
\dim_gset:Nn \g__edu_bbbaselineskip_dim {\g_edu_bbfontsize_dim + \g_edu_bbbaselineoffset_dim}

\newsavebox{\bbbox}

\DeclareDocumentEnvironment {bbpart} { s O{t} m } {
  \g_edu_setbblengths_tl%
  \hspace{\fboxrule}%
  \begin{lrbox}{\bbbox}
    \begin{minipage}[c][\g_edu_bbheight_dim][#2]{#3 - 2\fboxsep - 2\fboxrule}%
      \g_edu_bbstyle_tl%
      \fontsize{\g_edu_bbfontsize_dim}{\g__edu_bbbaselineskip_dim}\selectfont%
      \setpar%
      \IfBooleanT {#1} {
        \begin{center}
      }
}{
      \mbox{}  % Damit Tafel auch ohne Inhalt angezeigt wird.
      \IfBooleanT {#1} {
        \end{center}
      }
    \end{minipage}%
  \end{lrbox}
  \fbox{\usebox{\bbbox}}
  \hspace{-3\fboxrule}%
  \hspace{-1.9pt} % Removes strange space between minipages.
}

% Hack, because xparse doesn't generate starred environments:
\cs_new:cpn {bbpart*} {\bbpart*}
\cs_new_eq:cN {endbbpart*} \endbbpart

\end{MacroCode}
\end{environment}
\end{environment}
\end{macro*}


\begin{environment}{bbhalf}[1]{[<vertical alignment>]}
\begin{environment}{bbhalf*}[1]{[<vertical alignment>]}
Analog zu oben, jedoch mit fester Breite von einem Viertel der Seitenbreite.
Entspricht einer Klapptafel.
\begin{MacroCode}{class}
\DeclareDocumentEnvironment {bbhalf} { s O{t} } {
  \IfBooleanTF {#1} {
    \begin{bbpart*}[#2]{0.25\linewidth}%
  }{
    \begin{bbpart}[#2]{0.25\linewidth}%
  }
}{%
  \IfBooleanTF {#1} {
    \end{bbpart*}
  }{
    \end{bbpart}
  }
}

% Hack, because xparse doesn't generate starred environments:
\cs_new:cpn {bbhalf*} {\bbhalf*}
\cs_new_eq:cN {endbbhalf*} \endbbhalf

\end{MacroCode}
\end{environment}
\end{environment}


\begin{environment}{bbfull}[1]{[<vertical alignment>]}
\begin{environment}{bbfull*}[1]{[<vertical alignment>]}
Analog zu oben, jedoch mit fester Breite von der Hälfte der Seitenbreite.
Entspricht der zentralen, großen Tafelfläche.
\begin{MacroCode}{class}
\DeclareDocumentEnvironment {bbfull} { s O{t} } {
  \IfBooleanTF {#1} {
    \begin{bbpart*}[#2]{0.5\linewidth}%
  }{
    \begin{bbpart}[#2]{0.5\linewidth}%
  }
}{%
  \IfBooleanTF {#1} {
    \end{bbpart*}
  }{
    \end{bbpart}
  }
}

% Hack, because xparse doesn't generate starred environments:
\cs_new:cpn {bbfull*} {\bbfull*}
\cs_new_eq:cN {endbbfull*} \endbbfull

\end{MacroCode}
\end{environment}
\end{environment}

\begin{macro}{bbsection}[1]{<name>}
\begin{macro}{bbsection*}[1]{<name>}
\begin{macro}{bbsubsection}[1]{<name>}
\begin{macro}{bbsubsection*}[1]{<name>}
\begin{macro}{bbsubsubsection}[1]{<name>}
\begin{macro}{bbsubsubsection*}[1]{<name>}
Macro für zentrierte Überschriften der obersten Gliederungsebene auf Tafelbildern.
\begin{MacroCode}{class}
\DeclareDocumentCommand \bbsection { s m } {%
  \group_begin:
  \IfBooleanT {#1} {
    \begin{center}
  }
    {\g_edu_bbsectionstyle_tl #2}
  \IfBooleanT {#1} {
    \end{center}
  }
  \group_end:
}

\DeclareDocumentCommand \bbsubsection { s m } {%
  \group_begin:
  \IfBooleanT {#1} {
    \begin{center}
  }
    {\g_edu_bbsubsectionstyle_tl #2}
  \IfBooleanT {#1} {
    \end{center}
  }
  \group_end:
}

\DeclareDocumentCommand \bbsubsubsection { s m } {%
  \group_begin:
  \IfBooleanT {#1} {
    \begin{center}
  }
    {\g_edu_bbsubsubsectionstyle_tl #2}
  \IfBooleanT {#1} {
    \end{center}
  }
  \group_end:
}

\end{MacroCode}
\end{macro}
\end{macro}
\end{macro}
\end{macro}
\end{macro}
\end{macro}


\subsection{Mathematik}

\subsubsection{Längen anpassen}

Zum Einsparen von Platz, werden Abstände vor und nach Gleichungen 
verkleinert.

\begin{MacroCode}{class}
\AtBeginDocument{
  \setlength{\abovedisplayskip}{1.2ex plus 0.2ex minus 0.1ex}
  \setlength{\abovedisplayshortskip}{1ex plus 0.2ex minus 0.2ex}
  \setlength{\belowdisplayskip}{1.2ex plus 0.2ex minus 0.1ex}
  \setlength{\belowdisplayshortskip}{1ex plus 0.2ex minus 0.2ex}
}

\end{MacroCode}


\subsubsection{Komma als Dezimaltrenner}

Abhängig von der Option \opt{commasep} wird mithilfe des Packages \pkg{icomma} das Komma als Dezimaltrenner verwendet.

\begin{MacroCode}{class}
\bool_if:NT \g_edu_commasep_bool {
  \RequirePackage{icomma}
}

\end{MacroCode}


\subsubsection{Gleichungsumgebungen}

\begin{environment}{aligntr}
\begin{environment}{aligntr*}
Gleichungsumgebung zum Setzen von Äquivalenzumformungen 
(Transformations). Verwendet intern \env*{alignat}. Als Trenner für 
Umformungen sollte \cs{tr} verwendet werden. Die Sternvariante erzeugt 
Gleichungen ohne Nummerierung.
\begin{MacroCode}{class}

\DeclareDocumentEnvironment {aligntr} { s } {
  \IfBooleanTF {#1} {
    \csname alignat*\endcsname{2}
  }{
    \alignat{2}
  }
}{
  \IfBooleanTF {#1} {
  \csname endalignat*\endcsname
  }{
  \endalignat
  }
}

% Hack, because xparse doesn't generate starred environments:
\cs_new:cpn {aligntr*} {\aligntr*}
\cs_new_eq:cN {endaligntr*} \endaligntr

\end{MacroCode}
\end{environment}
\end{environment}

\begin{macro}{\tr}
Dient als Trenner für Umformungen in \env*{aligntr}.
\begin{MacroCode}{class}
\DeclareDocumentCommand \tr { } {&& \mid}

\end{MacroCode}
\end{macro}


\subsubsection{Rechenbäume}

Rechenbäume können mithilfe des Packages \textsf{tikz-qtree} erstellt werden. Dieses wir hier geladen und konfiguriert.

\begin{MacroCode}{class}
\tl_if_eq:VnT \g_edu_struktexqtree_tl {tikz-qtree} {
	\tikzset{
	  edge~from~parent/.style={
	    draw,
	    edge~from~parent~path={
	        (\tikzparentnode.north) -| (\tikzchildnode)
	    }
	  }
	}
	
	\tikzset{
	  bet/.style = {
	    grow'=up, 
	    every~internal~node/.style={
	        draw,
	        circle,
	        inner~sep=0pt,
	        minimum~size=2ex
	    }
	  },
	  betr/.style={
	    draw=none,
	    inner~sep=0pt
	  },
	  bete/.style={
	    draw,
	    rectangle,
	    inner~sep=2pt
	  }
	}
	
	\tikzset{
	  level~distance=3.5ex, 
	  sibling~distance=2ex, 
	  frontier/.style={distance from root=10.5ex}
	}
}

\end{MacroCode}




\subsubsection{Theorem-Umgebungen}

First, check option-dependencies:

\begin{MacroCode}{class}
\bool_if:nT {
  \bool_if_p:N \g_edu_thmbox_bool && 
  (!\bool_if_p:N \g_edu_thmimpnumbered_bool || !\bool_if_p:N \g_edu_thmunimpnumbered_bool)
} {
  \msg_error:nnnn {edu} {option-dep-enable} {`thmbox'} {`thmimpnumbered', `thmunimpnumbered'}
}

\end{MacroCode}


\begin{macro}{\g__edu_thmafterskip_skip}
\begin{macro}{\g__edu_thmbeforeskip_skip}
\begin{macro}{\g__edu_thmimp_seq}
\begin{macro}{\g__edu_thmunimp_seq}
Zuerst werden benötigte Längen definiert. After that, sequences with the different names of the theorems are defined.
\begin{MacroCode}{class}
\skip_new:N \g__edu_thmafterskip_skip
\skip_new:N \g__edu_thmbeforeskip_skip

\skip_gset:Nn \g__edu_thmafterskip_skip {0.5\baselineskip-0.5\parskip}
\skip_gset:Nn \g__edu_thmbeforeskip_skip {0.5\baselineskip}

\seq_new:N \g__edu_thmimp_seq

\seq_gput_right:Nn \g__edu_thmimp_seq {theorem}
\seq_gput_right:Nn \g__edu_thmimp_seq {definition}
\seq_gput_right:Nn \g__edu_thmimp_seq {defitheo}

\seq_new:N \g__edu_thmunimp_seq

\seq_gput_right:Nn \g__edu_thmunimp_seq {example}
\seq_gput_right:Nn \g__edu_thmunimp_seq {exampleexe}
\seq_gput_right:Nn \g__edu_thmunimp_seq {hint}
\seq_gput_right:Nn \g__edu_thmunimp_seq {remark}
\seq_gput_right:Nn \g__edu_thmunimp_seq {solution}

\end{MacroCode}
\end{macro}
\end{macro}
\end{macro}
\end{macro}

Nun werden die Theoremstyles definiert. For this purpose, two control sequences \cs{__edu_theoremstyleimp:n} and \cs{__edu_theoremstyleimp:n} are defined, which takes an argument for numbered theorems (\texttt{yes} or \texttt{no}).

Depending on the options \opt{theoremstyleimp} and \opt{theoremstyleunimp} the theoremstyles are declared.

\begin{macro}{\edu_theoremstyleimp:n}[1]{<numbered yes|no>}
\begin{macro}{\edu_theoremstyleunimp:n}[1]{<numbered yes|no>}
\begin{MacroCode}{class}
\cs_new:Npn \edu_theoremstyleimp:n #1 {
	\declaretheoremstyle[%
	  spaceabove=\g__edu_thmbeforeskip_skip, spacebelow=\g__edu_thmafterskip_skip,
	  headfont=\g_edu_thmimplabelstyle_tl\color{\g_edu_thmlabelfg_tl},
	  notefont=\g_edu_thmimpnotestyle_tl, notebraces={\!:\hspace{0.5em}}{},
	  bodyfont=\g_edu_thmimpbodystyle_tl,
	  headpunct={},
	  postheadspace=0.75em,
	  numbered=#1
	]{important}
}

\cs_new:Npn \edu_theoremstyleunimp:n #1 {
	\declaretheoremstyle[%
	  spaceabove=\g__edu_thmbeforeskip_skip, spacebelow=\g__edu_thmafterskip_skip,
	  headfont=\g_edu_thmunimplabelstyle_tl\color{\g_edu_thmlabelfg_tl},
	  notefont=\g_edu_thmunimpnotestyle_tl, notebraces={\hspace{0.2em}(}{)},
	  bodyfont=\g_edu_thmunimpbodystyle_tl,
	  headpunct={:},
	  postheadspace=0.25em,
	  numbered=#1
	]{unimportant}
}

\bool_if:NTF \g_edu_thmimpnumbered_bool {
  \edu_theoremstyleimp:n {yes}
}{
  \edu_theoremstyleimp:n {no}
}

\bool_if:NTF \g_edu_thmunimpnumbered_bool {
  \edu_theoremstyleunimp:n {yes}
}{
  \edu_theoremstyleunimp:n {no}
}

\end{MacroCode}
\end{macro}
\end{macro}


\begin{environment}{theorem}[1]{[<theorem name>]}
\begin{environment}{definition}[1]{[<theorem name>]}
\begin{environment}{defitheo}[1]{[<theorem name>]}
\begin{environment}{example}[1]{[<theorem name>]}
\begin{environment}{exampleexe}[1]{[<theorem name>]}
\begin{environment}{hint}[1]{[<theorem name>]}
\begin{environment}{remark}[1]{[<theorem name>]}
\begin{environment}{solution}[1]{[<theorem name>]}
Dann werden die Theoremumgebungen definiert. Zuerst die Standard-Umgebungen vom \pkg{amsthm}:

\begin{MacroCode}{class}
\bool_if:NT \g_edu_amsthm_bool {
  \seq_map_inline:Nn \g__edu_thmimp_seq {
    \declaretheorem[style=important, name=\use:c {g_edu_thm #1 label_tl}]{#1}
  }
  
  \seq_map_inline:Nn \g__edu_thmunimp_seq {
    \declaretheorem[style=unimportant, name=\use:c {g_edu_thm #1 label_tl}]{#1}
  }
}

\end{MacroCode}
\end{environment}
\end{environment}
\end{environment}
\end{environment}
\end{environment}
\end{environment}
\end{environment}
\end{environment}



\begin{macro}{\edu_framedthmimp:nx}[2]{<theoremname>}{<sharenumber-option>}
\begin{macro}{\edu_framedthmunimp:nx}[2]{<theoremname>}{<sharenumber-option>}
Nun werden die Theoremugebungen mit Hintergrund und/oder Rahmen (\pkg{thmtools} verwendet hierzu \pkg{shadethm}) definiert. Es wird mit einem Wrapper gearbeitet, um Abstände anzupassen.

Two control sequences to build \cs{declaretheorem} for framed/shared theorems. \texttt{<theoremname>} should be one of the names of \cs{__edu_thmimp}/\cs{__edu_thmunimp}. \texttt{<sharenumber-option>} should be \texttt{sharenumber=X,}, where X can be replaced by a theorem name.

After that, the control sequences are called for all names within \cs{__edu_thmimp}/\cs{__edu_thmunimp}.
\begin{MacroCode}{class}

\bool_if:NT \g_edu_framedthm_bool {
  
  \cs_new:Npn \edu_framedthmimp:nx #1 #2 {
    \declaretheorem[%
	    style=important,
	    name=\use:c {g_edu_thm #1 label_tl},
	    #2
	    shaded={%
	      bgcolor=\g_edu_thmframebg_tl,%
	      textwidth=\linewidth-1em-2pt,%
	      margin=0.5em,%
	      leftmargin=0em,%
	      rightmargin=0em,%
	      rulecolor=\g_edu_thmframefg_tl,%
	      rulewidth=1pt
	    },
	    preheadhook=,
	    postheadhook={%
	      \setpar%
	      \bool_if:NT \g_edu_parskip_bool {
	        \vspace{-0.8\parskip}%
	      }
	    }
    ]{#1fthm}
  }
  
  \cs_new:Npn \edu_framedthmunimp:nx #1 #2 {
    \declaretheorem[%
	    style=unimportant,
	    name=\use:c {g_edu_thm #1 label_tl},
	    #2
	    shaded={%
	      bgcolor=\g_edu_thmframebg_tl,%
	      textwidth=\linewidth-1em-2pt,%
	      margin=0.5em,%
	      leftmargin=0em,%
	      rightmargin=0em,%
	      rulecolor=\g_edu_thmframefg_tl,%
	      rulewidth=1pt
	    },
	    preheadhook=,
	    postheadhook={%
	      \setpar%
	      \bool_if:NT \g_edu_parskip_bool {
	        \vspace{-0.8\parskip}%
	      }
	    }
    ]{#1fthm}
  }
  
  
  \seq_map_inline:Nn \g__edu_thmimp_seq {
    
    \bool_if:NTF \g_edu_thmimpnumbered_bool {
      \edu_framedthmimp:nx {#1}{sharenumber=#1,}
    }{
      \edu_framedthmimp:nx {#1}{}
    }
  }
  
  \seq_map_inline:Nn \g__edu_thmunimp_seq {
    
    \bool_if:NTF \g_edu_thmunimpnumbered_bool {
      \edu_framedthmunimp:nx {#1}{sharenumber=#1,}
    }{
      \edu_framedthmunimp:nx {#1}{}
    }
  }
}

\end{MacroCode}
\end{macro}
\end{macro}



\begin{environment}{theoremfthm}[1]{[<theorem name>]}
\begin{environment}{definitionfthm}[1]{[<theorem name>]}
\begin{environment}{defitheofthm}[1]{[<theorem name>]}
\begin{environment}{examplefthm}[1]{[<theorem name>]}
\begin{environment}{exampleexefthm}[1]{[<theorem name>]}
\begin{environment}{hintfthm}[1]{[<theorem name>]}
\begin{environment}{remarkfthm}[1]{[<theorem name>]}
\begin{environment}{solutionfthm}[1]{[<theorem name>]}
Now, the user commands are declared separately, to manipulate the spacing arount the theorems.
\begin{MacroCode}{class}

\bool_if:NT \g_edu_framedthm_bool {
  \seq_map_inline:Nn \g__edu_thmimp_seq {
    
   \DeclareDocumentEnvironment {\use:n {#1 f}} { o } {
      \vspace{-0.3\baselineskip}%
      \vspace{0.5\parskip}%
      \IfNoValueTF {##1} {
        \begin{\use:n {#1 fthm}}%
      }{
        \begin{\use:n {#1 fthm}}[##1]
      }
    }{
      \end{\use:n {#1 fthm}}%
      \vspace{-0.3\baselineskip}%
      \vspace{0.5\parskip}%
    }
  }
  
  \seq_map_inline:Nn \g__edu_thmunimp_seq {
    
    \DeclareDocumentEnvironment {\use:n {#1 f}} { o } {
      \vspace{-0.3\baselineskip}%
      \vspace{0.5\parskip}%
      \IfNoValueTF {##1} {
        \begin{\use:n {#1 fthm}}%
      }{
        \begin{\use:n {#1 fthm}}[##1]
      }
    }{
      \end{\use:n {#1 fthm}}%
      \vspace{-0.3\baselineskip}%
      \vspace{0.5\parskip}%
    }
  }  
}

\end{MacroCode}
\end{environment}
\end{environment}
\end{environment}
\end{environment}
\end{environment}
\end{environment}
\end{environment}
\end{environment}


\begin{environment}{theorembs}[1]{[<theorem name>]}
\begin{environment}{theorembm}[1]{[<theorem name>]}
\begin{environment}{theorembl}[1]{[<theorem name>]}
\begin{environment}{definitionbs}[1]{[<theorem name>]}
\begin{environment}{definitionbm}[1]{[<theorem name>]}
\begin{environment}{definitionbl}[1]{[<theorem name>]}
\begin{environment}{defitheobs}[1]{[<theorem name>]}
\begin{environment}{defitheobm}[1]{[<theorem name>]}
\begin{environment}{defitheobl}[1]{[<theorem name>]}
\begin{environment}{examplebs}[1]{[<theorem name>]}
\begin{environment}{examplebm}[1]{[<theorem name>]}
\begin{environment}{examplebl}[1]{[<theorem name>]}
\begin{environment}{exampleexebs}[1]{[<theorem name>]}
\begin{environment}{exampleexebm}[1]{[<theorem name>]}
\begin{environment}{exampleexebl}[1]{[<theorem name>]}
\begin{environment}{hintbs}[1]{[<theorem name>]}
\begin{environment}{hintbm}[1]{[<theorem name>]}
\begin{environment}{hintbl}[1]{[<theorem name>]}
\begin{environment}{remarkbs}[1]{[<theorem name>]}
\begin{environment}{remarkbm}[1]{[<theorem name>]}
\begin{environment}{remarkbl}[1]{[<theorem name>]}
\begin{environment}{solutionbs}[1]{[<theorem name>]}
\begin{environment}{solutionbm}[1]{[<theorem name>]}
\begin{environment}{solutionbl}[1]{[<theorem name>]}
Es folgen die Theoremugebungen durch \pkg{thmbox} in allen Varianten. Seltsamerweise muss der Wert von \cs{thmbox@leftmargin} neu gesetzt werden, ansonsten kommt es zu Komplikationen bei \texttt{parskip=true}:
\begin{MacroCode}{class}
\bool_if:NT \g_edu_thmbox_bool {
  
  \seq_map_inline:Nn \g__edu_thmimp_seq {
	  \declaretheorem[%
	    style=important,
	    name=\use:c {g_edu_thm #1 label_tl},
	    sharenumber=#1,
	    thmbox=S,
	    postheadhook=\setpar\hspace{-0.5em},
	    postfoothook=\setpar\parskipreduce
	  ]{#1bs}
  
	  \declaretheorem[%
	    style=important,
	    name=\use:c {g_edu_thm #1 label_tl},
	    sharenumber=#1,
	    thmbox=M,
	    postheadhook=\setpar\hspace{-0.5em},
	    postfoothook=\setpar\parskipreduce
	  ]{#1bm}
	    
	  \declaretheorem[%
	    style=important,
	    name=\use:c {g_edu_thm #1 label_tl},
	    sharenumber=#1,
	    thmbox=L,
	    postheadhook=\setpar\hspace{-0.5em},
	    postfoothook=\setpar\parskipreduce
	  ]{#1bl}
  }
  
  \seq_map_inline:Nn \g__edu_thmunimp_seq {
	 \declaretheorem[%
	    style=unimportant,
	    name=\use:c {g_edu_thm #1 label_tl},
	    sharenumber=#1,
	    thmbox=S,
	    postheadhook=\setpar\hspace{-0.5em},
	    postfoothook=\setpar\parskipreduce
	  ]{#1bs}
	  
	 \declaretheorem[%
	    style=unimportant,
	    name=\use:c {g_edu_thm #1 label_tl},
	    sharenumber=#1,
	    thmbox=M,
	    postheadhook=\setpar\hspace{-0.5em},
	    postfoothook=\setpar\parskipreduce
	  ]{#1bm}
	  
	 \declaretheorem[%
	    style=unimportant,
	    name=\use:c {g_edu_thm #1 label_tl},
	    sharenumber=#1,
	    thmbox=L,
	    postheadhook=\setpar\hspace{-0.5em},
	    postfoothook=\setpar\parskipreduce
	  ]{#1bl}
  }
  
  \setlength{\thmbox@leftmargin}{1.5em}

}%

\end{MacroCode}
\end{environment}
\end{environment}
\end{environment}
\end{environment}
\end{environment}
\end{environment}
\end{environment}
\end{environment}
\end{environment}
\end{environment}
\end{environment}
\end{environment}
\end{environment}
\end{environment}
\end{environment}
\end{environment}
\end{environment}
\end{environment}
\end{environment}
\end{environment}
\end{environment}
\end{environment}
\end{environment}
\end{environment}



\subsubsection{Symbole für spezielle Mengen}

\begin{macro}{\N}
\begin{macro}{\Z}
\begin{macro}{\Q}
\begin{macro}{\R}
\begin{macro}{\I}
\begin{macro}{\C}
\begin{macro}{\L}
Definiert Symbole für spezielle Mengen.
\begin{MacroCode}{class}
\DeclareDocumentCommand \N { } {\ensuremath{\mathbb{N}}}
\DeclareDocumentCommand \Z { } {\ensuremath{\mathbb{Z}}}
\DeclareDocumentCommand \Q { } {\ensuremath{\mathbb{Q}}}
\DeclareDocumentCommand \R { } {\ensuremath{\mathbb{R}}}
\DeclareDocumentCommand \I { } {\ensuremath{\mathbb{I}}}
\DeclareDocumentCommand \C { } {\ensuremath{\mathbb{C}}}
\DeclareDocumentCommand \L { } {\ensuremath{\mathbb{L}}}

\end{MacroCode}
\end{macro}
\end{macro}
\end{macro}
\end{macro}
\end{macro}
\end{macro}
\end{macro}


\subsubsection{Vektoren}

\begin{macro}{\vec}[1]{<expression>}
Anderer Name für einen Vektor markiert durch einen Pfeil, der durch \cs*{vv} aus dem Package \pkg{esvect} erzeugt wird.
\begin{MacroCode}{class}
\DeclareDocumentCommand \vec { m } {\vv{#1}}

\end{MacroCode}
\end{macro}


\begin{macro}{\vect}[1]{<expressions in matrix-syntax>}
Verkürzte Erzeugung eines Spaltenvektors durch \env*{pmatrix}-Umgebung.
\begin{MacroCode}{class}
\DeclareDocumentCommand \vect { m } {\begin{pmatrix} #1 \end{pmatrix}}

\end{MacroCode}
\end{macro}


\subsubsection{Gleichungssysteme/Gauß-Verfahren}

In \pkg{gauss} sollen die Zeilenumformungen angezeigt werden:

\begin{MacroCode}{class}
\DeclareDocumentCommand \rowswapfromlabel { m } {#1}
\DeclareDocumentCommand \rowswaptolabel { m } {#1}

\end{MacroCode}

\begin{environment}{gmatrix*}[1]{[<arraycolsep>]}
\begin{environment}{gmatrixp*}[1]{[<arraycolsep>]}
\begin{environment}{gmatrixv*}[1]{[<arraycolsep>]}
Declare starred versions of the \pkg{gauss}-environments with optional \cs{arraycolsep}.
\begin{MacroCode}{class}
\DeclareDocumentEnvironment {gmatrix*} { O{2pt} } {%
  \setlength{\arraycolsep}{#1}
  \begin{gmatrix}%
}{%
  \end{gmatrix}
}

\DeclareDocumentEnvironment {gmatrixp*} { O{4pt} } {%
  \setlength{\arraycolsep}{#1}
  \begin{gmatrix}[p]%
}{%
  \end{gmatrix}
}

\DeclareDocumentEnvironment {gmatrixv*} { O{4pt} } {%
  \setlength{\arraycolsep}{#1}
  \begin{gmatrix}[v]%
}{%
  \end{gmatrix}
}

\end{MacroCode}
\end{environment}
\end{environment}
\end{environment}

\begin{macro}{\vect}[1]{<expressions in matrix-syntax>}
This command can be used to typeset a horizontal rule inside an array which spans over the whole height of the cell.
\begin{MacroCode}{class}
\DeclareDocumentCommand \mvsep { } {
  % From localghost@golatex
  \hspace{0.25em}\kern-\tabcolsep\vrule height\arraystretch\ht\strutbox depth\arraystretch\dp\strutbox\kern-\tabcolsep\hspace{0.25em}
} 

\end{MacroCode}
\end{macro}


\subsubsection{Schriftliche Grundrechenarten}

Set up \textsf{xlop} to look as much as possible like the german school notation.

\begin{MacroCode}{class}
\opset{voperation=top}                % Vertikale Ausrichtung der Rechnung
\opset{voperator=bottom}              % Rechenzeichen in unterer Zeile
\opset{carrysub=true}                      % Übertrag bei Subtraktion
\opset{decimalsepsymbol={,}}          % Dezimaltrenner
\opset{shiftintermediarysymbol=0}     % Mult: Nullen zum Auffüllen
\opset{displayshiftintermediary=all}  % Mult: Zeilen mit Nullen auffüllen
\opset{displayintermediary=all}       % Mult: Auch Nullzeilen anzeigen
\opset{shiftdecimalsep=divisor}       % Div: Nur Nachkommastellen des Divisiors beseitigen

\end{MacroCode}




\subsubsection{Polynomdivision}

Polynomdivision wird durch \textsf{polynom} durchgeführt. Dieses Package wird an dieser Stelle konfiguriert.

\begin{MacroCode}{class}
\polyset{style=C, div=:}

\end{MacroCode}

\subsubsection{Typographie}

\begin{macro}{\qtext}[1]{<text>}
\begin{macro}{\qqtext}[1]{<text>}
\begin{macro}{\qund}
\begin{macro}{\qqund}
\begin{macro}{\qoder}
\begin{macro}{\qqoder}
\begin{macro}{\qmath}[1]{<expression>}
\begin{macro}{\qqmath}[1]{<expression>}
\begin{macro}{\qRightarrow}
\begin{macro}{\qrightarrow}
\begin{macro}{\qLeftarrow}
\begin{macro}{\qleftarrow}
\begin{macro}{\qLeftrightarrow}
\begin{macro}{\qleftrightarrow}
\begin{macro}{\qqRightarrow}
\begin{macro}{\qqrightarrow}
\begin{macro}{\qqLeftarrow}
\begin{macro}{\qqleftarrow}
\begin{macro}{\qqLeftrightarrow}
\begin{macro}{\qqleftrightarrow}
Einfügen von Text/Formeln mit beidseitigem Abstand \cs*{quad} bzw. \cs{qquad} im Mathemodus.
\begin{MacroCode}{class}
\DeclareDocumentCommand \qtext { m } {\ensuremath{\quad\text{#1}\quad}}
\DeclareDocumentCommand \qqtext { m } {\ensuremath{\qquad\text{#1}\qquad}}

\DeclareDocumentCommand \qund {  } {\qtext{und}}
\DeclareDocumentCommand \qqund {  } {\qqtext{und}}

\DeclareDocumentCommand \qoder {  } {\qtext{oder}}
\DeclareDocumentCommand \qqoder {  } {\qqtext{oder}}

\DeclareDocumentCommand \qmath { m } {\ensuremath{\quad #1 \quad}}
\DeclareDocumentCommand \qqmath { m } {\ensuremath{\qquad #1 \qquad}}

\DeclareDocumentCommand \qRightarrow {  } {\qmath{\Rightarrow}}
\DeclareDocumentCommand \qrightarrow {  } {\qmath{\rightarrow}}
\DeclareDocumentCommand \qLeftarrow {  } {\qmath{\Leftarrow}}
\DeclareDocumentCommand \qleftarrow {  } {\qmath{\lefttarrow}}
\DeclareDocumentCommand \qLeftrightarrow {  } {\qmath{\Leftrightarrow}}
\DeclareDocumentCommand \qleftrightarrow {  } {\qmath{\leftrightarrow}}

\DeclareDocumentCommand \qqRightarrow {  } {\qqmath{\Rightarrow}}
\DeclareDocumentCommand \qqrightarrow {  } {\qqmath{\rightarrow}}
\DeclareDocumentCommand \qqLeftarrow {  } {\qqmath{\Leftarrow}}
\DeclareDocumentCommand \qqleftarrow {  } {\qqmath{\lefttarrow}}
\DeclareDocumentCommand \qqLeftrightarrow {  } {\qqmath{\Leftrightarrow}}
\DeclareDocumentCommand \qqleftrightarrow {  } {\qqmath{\leftrightarrow}}

\end{MacroCode}
\end{macro}
\end{macro}
\end{macro}
\end{macro}
\end{macro}
\end{macro}
\end{macro}
\end{macro}
\end{macro}
\end{macro}
\end{macro}
\end{macro}
\end{macro}
\end{macro}
\end{macro}
\end{macro}
\end{macro}
\end{macro}
\end{macro}
\end{macro}


\subsubsection{Betrag und Norm -- auch von Vektoren}

\begin{macro}{\abs}[1]{<expression>}
\begin{macro}{\absvec}[1]{<expression>}
\begin{macro}{\abs*}[1]{<expression>}
\begin{macro}{\absvec*}[1]{<expression>}
\begin{macro}{\norm}[1]{<expression>}
\begin{macro}{\normvec}[1]{<expression>}
\begin{macro}{\norm*}[1]{<expression>}
\begin{macro}{\normvec*}[1]{<expression>}
Makros für (Vektor-)Beträge und (Vektor-)Normen. In den Sternvarianten skalieren die Klammern nicht.
\begin{MacroCode}{class}
\DeclareDocumentCommand \abs { s m } {
  \IfBooleanTF{#1} {
    \ensuremath{\lvert #2 \rvert}
  }{
    \ensuremath{\left| #2 \right|}
  }
}

\DeclareDocumentCommand \absvec { s m } {
  \IfBooleanTF{#1} {
    \ensuremath{\abs*{\vec{#2}}}
  }{
    \ensuremath{\abs{\vec{#2}}}
  }
}

\DeclareDocumentCommand \norm { s m } {
  \IfBooleanTF{#1} {
    \ensuremath{\lVert #2 \rVert}
  }{
    \ensuremath{\abs*{\vec{#2}}}
  }
}

\DeclareDocumentCommand \normvec { s m } {
  \IfBooleanTF{#1} {
    \ensuremath{\norm*{\vec{#2}}}
  }{
    \ensuremath{\norm{\vec{#2}}}
  }

}

\end{MacroCode}
\end{macro}
\end{macro}
\end{macro}
\end{macro}
\end{macro}
\end{macro}
\end{macro}
\end{macro}


\subsubsection{Special Equations/Formulas}

\begin{macro}{\qe}
\begin{macro}{\qevar}
Special equations/formulas:
\begin{MacroCode}{class}
\DeclareDocumentCommand \qe { m m } {
  \ensuremath -\frac{#1}{2} \pm \sqrt{\left(\frac{#1}{2}\right)^2 - #2}
}

\DeclareDocumentCommand \qevar { m m } {
  \ensuremath #1 \pm \sqrt{#2}
}

\end{MacroCode}
\end{macro}
\end{macro}




\subsubsection{Verschiedenes}

\begin{macro}{\corr}
\begin{macro}{\ds}
\begin{macro}{\der}
\begin{macro}{\i}
\begin{macro}{\minusp}
\begin{macro}{\sep}
\begin{macro}{\solset}
\begin{macro}{\textlightning}
Sonstige Symbole, Konstanten, Abkürzungen etc. Selbsterklärend.
\begin{MacroCode}{class}
\DeclareDocumentCommand \corr {  } {\ensuremath{\mathrel{\hat{=}}}}
\DeclareDocumentCommand \ds {  } {\ensuremath{\displaystyle}}
\DeclareDocumentCommand \der {  } {\ensuremath{\ \mathrm{d}}}
\DeclareDocumentCommand \i {  } {\ensuremath{\mathrm{i}}}
\DeclareDocumentCommand \minusp {  } {\ensuremath{\hphantom{-}}}
\DeclareDocumentCommand \sep {  } {\,\vert\,}    
\DeclareDocumentCommand \solset { m } {\ensuremath \mathbb{L} = \left\lbrace #1 \right\rbrace}
\DeclareDocumentCommand \textlightning {  } {\ensuremath{\lightning}}

\end{MacroCode}
\end{macro}
\end{macro}
\end{macro}
\end{macro}
\end{macro}
\end{macro}
\end{macro}
\end{macro}


\subsection{Informatik}

\subsubsection{\textsf{listings}}

Zuerst werden Styles für abgesetzte Listings und Inline-Listings definiert:

\begin{macro*}{\g_edu_lstbelowskip}
\begin{MacroCode}{class}
\dim_new:N \g_edu_lstbelowskip
\dim_gset:Nn \g_edu_lstbelowskip {0.2\baselineskip-0.5\parskip}

\lstdefinestyle{lststyle}{%
  aboveskip=0.75\baselineskip,
  belowskip=\g_edu_lstbelowskip,
  language=Java, 
  basicstyle=\ttfamily\footnotesize, 
  keywordstyle=\color{\g_edu_lstnumberfg_tl}\bfseries, 
  stringstyle=\emph, 
  numberstyle=\color{\g_edu_lstnumberfg_tl}\ttfamily\scriptsize,
  numbers=left, 
  numbersep=8pt, 
  frame=trbl, 
  framesep=0pt,
  framexleftmargin=2pt,
  framexrightmargin=2pt,
  framerule=0.7pt,
  rulecolor=\color{\g_edu_lstrulefg_tl},
  captionpos=b, 
  tabsize=2, 
  showstringspaces=false, 
  xleftmargin=2.5em,
  xrightmargin=0cm,
  breaklines=true,
  prebreak={\,\,\Pisymbol{psy}{191}},
%  backgroundcolor=\color{wuLightGray},
  escapeinside={/@}{@/},
  lineskip=1pt
}

\lstdefinestyle{lstistyle}{%
  language=Java, 
  basicstyle=\ttfamily, 
  keywordstyle=\color{\g_edu_lstnumberfg_tl}\bfseries, 
  stringstyle=\emph, 
  breaklines=true,
  prebreak={\,\,\Pisymbol{psy}{191}}
}

\lstset{%
  style=lststyle
}

% Support for Umlaute
\lstset{literate=%
    {Ö}{{\"O}}1
    {Ä}{{\"A}}1
    {Ü}{{\"U}}1
    {ß}{{\ss}}1
    {ü}{{\"u}}1
    {ä}{{\"a}}1
    {ö}{{\"o}}1
}

\end{MacroCode}
\end{macro*}


\subsection{Sonstiges}

Various additional common commands.

\begin{macro}{\templength}
\begin{macro}{tempcounter}
Length/counter for temporary usage.
\begin{MacroCode}{class}
\newlength{\templength}
\newcounter{tempcounter}

\end{MacroCode}
\end{macro}
\end{macro}


\begin{MacroCode}{class}
\ExplSyntaxOff

\end{MacroCode}


%  \begin{thebibliography}{mm}
%    \bibitem{cancel} \textsc{Donald Arseneau}: \emph{cancel}. \url{http://www.ctan.org/pkg/cancel}.
%    \bibitem{ulsy} \textsc{Ulrich Goldschmitt}: \emph{ulsy}. \url{http://www.ctan.org/pkg/ulsy}.
%    \bibitem{booktabs} \textsc{Simon Fear} und \textsc{Danie Els} \emph{listings}. \url{http://www.ctan.org/pkg/booktabs}.
%    \bibitem{polynom} \textsc{Carsten Heinz}: \emph{polynom}. \url{http://www.ctan.org/pkg/polynom}.
%    \bibitem{listings} \textsc{Carsten Heinz} und \textsc{Brooks Moses} \emph{listings}. \url{http://www.ctan.org/pkg/listings}.
%    \bibitem{koma} \textsc{Markus Kohm}: \emph{KOMA-Script}. \url{http://www.ctan.org/pkg/koma-script}.
%    \bibitem{pdfpages} \textsc{Andreas Matthias}: \emph{pdfpages}. \url{http://www.ctan.org/pkg/pdfpages}.
%    \bibitem{multicol} \textsc{Frank Mittelbach}: \emph{multicol}. \url{http://www.ctan.org/pkg/multicol}.
%    \bibitem{lato} \textsc{Mohamed El Morabity}: \emph{lato}. \url{http://www.ctan.org/pkg/lato}.
%    \bibitem{units} \textsc{Axel Reichert}: \emph{units}. \url{http://www.ctan.org/pkg/units}.
%    \bibitem{esvect} \textsc{Eddie Saudrais}: \emph{esvect}. \url{http://www.ctan.org/pkg/esvect}.
%    \bibitem{icomma} \textsc{Walter Schmidt}: \emph{icomma}. \url{http://www.ctan.org/pkg/icomma}.
%    \bibitem{fonts} \textsc{Walter Schmidt}: \emph{psnfss}. \url{http://www.ctan.org/pkg/pifont}.
%    \bibitem{tikz} \textsc{Till Tantau}: \emph{tikz}. \url{http://www.texample.net/tikz/}.
%    \bibitem{eurosym} \textsc{Henrik Theiling}: \emph{eurosym}. \url{http://www.ctan.org/pkg/eurosym}.
%  \end{thebibliography}



%% Die folgende Umgebung ist eine Abkürzung für die |listing|-Umgebung.
%% \begin{environment}{lst}
%%    \begin{macrocode}
%\lstnewenvironment{lst}{}{}
%      
%%    \end{macrocode}
%% \end{environment}
%% 
%% \begin{macro}{\lsti}
%% \begin{macro}{\lstiv}
%% \begin{macro}{\lstib}
%% 
%% Damit abgesetzte Listings und Inline-Listungs in unterschiedlichen 
%% Schriftgrößen gesetzt werden, sollte eines der folgenden Makros für 
%% Inline-Listings verwendet werde:
%%    \begin{macrocode}
%\newcommand{\lsti}[1]{\lstinline[style=lstistyle]~#1~}  % [ Highlight-escape
%\newcommand{\lstiv}[1]{\lstinline[style=lstistyle]!#1!}  % [ Highlight-escape
%\newcommand{\lstib}[1]{\lstinline[style=lstistyle]{#1}}  % [ Highlight-escape
%      
%%    \end{macrocode}
%% \end{macro}
%% \end{macro}
%% \end{macro}





    \Finale
%    \section{Installation}
%    The easiest way to install this package is using the package
%    manager provided by your \LaTeX\ installation if such a program
%    is available. Failing that, provided you have obtained the package
%    source (\file{skrapport.tex} and \file{Makefile}) from either CTAN
%    or Github, running \texttt{make install} inside the source directory
%    works well. This will extract the documentation and code from
%    \file{skrapport.tex}, install all files into the TDS tree at
%    \texttt{TEXMFHOME} and run \texttt{mktexlsr}.
%
%    If you want to extract code and documentation without installing
%    the package, run \texttt{make all} instead. If you insist on not
%    using \texttt{make}, remember that packages distributed using
%    \pkg{skdoc} must be extracted using \texttt{pdflatex}, \emph{not}
%    \texttt{tex} or \texttt{latex}.

    \PrintChanges
    
    Index
    \PrintIndex
    \printbibliography
\end{document}
